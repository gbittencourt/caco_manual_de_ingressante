\section{Siglas!}

Antes de qualquer coisa, você vai logo perceber que, dentro da Unicamp, existem um zilhão de siglas diferentes- Não se preocupe em decorar tudo!!! Com o tempo, você vai se familiarizar bem com elas, prometo!

Separamos, nessa sessão, explicações sobre algumas siglas, códigos e outros termos que são muito usados dentro da Unicamp- Assim, enquanto estiver lendo o manual ou vivendo sua vida de ingressante, terá em mãos um glossário para consultar! Se houverem mais dúvidas ou siglas que não constam aqui, não hesite em perguntar a sues veterenes ou a nós do Centro Acadêmico. ;)

\begin{itemize}
    \item \textbf{Currículo pleno}: é o conjunto de disciplinas do curso que e alune tem que cursar para se formar.
    \item \textbf{Créditos}: unidade elementar de horas-aula de qualquer curso da Unicamp. Um crédito equivale a uma hora-aula semanal, ou a 15 horas-aula semestrais. Cada disciplina possui um “número de créditos” equivalente a ela- Normalmente, esse número de créditos indica quantas horas semanais você terá de aulas dessa disciplina- Por exemplo: cálculo 1 é uma matéria de 6 créditos; você terá, portanto, 3 aulas de 2 horas por semana (ou 6 horas-aula por semana). A cada semestre, você irá acumulando créditos. Para se formar no curso, você precisa completar um número específico de créditos (explicitados no currículo pleno).
    \item \textbf{CR (Coeficiente de Rendimento)}: valor entre 0 e 1 da média das notas em todas as disciplinas cursadas ponderada pelos créditos (ou seja, ir mal numa matéria de 6 créditos pode prejudicar muito seu CR). PS: atenção redobrada se sua intenção é seguir carreira acadêmica e/ou intercâmbio (mais detalhes sobre isso à frente).
    \item \textbf{CP (Coeficiente de Progressão)}: parte do curso que você já cumpriu (CP = 0,6123 significa que se cumpriu 61,23\% do curso).
    \item \textbf{CPF (Coeficiente de Progressão Futuro)}: é o CP que você terá no fim do semestre caso passe em todas as disciplinas que você se matriculou.
    \item \textbf{CPE (Coeficiente de Progressão Exigido)}: usado para fins de cancelamento de matrícula. Para que o aluno possa continuar a fazer o curso, ele precisa ter um CP maior ou igual ao CPE daquele semestre.
    \item \textbf{GDE}: incrível facilitador de vidas! É um site, e também rede social, onde você pode usar seu login da DAC para navegar entre disciplinas, salas de aula e listas de colegas. O Planejador e a avaliação de professores (extra-oficial) são a principal ferramenta. Projeto de Felipe Guaycuru, um ex-aluno da Engenharia de Computação, reconhecido pela DAC. O link é \textbf{\url{gde.ir}}.
    \item \textbf{Pré-requisito}: matéria(s) que precisa(m) ter sido cursada(s) para que se possa fazer outra(s) matéria(s). Existem dois tipos de pré-requisitos: Os pré-requisitos totais, mais comuns, do qual é exigido tanto a aprovação por nota como por frequência e os pré-requisitos parciais, mais raros, do qual o aluno não precisa ter sido aprovado por nota, mas tem que ter tido aprovação por frequência e nota final maior ou igual a 3,0. Os pré-requisitos parciais são identificados com um asterisco na frente do código da disciplina.
    \item \textbf{AA4xy}: um tipo de pré-requisito. Não se trata de nenhuma disciplina. Para fazer disciplinas com esse pré-requisito, e alune tem que tem um CP maior ou igual a 0,xy.
    \item \textbf{AA200}: outro tipo de pré-requisito existente, mais presente em disciplinas eletivas. Também não se trata de nenhuma disciplina. É apenas uma autorização da coordenadoria do curso. Se sobrar vagas para a disciplina e a coordenadoria do curso for com a sua cara, você faz a disciplina.
    \item \textbf{CCG (Comissão Central de Graduação)}: órgão colegiado da Unicamp, é encarregada da orientação, supervisão e revisão periódica do ensino na Universidade. Cabe recurso à CCG de quaisquer decisões das Unidades afetando o ensino.
\end{itemize}