\section{Centro de Ensino de Línguas - CEL}

O CEL, como o nome já diz, é o órgão responsável por oferecer aulas de diferentes  idiomas a alunes da Unicamp. Seja como disciplinas obrigatórias (para a engenharia de  computação, Inglês Instrumental I é uma delas) ou como eletivas. O CEL possui turmas  de inglês, espanhol, alemão, francês, japonês, dentre outras línguas.

Todas as línguas no CEL oferecem níveis diferentes; assim, se você souber um pouco  de Francês, não precisa começar do início. É só realizar o teste de proficiência, oferecido  no primeiro semestre de cada ano, e você pode avançar algumas turmas (ou todas) em  uma determinada língua. Também é possível fazer o teste de proficiência pra eliminar  disciplinas (o que a maioria das pessoas acaba fazendo com Inglês Instrumental). Todes  es alunes que quiserem se inscrever em \textbf{Inglês I (LA112)} ou \textbf{Japonês I (LA111)} devem  fazer o teste, mesmo que não saibam nada da língua; mas ele não é necessário para  matrículas nas outras línguas. 

A matrícula em disciplinas do CEL é feita de maneira natural, na DAC, durante o  período de matrícula. 
