\section{Pra quê estudar isso?}

O ensino médio acabou, você finalmente está livre de todas as “inutilidades”, como  química orgânica e separação silábica de verbos parnasianos, só vai ver coisas  relevantes para a profissão, e… Pimba! Uma disciplina que você acha completamente  nada a ver. Pode isso? A regra é clara, pode sim. 

Primeiro, você precisa saber que a Universidade não é um curso técnico. A ideia não é  só te dar capacitação profissional, mas sim formar pessoas melhores! Para que ume computeire precisa de contabilidade? Aparentemente para nada, mas  uma pessoa precisa ter uma noção disso, sobretudo de exatas. Outra coisa: o que  exatamente é “relevante para a sua profissão”? A computação é uma área muito vasta, e  a graduação é muito generalista para te dar base para escolher. Nosso conhecimento  pode ser aplicado em quase qualquer outra área do conhecimento com alguma dose  sucesso. Se você continuar na universidade, na pós-graduação, você só terá matérias da  sua área, já que você já sabe o suficiente pra dizer que área é essa. Mas ainda falta  muito chão até lá… Respira! 

Para quem é da Engenharia, para conseguir o CREA existem algumas matérias  obrigatórias, como Resistência dos Materiais. As coordenadorias da engenharia têm  lutado para diminuir créditos obrigatórios e aumentar eletivos – mas há matérias em  que es professores dificilmente concordariam em alterar.  

Tanto para a ciência quanto para a engenharia, o curso não é para formar simples  programadores. Vocês serão mais que isso: serão cientistas, engenheires; e isso envolve  ver coisas além de computação.