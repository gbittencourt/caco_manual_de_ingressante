\section{Licenças de Software Gratuitas}

Estar na Unicamp nos dá acesso a várias licenças gratuitas de software proprietário  especiais para estudantes. 

O \textbf{LMS (Laboratório Microsoft)} ligado ao Instituto de Computação oferece download  gratuito de uma grande variedade de software da Microsoft, como o sistema Windows e  o Visual Studio. Para mais informações sobre como se cadastrar e baixar, acesse com seu  login do IC: \url{www.lms.ic.unicamp.br}. 

A \textbf{CTIC (Coordenadoria de Tecnologia da Informação e Comunicação)} obtém licenças  de muitos pacotes de software e disponibiliza para a comunidade acadêmica, inclusive  alunes. As aplicações de computação científica \textit{Wolfram Mathematica} e \textit{MATLAB} estão  disponíveis gratuitamente. Para mais informações, acesse: \url{www.ctic.unicamp.br/softwares}. 

A multinacional \textbf{Autodesk} oferece licenças de estudante gratuitas para muitos de  seus programas, incluindo o \textit{AutoCAD} e a aplicação de modelagem 3D \textit{Maya}. Para baixar,  cadastre-se usando seu e-mail da DAC, do IC ou da FEEC em \url{students.autodesk.com}. 

A \textbf{JetBrains}, empresa que produz IDEs muito boas, aprova em menos de 5 minutos os  pedidos para conta estudantil que possuem e-mail da Unicamp, o que permite o uso de  programas pagos de graça. Cadastre o e-mail do IC em \url{www.jetbrains.com/student}.

\subsection{Conta educacional no GitHub}

Além de ter acesso a alguns softwares, estudar na Unicamp também nos dá acesso a contas especiais em alguns sites, como o Github. Não se assuste com esse nome estranho, em breve você descobrirá a magia que é um sistema de controle de versão e que o git é um dos melhores deles. Além da ferramenta git por si só ser muito útil, existe o \textit{GitHub}, que disponibiliza repositórios remotos, além de uma boa forma de contribuir para projetos existentes.

Um pequeno problema do \textit{GitHub} é que em sua conta gratuita, só podemos criar  repositórios públicos, o que não combina muito com trabalhos de faculdade. O \textit{GitHub}  disponibiliza, no entanto, um \textit{Student Developer Pack}: dentre os diversos benefícios,  temos infinitos repositórios privados, descontos em serviços como a AWS, acesso às  ferramentas da \textit{Unreal Engine}, etc.

Para conseguir isso, basta utilizar seu e-mail acadêmico ao se inscrever pelo link \textbf{\url{education.github.com}}.

\subsection{Google Apps for Education}

Em 2016, a Unicamp firmou um acordo com a \textbf{Google} para disponibilizar o Google  Apps for Education para estudantes, docentes e funcionáries. Logo no começo das aulas,  você vai receber uma conta que pode acessar em \textbf{\url{googleapps.unicamp.br}}; seu login será  seu RA, e sua senha, a mesma que você usa para entrar no sistema da DAC.

Nessa conta, você vai ter acesso a tudo o que uma conta normal do Google tem, mas  com armazenamento ilimitado tanto no Gmail quanto no Drive! Ou seja, podemos  armazenar tudo o que quisermos sem preocupação! Além disso, é mais um email  acadêmico no formato: \textbf{<inicial do nome><RA>@g.unicamp.br} que você pode usar a bel prazer.

Você também pode usar Drives de equipe do Google Drive da Unicamp, que é algo  como uma pasta compartilhada de espaço ilimitado (em breve será limitado...). A  duração das chamadas pelo Google Meet também não tem limite, dentre outros  benefícios.