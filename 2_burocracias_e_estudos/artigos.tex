\section{Acesso a artigos e revistas científicas}

Os resultados de pesquisas científicas, no Brasil e no mundo, costumam ser  publicados por meio de periódicos e conferências, os quais normalmente podem ser acessados pela internet.

No Brasil, quase todas as instituições públicas de ensino superior, como a Unicamp,  participam de um sistema conhecido como \textbf{Portal de Periódicos da Capes (\url{periodicos.capes.gov.br})}, que garante acesso a grande parte das publicações científicas  das principais editoras do mundo sem necessidade de pagar nada a mais por isso. 

Nas áreas de engenharia e de computação, quase todas as publicações relevantes são  acessíveis através deste sistema. É importante você saber, porém, que esse tipo de  acesso só é possível a partir de endereços IP da Universidade; então, se você quiser  acessar algum artigo quando estiver em casa, o ideal é usar o sistema de acesso VPN  (Virtual Private Network) disponibilizado pela Unicamp: \url{www.ccuec.unicamp.br/ccuec/servicos/acesso_remoto_vpn}. 

Através da Comunidade Acadêmica Federada (CAFe), foi disponibilizado recentemente um método de acesso remoto aos periódicos sem necessidade de usar a VPN; confira  esse link para mais informações:
\url{www-periodicos-capes-gov-br.ezl.periodicos.capes.gov.br/index.php/acesso-cafe}.

Na Unicamp, você ainda tem acesso a diversas outras publicações e e-books que não  são cobertos pelo sistema da Capes, além de alguns periódicos impressos, que podem  ser encontrados nas bibliotecas. Caso você queira buscar algo nos acervos físico e virtual  da Universidade, acesse o site do Sistema de Bibliotecas da Unicamp: \url{www.sbu.unicamp.br}.

Para uma busca mais abrangente de artigos científicos na internet, você pode usar o  Google Acadêmico (\url{scholar.google.com}). Mas atenção! Você pode encontrar artigos que  não são cobertos pelo Portal da Capes nem pela Unicamp e que exigem pagamento.

Além do Portal de Periódicos, existe também um novo modelo de publicações  científicas de acesso gratuito, chamado \textbf{\textit{open access}}. Esse modelo tem origem muito  próxima do movimento pelo software livre. Publicações feitas nesse sistema são  acessíveis a qualquer momento, de qualquer IP e sem qualquer custo. Alguns exemplos  de grandes repositórios e editoras \textit{open access} são:

\textbf{SciELO: \url{www.scielo.org}}

\textbf{PLOS: \url{www.plos.org}}

\textbf{arXiv: \url{www.arxiv.org}}

\textbf{PMC: \url{ncbi.nlm.nih.gov/pmc}}

A rede \textbf{SciELO} é onde a maior parte dos artigos em português é publicada. O acervo  \textbf{PMC} é de publicações da área biomédica.

Existe uma grande circulação de pirataria de artigos científicos na internet. Apesar  dessa prática trazer muitas coisas boas, como:

\begin{itemize}
    \item Possibilitar muitas pesquisas que não aconteceriam se pesquisadores tivessem que  pagar valores absurdos pelo acesso a conhecimentos que deveriam ser públicos;
    \item Ajudar alunes a estudarem;
    \item Democratizar o acesso ao conhecimento.
\end{itemize}

Ela também traz uma série de questões ruins, como por exemplo:

\begin{itemize}
    \item É uma grande vergonha, por isso não somos a favor de que vocês fiquem sabendo de  sites como \url{sci-hub.hkvisa.net}, \url{libgen.is} e \url{shadowlibraries.github.io}.
\end{itemize}