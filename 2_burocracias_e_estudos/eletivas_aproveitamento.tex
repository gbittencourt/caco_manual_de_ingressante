\section{Eletivas, proficiência e aproveitamento}

A Unicamp oferece a oportunidade de personalizar seu currículo de acordo com seu interesse por meio das disciplinas eletivas. Ao contrário das disciplinas obrigatórias, com as eletivas você pode escolher a matéria que vai cursar. Alguns créditos podem ser cumpridos com qualquer disciplina oferecida pela Universidade, outros estão restritos a um determinado conjunto. Para mais detalhes, consulte seu catálogo. Você pode socializar com outros cursos, aprender uma língua estrangeira, assistir a seminários ou obter um certificado de estudos na FEEC ou no IC.

\textbf{Teste de proficiência} é uma prova que permite dispensa de cursar uma disciplina (desde que você obtenha a nota mínima, é claro). Se você acha que sabe o suficiente sobre eletromagnetismo, por exemplo, pode tentar a proficiência de Física Geral III. Nem todas as disciplinas oferecem o teste e você só pode fazê-lo uma vez por disciplina – e se você já se matriculou na disciplina e não passou, não pode fazer. Para as matérias específicas de computação, a proficiência é mais voltada para quem já estudou ou trabalhou com programação, pois requer alguma experiência. Além disso, fazer o teste de proficiência também é obrigatório para se matricular nas disciplinas de língua inglesa e japonesa, independentemente de conhecimento prévio na língua.

Fique ligade no calendário da DAC para não perder as datas de inscrição! As datas dos testes de línguas são sempre no começo do ano, diferentes das demais, que são no fim de cada semestre. 

Disciplinas eletivas e teste de proficiência estão relacionados porque muitas pessoas fazem proficiência em disciplinas de línguas – eliminando créditos de eletivas, em alguns casos para evitar o jubilamento, em outros para não ter que passar mais um semestre na faculdade. Converse com veteranes para descobrir o melhor jeito de usufruir dessa liberdade que poucas universidades oferecem! Dificilmente você não encontrará algo com o qual se identifica ou que não ensine lições interessantes.

Por fim, caso você tenha iniciado um curso de graduação antes e queira aproveitar os créditos no novo curso, existe o \textbf{aproveitamento de estudos}. Se as disciplinas foram feitas na UNICAMP esse processo é automático. Se foi em outra instituição, a partir do seu pedido, é feita análise e comparação das ementas das disciplinas. Atenção: só pode ser feita no 1o semestre!

Para mais informações sobre proficiência e aproveitamento, acesse: \url{bit.ly/2BS5O6e}.