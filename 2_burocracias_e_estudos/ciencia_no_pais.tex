\section{A atual situação da ciência no país}

Como você viu neste manual, a universidade pública tem sido atacada por todos  os lados. Na seção de Convivência, você verá os aspectos ideológicos desses ataques; na  seção Como Funciona a Universidade, verá que os ataques orçamentários tiram uma grana  que faz falta para a pesquisa e a permanência, além do projeto de entrega total e  terceirização do ensino superior através dos “Future-se” da vida. Em 2019, nossos  pesquisadores não sabiam mais se teriam bolsa - ou universidade pública - até o fim do  ano. 

Não é fácil. \textbf{A ciência e tecnologia de um povo só avançam com base em políticas  públicas sérias e investimento pesado – que retorna para o povo}. Mas esse não parece  ser o plano para nosso país, em que as obras do acelerador síncrotron quase foram  suspensas, o Museu Nacional virou cinzas e a Amazônia (ainda) é uma floresta onde  cada vez mais se pode garimpar, desmatar, minerar e matar. 

Na Unicamp, e especialmente na Computação, você talvez tenha a noção falsa de  que estamos protegidos desses ataques. Tem bolsa FAPESP, tem interna, tem de empresa,  intercâmbio… por enquanto. Nada disso é garantido. 

Não se engane, é no olho do furacão que o vento parece calmo. A universidade é  grande, mas é só uma parte de todo um ecossistema de geração de ciência que é tão  frágil quanto um decreto, lei ou fake news – e, por isso, precisa ser defendido, quase  sempre com unhas e dentes. 

Nossa única “sorte” é que, unidos e mobilizados, vencemos: as grandes marchas  pela Educação em maio de 2019 conseguiram reverter cortes e demonstrar ao governo  que não aguentaremos o desmonte calades e que temos apoio da sociedade! 
