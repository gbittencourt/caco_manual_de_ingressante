\documentclass{book}

\usepackage[T1]{fontenc}
\usepackage[portuguese]{babel}


\usepackage{manual_template}
% tem que decidir uma fonte legal
% https://tug.org/FontCatalogue/typewriterfonts.html

% para gerar texto lixo
\usepackage{blindtext}
\usepackage{lipsum}

\title{Manual de Ingressante}
\author{Centro Acadêmico da Computação}
\newcommand{\ano}{2024}
\date{\ano}
\newcommand{\shortauthor}{CACo}

\newcommand{\autoresCapa}{%
autor1 (EC021) \\
autor 2 (CC020)
}%

\newcommand{\autoresDiagramacao}{%
autora.e (curso) \\
autora.e (curso) \\
autora.e (curso)
}%

\newcommand{\autoresRevisao}{%
autora.e (curso) \\
autora.e (curso) \\
autora.e (curso) \\
autora.e (curso) \\
autora.e (curso) \\
autora.e (curso)
}%

\newcommand{\gestao}{nome da gestao (\ano)}

\begin{document}

\documentclass{book}

\usepackage[T1]{fontenc}
\usepackage[portuguese]{babel}


\usepackage{manual_template}
% tem que decidir uma fonte legal
% https://tug.org/FontCatalogue/typewriterfonts.html

% para gerar texto lixo
\usepackage{blindtext}
\usepackage{lipsum}

\title{Manual de Ingressante}
\author{Centro Acadêmico da Computação}
\newcommand{\ano}{2024}
\date{\ano}
\newcommand{\shortauthor}{CACo}

\newcommand{\autoresCapa}{%
autor1 (EC021) \\
autor 2 (CC020)
}%

\newcommand{\autoresDiagramacao}{%
autora.e (curso) \\
autora.e (curso) \\
autora.e (curso)
}%

\newcommand{\autoresRevisao}{%
autora.e (curso) \\
autora.e (curso) \\
autora.e (curso) \\
autora.e (curso) \\
autora.e (curso) \\
autora.e (curso)
}%

\newcommand{\gestao}{nome da gestao (\ano)}

\begin{document}

\documentclass{book}

\usepackage[T1]{fontenc}
\usepackage[portuguese]{babel}


\usepackage{manual_template}
% tem que decidir uma fonte legal
% https://tug.org/FontCatalogue/typewriterfonts.html

% para gerar texto lixo
\usepackage{blindtext}
\usepackage{lipsum}

\title{Manual de Ingressante}
\author{Centro Acadêmico da Computação}
\newcommand{\ano}{2024}
\date{\ano}
\newcommand{\shortauthor}{CACo}

\newcommand{\autoresCapa}{%
autor1 (EC021) \\
autor 2 (CC020)
}%

\newcommand{\autoresDiagramacao}{%
autora.e (curso) \\
autora.e (curso) \\
autora.e (curso)
}%

\newcommand{\autoresRevisao}{%
autora.e (curso) \\
autora.e (curso) \\
autora.e (curso) \\
autora.e (curso) \\
autora.e (curso) \\
autora.e (curso)
}%

\newcommand{\gestao}{nome da gestao (\ano)}

\begin{document}

\documentclass{book}

\usepackage[T1]{fontenc}
\usepackage[portuguese]{babel}


\usepackage{manual_template}
% tem que decidir uma fonte legal
% https://tug.org/FontCatalogue/typewriterfonts.html

% para gerar texto lixo
\usepackage{blindtext}
\usepackage{lipsum}

\title{Manual de Ingressante}
\author{Centro Acadêmico da Computação}
\newcommand{\ano}{2024}
\date{\ano}
\newcommand{\shortauthor}{CACo}

\newcommand{\autoresCapa}{%
autor1 (EC021) \\
autor 2 (CC020)
}%

\newcommand{\autoresDiagramacao}{%
autora.e (curso) \\
autora.e (curso) \\
autora.e (curso)
}%

\newcommand{\autoresRevisao}{%
autora.e (curso) \\
autora.e (curso) \\
autora.e (curso) \\
autora.e (curso) \\
autora.e (curso) \\
autora.e (curso)
}%

\newcommand{\gestao}{nome da gestao (\ano)}

\begin{document}

\input{_adendos/main}

%\maketitle
%\tableofcontents

%\setcounter{page}{1}

\include{0_boas_vindas/main}

\include{1_infraestrutura_unicamp/main}

\include{2_burocracias_e_estudos/main}

\chapter{Convivendo na Unicam}
\lipsum[1-5]

\section{whatever}
\lipsum[1-5]


\chapter{outro capitulo}
\lipsum[1-5]

%\chapter{Convivendo na Unicamp\\ {\color{white}. } \hspace{5mm} {\normalsize e no mundo}}
\chapter{Convivendo na Unicamp}
\lipsum[1-5]

\setcounter{section}{17}
\section{Mais secao (numero 17)}

\lipsum[1-5]

\section{E outra secao}

\lipsum[1-5]

\section{mais outra secao}

\lipsum[1-5]

\section{Eu amo secao}
\setcounter{subsection}{3}
\subsection{Algumas formas de evitar e combater o \\ machismo no dia a dia da universidade}

\lipsum[1-5]

%\chapter{Vivendo em Barão \\ {\color{white}. } \hspace{5mm} {\normalsize Barão Geraldo ou BG}}
\chapter{Vivendo em Barão Geraldo}
\lipsum[1-5]

\chapter{Além da Graduação}
\lipsum[1-5]

\chapter{aba}
\section{aoba}
oi

terminal \faTerminal, linux \faLinux, sapo \faFrog

github \faGithub, linkedin \faLinkedin, gitlab \faGitlab, email \faEnvelope, telegram \faPaperPlane[regular] \faTelegram\ \faTelegramPlane, whats \faWhatsapp

map marker \faMapMarker*, restaurantes \faUtensils, cafe \faCoffee, ra \faIdCard[regular] \faAddressCard[regular], casa \faHome

\begin{list}{\faTerminal}{}  
\item item A bla bla bla
\item item B blablabla
\end{list}

\section{Testes de links}


\instagram{meuinstagram}

\telegram{meugrupotelegram}

\discord{meugrupodiscord}

%\pagebreak

\lipsum[1-5]

\begin{tags}
    \github{meuusergithub} \sep \gitlab{meuusergitlab}    
\end{tags}

\end{document}


%\maketitle
%\tableofcontents

%\setcounter{page}{1}

\documentclass{book}

\usepackage[T1]{fontenc}
\usepackage[portuguese]{babel}


\usepackage{manual_template}
% tem que decidir uma fonte legal
% https://tug.org/FontCatalogue/typewriterfonts.html

% para gerar texto lixo
\usepackage{blindtext}
\usepackage{lipsum}

\title{Manual de Ingressante}
\author{Centro Acadêmico da Computação}
\newcommand{\ano}{2024}
\date{\ano}
\newcommand{\shortauthor}{CACo}

\newcommand{\autoresCapa}{%
autor1 (EC021) \\
autor 2 (CC020)
}%

\newcommand{\autoresDiagramacao}{%
autora.e (curso) \\
autora.e (curso) \\
autora.e (curso)
}%

\newcommand{\autoresRevisao}{%
autora.e (curso) \\
autora.e (curso) \\
autora.e (curso) \\
autora.e (curso) \\
autora.e (curso) \\
autora.e (curso)
}%

\newcommand{\gestao}{nome da gestao (\ano)}

\begin{document}

\input{_adendos/main}

%\maketitle
%\tableofcontents

%\setcounter{page}{1}

\include{0_boas_vindas/main}

\include{1_infraestrutura_unicamp/main}

\include{2_burocracias_e_estudos/main}

\chapter{Convivendo na Unicam}
\lipsum[1-5]

\section{whatever}
\lipsum[1-5]


\chapter{outro capitulo}
\lipsum[1-5]

%\chapter{Convivendo na Unicamp\\ {\color{white}. } \hspace{5mm} {\normalsize e no mundo}}
\chapter{Convivendo na Unicamp}
\lipsum[1-5]

\setcounter{section}{17}
\section{Mais secao (numero 17)}

\lipsum[1-5]

\section{E outra secao}

\lipsum[1-5]

\section{mais outra secao}

\lipsum[1-5]

\section{Eu amo secao}
\setcounter{subsection}{3}
\subsection{Algumas formas de evitar e combater o \\ machismo no dia a dia da universidade}

\lipsum[1-5]

%\chapter{Vivendo em Barão \\ {\color{white}. } \hspace{5mm} {\normalsize Barão Geraldo ou BG}}
\chapter{Vivendo em Barão Geraldo}
\lipsum[1-5]

\chapter{Além da Graduação}
\lipsum[1-5]

\chapter{aba}
\section{aoba}
oi

terminal \faTerminal, linux \faLinux, sapo \faFrog

github \faGithub, linkedin \faLinkedin, gitlab \faGitlab, email \faEnvelope, telegram \faPaperPlane[regular] \faTelegram\ \faTelegramPlane, whats \faWhatsapp

map marker \faMapMarker*, restaurantes \faUtensils, cafe \faCoffee, ra \faIdCard[regular] \faAddressCard[regular], casa \faHome

\begin{list}{\faTerminal}{}  
\item item A bla bla bla
\item item B blablabla
\end{list}

\section{Testes de links}


\instagram{meuinstagram}

\telegram{meugrupotelegram}

\discord{meugrupodiscord}

%\pagebreak

\lipsum[1-5]

\begin{tags}
    \github{meuusergithub} \sep \gitlab{meuusergitlab}    
\end{tags}

\end{document}


\documentclass{book}

\usepackage[T1]{fontenc}
\usepackage[portuguese]{babel}


\usepackage{manual_template}
% tem que decidir uma fonte legal
% https://tug.org/FontCatalogue/typewriterfonts.html

% para gerar texto lixo
\usepackage{blindtext}
\usepackage{lipsum}

\title{Manual de Ingressante}
\author{Centro Acadêmico da Computação}
\newcommand{\ano}{2024}
\date{\ano}
\newcommand{\shortauthor}{CACo}

\newcommand{\autoresCapa}{%
autor1 (EC021) \\
autor 2 (CC020)
}%

\newcommand{\autoresDiagramacao}{%
autora.e (curso) \\
autora.e (curso) \\
autora.e (curso)
}%

\newcommand{\autoresRevisao}{%
autora.e (curso) \\
autora.e (curso) \\
autora.e (curso) \\
autora.e (curso) \\
autora.e (curso) \\
autora.e (curso)
}%

\newcommand{\gestao}{nome da gestao (\ano)}

\begin{document}

\input{_adendos/main}

%\maketitle
%\tableofcontents

%\setcounter{page}{1}

\include{0_boas_vindas/main}

\include{1_infraestrutura_unicamp/main}

\include{2_burocracias_e_estudos/main}

\chapter{Convivendo na Unicam}
\lipsum[1-5]

\section{whatever}
\lipsum[1-5]


\chapter{outro capitulo}
\lipsum[1-5]

%\chapter{Convivendo na Unicamp\\ {\color{white}. } \hspace{5mm} {\normalsize e no mundo}}
\chapter{Convivendo na Unicamp}
\lipsum[1-5]

\setcounter{section}{17}
\section{Mais secao (numero 17)}

\lipsum[1-5]

\section{E outra secao}

\lipsum[1-5]

\section{mais outra secao}

\lipsum[1-5]

\section{Eu amo secao}
\setcounter{subsection}{3}
\subsection{Algumas formas de evitar e combater o \\ machismo no dia a dia da universidade}

\lipsum[1-5]

%\chapter{Vivendo em Barão \\ {\color{white}. } \hspace{5mm} {\normalsize Barão Geraldo ou BG}}
\chapter{Vivendo em Barão Geraldo}
\lipsum[1-5]

\chapter{Além da Graduação}
\lipsum[1-5]

\chapter{aba}
\section{aoba}
oi

terminal \faTerminal, linux \faLinux, sapo \faFrog

github \faGithub, linkedin \faLinkedin, gitlab \faGitlab, email \faEnvelope, telegram \faPaperPlane[regular] \faTelegram\ \faTelegramPlane, whats \faWhatsapp

map marker \faMapMarker*, restaurantes \faUtensils, cafe \faCoffee, ra \faIdCard[regular] \faAddressCard[regular], casa \faHome

\begin{list}{\faTerminal}{}  
\item item A bla bla bla
\item item B blablabla
\end{list}

\section{Testes de links}


\instagram{meuinstagram}

\telegram{meugrupotelegram}

\discord{meugrupodiscord}

%\pagebreak

\lipsum[1-5]

\begin{tags}
    \github{meuusergithub} \sep \gitlab{meuusergitlab}    
\end{tags}

\end{document}


\documentclass{book}

\usepackage[T1]{fontenc}
\usepackage[portuguese]{babel}


\usepackage{manual_template}
% tem que decidir uma fonte legal
% https://tug.org/FontCatalogue/typewriterfonts.html

% para gerar texto lixo
\usepackage{blindtext}
\usepackage{lipsum}

\title{Manual de Ingressante}
\author{Centro Acadêmico da Computação}
\newcommand{\ano}{2024}
\date{\ano}
\newcommand{\shortauthor}{CACo}

\newcommand{\autoresCapa}{%
autor1 (EC021) \\
autor 2 (CC020)
}%

\newcommand{\autoresDiagramacao}{%
autora.e (curso) \\
autora.e (curso) \\
autora.e (curso)
}%

\newcommand{\autoresRevisao}{%
autora.e (curso) \\
autora.e (curso) \\
autora.e (curso) \\
autora.e (curso) \\
autora.e (curso) \\
autora.e (curso)
}%

\newcommand{\gestao}{nome da gestao (\ano)}

\begin{document}

\input{_adendos/main}

%\maketitle
%\tableofcontents

%\setcounter{page}{1}

\include{0_boas_vindas/main}

\include{1_infraestrutura_unicamp/main}

\include{2_burocracias_e_estudos/main}

\chapter{Convivendo na Unicam}
\lipsum[1-5]

\section{whatever}
\lipsum[1-5]


\chapter{outro capitulo}
\lipsum[1-5]

%\chapter{Convivendo na Unicamp\\ {\color{white}. } \hspace{5mm} {\normalsize e no mundo}}
\chapter{Convivendo na Unicamp}
\lipsum[1-5]

\setcounter{section}{17}
\section{Mais secao (numero 17)}

\lipsum[1-5]

\section{E outra secao}

\lipsum[1-5]

\section{mais outra secao}

\lipsum[1-5]

\section{Eu amo secao}
\setcounter{subsection}{3}
\subsection{Algumas formas de evitar e combater o \\ machismo no dia a dia da universidade}

\lipsum[1-5]

%\chapter{Vivendo em Barão \\ {\color{white}. } \hspace{5mm} {\normalsize Barão Geraldo ou BG}}
\chapter{Vivendo em Barão Geraldo}
\lipsum[1-5]

\chapter{Além da Graduação}
\lipsum[1-5]

\chapter{aba}
\section{aoba}
oi

terminal \faTerminal, linux \faLinux, sapo \faFrog

github \faGithub, linkedin \faLinkedin, gitlab \faGitlab, email \faEnvelope, telegram \faPaperPlane[regular] \faTelegram\ \faTelegramPlane, whats \faWhatsapp

map marker \faMapMarker*, restaurantes \faUtensils, cafe \faCoffee, ra \faIdCard[regular] \faAddressCard[regular], casa \faHome

\begin{list}{\faTerminal}{}  
\item item A bla bla bla
\item item B blablabla
\end{list}

\section{Testes de links}


\instagram{meuinstagram}

\telegram{meugrupotelegram}

\discord{meugrupodiscord}

%\pagebreak

\lipsum[1-5]

\begin{tags}
    \github{meuusergithub} \sep \gitlab{meuusergitlab}    
\end{tags}

\end{document}


\chapter{Convivendo na Unicam}
\lipsum[1-5]

\section{whatever}
\lipsum[1-5]


\chapter{outro capitulo}
\lipsum[1-5]

%\chapter{Convivendo na Unicamp\\ {\color{white}. } \hspace{5mm} {\normalsize e no mundo}}
\chapter{Convivendo na Unicamp}
\lipsum[1-5]

\setcounter{section}{17}
\section{Mais secao (numero 17)}

\lipsum[1-5]

\section{E outra secao}

\lipsum[1-5]

\section{mais outra secao}

\lipsum[1-5]

\section{Eu amo secao}
\setcounter{subsection}{3}
\subsection{Algumas formas de evitar e combater o \\ machismo no dia a dia da universidade}

\lipsum[1-5]

%\chapter{Vivendo em Barão \\ {\color{white}. } \hspace{5mm} {\normalsize Barão Geraldo ou BG}}
\chapter{Vivendo em Barão Geraldo}
\lipsum[1-5]

\chapter{Além da Graduação}
\lipsum[1-5]

\chapter{aba}
\section{aoba}
oi

terminal \faTerminal, linux \faLinux, sapo \faFrog

github \faGithub, linkedin \faLinkedin, gitlab \faGitlab, email \faEnvelope, telegram \faPaperPlane[regular] \faTelegram\ \faTelegramPlane, whats \faWhatsapp

map marker \faMapMarker*, restaurantes \faUtensils, cafe \faCoffee, ra \faIdCard[regular] \faAddressCard[regular], casa \faHome

\begin{list}{\faTerminal}{}  
\item item A bla bla bla
\item item B blablabla
\end{list}

\section{Testes de links}


\instagram{meuinstagram}

\telegram{meugrupotelegram}

\discord{meugrupodiscord}

%\pagebreak

\lipsum[1-5]

\begin{tags}
    \github{meuusergithub} \sep \gitlab{meuusergitlab}    
\end{tags}

\end{document}


%\maketitle
%\tableofcontents

%\setcounter{page}{1}

\documentclass{book}

\usepackage[T1]{fontenc}
\usepackage[portuguese]{babel}


\usepackage{manual_template}
% tem que decidir uma fonte legal
% https://tug.org/FontCatalogue/typewriterfonts.html

% para gerar texto lixo
\usepackage{blindtext}
\usepackage{lipsum}

\title{Manual de Ingressante}
\author{Centro Acadêmico da Computação}
\newcommand{\ano}{2024}
\date{\ano}
\newcommand{\shortauthor}{CACo}

\newcommand{\autoresCapa}{%
autor1 (EC021) \\
autor 2 (CC020)
}%

\newcommand{\autoresDiagramacao}{%
autora.e (curso) \\
autora.e (curso) \\
autora.e (curso)
}%

\newcommand{\autoresRevisao}{%
autora.e (curso) \\
autora.e (curso) \\
autora.e (curso) \\
autora.e (curso) \\
autora.e (curso) \\
autora.e (curso)
}%

\newcommand{\gestao}{nome da gestao (\ano)}

\begin{document}

\documentclass{book}

\usepackage[T1]{fontenc}
\usepackage[portuguese]{babel}


\usepackage{manual_template}
% tem que decidir uma fonte legal
% https://tug.org/FontCatalogue/typewriterfonts.html

% para gerar texto lixo
\usepackage{blindtext}
\usepackage{lipsum}

\title{Manual de Ingressante}
\author{Centro Acadêmico da Computação}
\newcommand{\ano}{2024}
\date{\ano}
\newcommand{\shortauthor}{CACo}

\newcommand{\autoresCapa}{%
autor1 (EC021) \\
autor 2 (CC020)
}%

\newcommand{\autoresDiagramacao}{%
autora.e (curso) \\
autora.e (curso) \\
autora.e (curso)
}%

\newcommand{\autoresRevisao}{%
autora.e (curso) \\
autora.e (curso) \\
autora.e (curso) \\
autora.e (curso) \\
autora.e (curso) \\
autora.e (curso)
}%

\newcommand{\gestao}{nome da gestao (\ano)}

\begin{document}

\input{_adendos/main}

%\maketitle
%\tableofcontents

%\setcounter{page}{1}

\include{0_boas_vindas/main}

\include{1_infraestrutura_unicamp/main}

\include{2_burocracias_e_estudos/main}

\chapter{Convivendo na Unicam}
\lipsum[1-5]

\section{whatever}
\lipsum[1-5]


\chapter{outro capitulo}
\lipsum[1-5]

%\chapter{Convivendo na Unicamp\\ {\color{white}. } \hspace{5mm} {\normalsize e no mundo}}
\chapter{Convivendo na Unicamp}
\lipsum[1-5]

\setcounter{section}{17}
\section{Mais secao (numero 17)}

\lipsum[1-5]

\section{E outra secao}

\lipsum[1-5]

\section{mais outra secao}

\lipsum[1-5]

\section{Eu amo secao}
\setcounter{subsection}{3}
\subsection{Algumas formas de evitar e combater o \\ machismo no dia a dia da universidade}

\lipsum[1-5]

%\chapter{Vivendo em Barão \\ {\color{white}. } \hspace{5mm} {\normalsize Barão Geraldo ou BG}}
\chapter{Vivendo em Barão Geraldo}
\lipsum[1-5]

\chapter{Além da Graduação}
\lipsum[1-5]

\chapter{aba}
\section{aoba}
oi

terminal \faTerminal, linux \faLinux, sapo \faFrog

github \faGithub, linkedin \faLinkedin, gitlab \faGitlab, email \faEnvelope, telegram \faPaperPlane[regular] \faTelegram\ \faTelegramPlane, whats \faWhatsapp

map marker \faMapMarker*, restaurantes \faUtensils, cafe \faCoffee, ra \faIdCard[regular] \faAddressCard[regular], casa \faHome

\begin{list}{\faTerminal}{}  
\item item A bla bla bla
\item item B blablabla
\end{list}

\section{Testes de links}


\instagram{meuinstagram}

\telegram{meugrupotelegram}

\discord{meugrupodiscord}

%\pagebreak

\lipsum[1-5]

\begin{tags}
    \github{meuusergithub} \sep \gitlab{meuusergitlab}    
\end{tags}

\end{document}


%\maketitle
%\tableofcontents

%\setcounter{page}{1}

\documentclass{book}

\usepackage[T1]{fontenc}
\usepackage[portuguese]{babel}


\usepackage{manual_template}
% tem que decidir uma fonte legal
% https://tug.org/FontCatalogue/typewriterfonts.html

% para gerar texto lixo
\usepackage{blindtext}
\usepackage{lipsum}

\title{Manual de Ingressante}
\author{Centro Acadêmico da Computação}
\newcommand{\ano}{2024}
\date{\ano}
\newcommand{\shortauthor}{CACo}

\newcommand{\autoresCapa}{%
autor1 (EC021) \\
autor 2 (CC020)
}%

\newcommand{\autoresDiagramacao}{%
autora.e (curso) \\
autora.e (curso) \\
autora.e (curso)
}%

\newcommand{\autoresRevisao}{%
autora.e (curso) \\
autora.e (curso) \\
autora.e (curso) \\
autora.e (curso) \\
autora.e (curso) \\
autora.e (curso)
}%

\newcommand{\gestao}{nome da gestao (\ano)}

\begin{document}

\input{_adendos/main}

%\maketitle
%\tableofcontents

%\setcounter{page}{1}

\include{0_boas_vindas/main}

\include{1_infraestrutura_unicamp/main}

\include{2_burocracias_e_estudos/main}

\chapter{Convivendo na Unicam}
\lipsum[1-5]

\section{whatever}
\lipsum[1-5]


\chapter{outro capitulo}
\lipsum[1-5]

%\chapter{Convivendo na Unicamp\\ {\color{white}. } \hspace{5mm} {\normalsize e no mundo}}
\chapter{Convivendo na Unicamp}
\lipsum[1-5]

\setcounter{section}{17}
\section{Mais secao (numero 17)}

\lipsum[1-5]

\section{E outra secao}

\lipsum[1-5]

\section{mais outra secao}

\lipsum[1-5]

\section{Eu amo secao}
\setcounter{subsection}{3}
\subsection{Algumas formas de evitar e combater o \\ machismo no dia a dia da universidade}

\lipsum[1-5]

%\chapter{Vivendo em Barão \\ {\color{white}. } \hspace{5mm} {\normalsize Barão Geraldo ou BG}}
\chapter{Vivendo em Barão Geraldo}
\lipsum[1-5]

\chapter{Além da Graduação}
\lipsum[1-5]

\chapter{aba}
\section{aoba}
oi

terminal \faTerminal, linux \faLinux, sapo \faFrog

github \faGithub, linkedin \faLinkedin, gitlab \faGitlab, email \faEnvelope, telegram \faPaperPlane[regular] \faTelegram\ \faTelegramPlane, whats \faWhatsapp

map marker \faMapMarker*, restaurantes \faUtensils, cafe \faCoffee, ra \faIdCard[regular] \faAddressCard[regular], casa \faHome

\begin{list}{\faTerminal}{}  
\item item A bla bla bla
\item item B blablabla
\end{list}

\section{Testes de links}


\instagram{meuinstagram}

\telegram{meugrupotelegram}

\discord{meugrupodiscord}

%\pagebreak

\lipsum[1-5]

\begin{tags}
    \github{meuusergithub} \sep \gitlab{meuusergitlab}    
\end{tags}

\end{document}


\documentclass{book}

\usepackage[T1]{fontenc}
\usepackage[portuguese]{babel}


\usepackage{manual_template}
% tem que decidir uma fonte legal
% https://tug.org/FontCatalogue/typewriterfonts.html

% para gerar texto lixo
\usepackage{blindtext}
\usepackage{lipsum}

\title{Manual de Ingressante}
\author{Centro Acadêmico da Computação}
\newcommand{\ano}{2024}
\date{\ano}
\newcommand{\shortauthor}{CACo}

\newcommand{\autoresCapa}{%
autor1 (EC021) \\
autor 2 (CC020)
}%

\newcommand{\autoresDiagramacao}{%
autora.e (curso) \\
autora.e (curso) \\
autora.e (curso)
}%

\newcommand{\autoresRevisao}{%
autora.e (curso) \\
autora.e (curso) \\
autora.e (curso) \\
autora.e (curso) \\
autora.e (curso) \\
autora.e (curso)
}%

\newcommand{\gestao}{nome da gestao (\ano)}

\begin{document}

\input{_adendos/main}

%\maketitle
%\tableofcontents

%\setcounter{page}{1}

\include{0_boas_vindas/main}

\include{1_infraestrutura_unicamp/main}

\include{2_burocracias_e_estudos/main}

\chapter{Convivendo na Unicam}
\lipsum[1-5]

\section{whatever}
\lipsum[1-5]


\chapter{outro capitulo}
\lipsum[1-5]

%\chapter{Convivendo na Unicamp\\ {\color{white}. } \hspace{5mm} {\normalsize e no mundo}}
\chapter{Convivendo na Unicamp}
\lipsum[1-5]

\setcounter{section}{17}
\section{Mais secao (numero 17)}

\lipsum[1-5]

\section{E outra secao}

\lipsum[1-5]

\section{mais outra secao}

\lipsum[1-5]

\section{Eu amo secao}
\setcounter{subsection}{3}
\subsection{Algumas formas de evitar e combater o \\ machismo no dia a dia da universidade}

\lipsum[1-5]

%\chapter{Vivendo em Barão \\ {\color{white}. } \hspace{5mm} {\normalsize Barão Geraldo ou BG}}
\chapter{Vivendo em Barão Geraldo}
\lipsum[1-5]

\chapter{Além da Graduação}
\lipsum[1-5]

\chapter{aba}
\section{aoba}
oi

terminal \faTerminal, linux \faLinux, sapo \faFrog

github \faGithub, linkedin \faLinkedin, gitlab \faGitlab, email \faEnvelope, telegram \faPaperPlane[regular] \faTelegram\ \faTelegramPlane, whats \faWhatsapp

map marker \faMapMarker*, restaurantes \faUtensils, cafe \faCoffee, ra \faIdCard[regular] \faAddressCard[regular], casa \faHome

\begin{list}{\faTerminal}{}  
\item item A bla bla bla
\item item B blablabla
\end{list}

\section{Testes de links}


\instagram{meuinstagram}

\telegram{meugrupotelegram}

\discord{meugrupodiscord}

%\pagebreak

\lipsum[1-5]

\begin{tags}
    \github{meuusergithub} \sep \gitlab{meuusergitlab}    
\end{tags}

\end{document}


\documentclass{book}

\usepackage[T1]{fontenc}
\usepackage[portuguese]{babel}


\usepackage{manual_template}
% tem que decidir uma fonte legal
% https://tug.org/FontCatalogue/typewriterfonts.html

% para gerar texto lixo
\usepackage{blindtext}
\usepackage{lipsum}

\title{Manual de Ingressante}
\author{Centro Acadêmico da Computação}
\newcommand{\ano}{2024}
\date{\ano}
\newcommand{\shortauthor}{CACo}

\newcommand{\autoresCapa}{%
autor1 (EC021) \\
autor 2 (CC020)
}%

\newcommand{\autoresDiagramacao}{%
autora.e (curso) \\
autora.e (curso) \\
autora.e (curso)
}%

\newcommand{\autoresRevisao}{%
autora.e (curso) \\
autora.e (curso) \\
autora.e (curso) \\
autora.e (curso) \\
autora.e (curso) \\
autora.e (curso)
}%

\newcommand{\gestao}{nome da gestao (\ano)}

\begin{document}

\input{_adendos/main}

%\maketitle
%\tableofcontents

%\setcounter{page}{1}

\include{0_boas_vindas/main}

\include{1_infraestrutura_unicamp/main}

\include{2_burocracias_e_estudos/main}

\chapter{Convivendo na Unicam}
\lipsum[1-5]

\section{whatever}
\lipsum[1-5]


\chapter{outro capitulo}
\lipsum[1-5]

%\chapter{Convivendo na Unicamp\\ {\color{white}. } \hspace{5mm} {\normalsize e no mundo}}
\chapter{Convivendo na Unicamp}
\lipsum[1-5]

\setcounter{section}{17}
\section{Mais secao (numero 17)}

\lipsum[1-5]

\section{E outra secao}

\lipsum[1-5]

\section{mais outra secao}

\lipsum[1-5]

\section{Eu amo secao}
\setcounter{subsection}{3}
\subsection{Algumas formas de evitar e combater o \\ machismo no dia a dia da universidade}

\lipsum[1-5]

%\chapter{Vivendo em Barão \\ {\color{white}. } \hspace{5mm} {\normalsize Barão Geraldo ou BG}}
\chapter{Vivendo em Barão Geraldo}
\lipsum[1-5]

\chapter{Além da Graduação}
\lipsum[1-5]

\chapter{aba}
\section{aoba}
oi

terminal \faTerminal, linux \faLinux, sapo \faFrog

github \faGithub, linkedin \faLinkedin, gitlab \faGitlab, email \faEnvelope, telegram \faPaperPlane[regular] \faTelegram\ \faTelegramPlane, whats \faWhatsapp

map marker \faMapMarker*, restaurantes \faUtensils, cafe \faCoffee, ra \faIdCard[regular] \faAddressCard[regular], casa \faHome

\begin{list}{\faTerminal}{}  
\item item A bla bla bla
\item item B blablabla
\end{list}

\section{Testes de links}


\instagram{meuinstagram}

\telegram{meugrupotelegram}

\discord{meugrupodiscord}

%\pagebreak

\lipsum[1-5]

\begin{tags}
    \github{meuusergithub} \sep \gitlab{meuusergitlab}    
\end{tags}

\end{document}


\chapter{Convivendo na Unicam}
\lipsum[1-5]

\section{whatever}
\lipsum[1-5]


\chapter{outro capitulo}
\lipsum[1-5]

%\chapter{Convivendo na Unicamp\\ {\color{white}. } \hspace{5mm} {\normalsize e no mundo}}
\chapter{Convivendo na Unicamp}
\lipsum[1-5]

\setcounter{section}{17}
\section{Mais secao (numero 17)}

\lipsum[1-5]

\section{E outra secao}

\lipsum[1-5]

\section{mais outra secao}

\lipsum[1-5]

\section{Eu amo secao}
\setcounter{subsection}{3}
\subsection{Algumas formas de evitar e combater o \\ machismo no dia a dia da universidade}

\lipsum[1-5]

%\chapter{Vivendo em Barão \\ {\color{white}. } \hspace{5mm} {\normalsize Barão Geraldo ou BG}}
\chapter{Vivendo em Barão Geraldo}
\lipsum[1-5]

\chapter{Além da Graduação}
\lipsum[1-5]

\chapter{aba}
\section{aoba}
oi

terminal \faTerminal, linux \faLinux, sapo \faFrog

github \faGithub, linkedin \faLinkedin, gitlab \faGitlab, email \faEnvelope, telegram \faPaperPlane[regular] \faTelegram\ \faTelegramPlane, whats \faWhatsapp

map marker \faMapMarker*, restaurantes \faUtensils, cafe \faCoffee, ra \faIdCard[regular] \faAddressCard[regular], casa \faHome

\begin{list}{\faTerminal}{}  
\item item A bla bla bla
\item item B blablabla
\end{list}

\section{Testes de links}


\instagram{meuinstagram}

\telegram{meugrupotelegram}

\discord{meugrupodiscord}

%\pagebreak

\lipsum[1-5]

\begin{tags}
    \github{meuusergithub} \sep \gitlab{meuusergitlab}    
\end{tags}

\end{document}


\documentclass{book}

\usepackage[T1]{fontenc}
\usepackage[portuguese]{babel}


\usepackage{manual_template}
% tem que decidir uma fonte legal
% https://tug.org/FontCatalogue/typewriterfonts.html

% para gerar texto lixo
\usepackage{blindtext}
\usepackage{lipsum}

\title{Manual de Ingressante}
\author{Centro Acadêmico da Computação}
\newcommand{\ano}{2024}
\date{\ano}
\newcommand{\shortauthor}{CACo}

\newcommand{\autoresCapa}{%
autor1 (EC021) \\
autor 2 (CC020)
}%

\newcommand{\autoresDiagramacao}{%
autora.e (curso) \\
autora.e (curso) \\
autora.e (curso)
}%

\newcommand{\autoresRevisao}{%
autora.e (curso) \\
autora.e (curso) \\
autora.e (curso) \\
autora.e (curso) \\
autora.e (curso) \\
autora.e (curso)
}%

\newcommand{\gestao}{nome da gestao (\ano)}

\begin{document}

\documentclass{book}

\usepackage[T1]{fontenc}
\usepackage[portuguese]{babel}


\usepackage{manual_template}
% tem que decidir uma fonte legal
% https://tug.org/FontCatalogue/typewriterfonts.html

% para gerar texto lixo
\usepackage{blindtext}
\usepackage{lipsum}

\title{Manual de Ingressante}
\author{Centro Acadêmico da Computação}
\newcommand{\ano}{2024}
\date{\ano}
\newcommand{\shortauthor}{CACo}

\newcommand{\autoresCapa}{%
autor1 (EC021) \\
autor 2 (CC020)
}%

\newcommand{\autoresDiagramacao}{%
autora.e (curso) \\
autora.e (curso) \\
autora.e (curso)
}%

\newcommand{\autoresRevisao}{%
autora.e (curso) \\
autora.e (curso) \\
autora.e (curso) \\
autora.e (curso) \\
autora.e (curso) \\
autora.e (curso)
}%

\newcommand{\gestao}{nome da gestao (\ano)}

\begin{document}

\input{_adendos/main}

%\maketitle
%\tableofcontents

%\setcounter{page}{1}

\include{0_boas_vindas/main}

\include{1_infraestrutura_unicamp/main}

\include{2_burocracias_e_estudos/main}

\chapter{Convivendo na Unicam}
\lipsum[1-5]

\section{whatever}
\lipsum[1-5]


\chapter{outro capitulo}
\lipsum[1-5]

%\chapter{Convivendo na Unicamp\\ {\color{white}. } \hspace{5mm} {\normalsize e no mundo}}
\chapter{Convivendo na Unicamp}
\lipsum[1-5]

\setcounter{section}{17}
\section{Mais secao (numero 17)}

\lipsum[1-5]

\section{E outra secao}

\lipsum[1-5]

\section{mais outra secao}

\lipsum[1-5]

\section{Eu amo secao}
\setcounter{subsection}{3}
\subsection{Algumas formas de evitar e combater o \\ machismo no dia a dia da universidade}

\lipsum[1-5]

%\chapter{Vivendo em Barão \\ {\color{white}. } \hspace{5mm} {\normalsize Barão Geraldo ou BG}}
\chapter{Vivendo em Barão Geraldo}
\lipsum[1-5]

\chapter{Além da Graduação}
\lipsum[1-5]

\chapter{aba}
\section{aoba}
oi

terminal \faTerminal, linux \faLinux, sapo \faFrog

github \faGithub, linkedin \faLinkedin, gitlab \faGitlab, email \faEnvelope, telegram \faPaperPlane[regular] \faTelegram\ \faTelegramPlane, whats \faWhatsapp

map marker \faMapMarker*, restaurantes \faUtensils, cafe \faCoffee, ra \faIdCard[regular] \faAddressCard[regular], casa \faHome

\begin{list}{\faTerminal}{}  
\item item A bla bla bla
\item item B blablabla
\end{list}

\section{Testes de links}


\instagram{meuinstagram}

\telegram{meugrupotelegram}

\discord{meugrupodiscord}

%\pagebreak

\lipsum[1-5]

\begin{tags}
    \github{meuusergithub} \sep \gitlab{meuusergitlab}    
\end{tags}

\end{document}


%\maketitle
%\tableofcontents

%\setcounter{page}{1}

\documentclass{book}

\usepackage[T1]{fontenc}
\usepackage[portuguese]{babel}


\usepackage{manual_template}
% tem que decidir uma fonte legal
% https://tug.org/FontCatalogue/typewriterfonts.html

% para gerar texto lixo
\usepackage{blindtext}
\usepackage{lipsum}

\title{Manual de Ingressante}
\author{Centro Acadêmico da Computação}
\newcommand{\ano}{2024}
\date{\ano}
\newcommand{\shortauthor}{CACo}

\newcommand{\autoresCapa}{%
autor1 (EC021) \\
autor 2 (CC020)
}%

\newcommand{\autoresDiagramacao}{%
autora.e (curso) \\
autora.e (curso) \\
autora.e (curso)
}%

\newcommand{\autoresRevisao}{%
autora.e (curso) \\
autora.e (curso) \\
autora.e (curso) \\
autora.e (curso) \\
autora.e (curso) \\
autora.e (curso)
}%

\newcommand{\gestao}{nome da gestao (\ano)}

\begin{document}

\input{_adendos/main}

%\maketitle
%\tableofcontents

%\setcounter{page}{1}

\include{0_boas_vindas/main}

\include{1_infraestrutura_unicamp/main}

\include{2_burocracias_e_estudos/main}

\chapter{Convivendo na Unicam}
\lipsum[1-5]

\section{whatever}
\lipsum[1-5]


\chapter{outro capitulo}
\lipsum[1-5]

%\chapter{Convivendo na Unicamp\\ {\color{white}. } \hspace{5mm} {\normalsize e no mundo}}
\chapter{Convivendo na Unicamp}
\lipsum[1-5]

\setcounter{section}{17}
\section{Mais secao (numero 17)}

\lipsum[1-5]

\section{E outra secao}

\lipsum[1-5]

\section{mais outra secao}

\lipsum[1-5]

\section{Eu amo secao}
\setcounter{subsection}{3}
\subsection{Algumas formas de evitar e combater o \\ machismo no dia a dia da universidade}

\lipsum[1-5]

%\chapter{Vivendo em Barão \\ {\color{white}. } \hspace{5mm} {\normalsize Barão Geraldo ou BG}}
\chapter{Vivendo em Barão Geraldo}
\lipsum[1-5]

\chapter{Além da Graduação}
\lipsum[1-5]

\chapter{aba}
\section{aoba}
oi

terminal \faTerminal, linux \faLinux, sapo \faFrog

github \faGithub, linkedin \faLinkedin, gitlab \faGitlab, email \faEnvelope, telegram \faPaperPlane[regular] \faTelegram\ \faTelegramPlane, whats \faWhatsapp

map marker \faMapMarker*, restaurantes \faUtensils, cafe \faCoffee, ra \faIdCard[regular] \faAddressCard[regular], casa \faHome

\begin{list}{\faTerminal}{}  
\item item A bla bla bla
\item item B blablabla
\end{list}

\section{Testes de links}


\instagram{meuinstagram}

\telegram{meugrupotelegram}

\discord{meugrupodiscord}

%\pagebreak

\lipsum[1-5]

\begin{tags}
    \github{meuusergithub} \sep \gitlab{meuusergitlab}    
\end{tags}

\end{document}


\documentclass{book}

\usepackage[T1]{fontenc}
\usepackage[portuguese]{babel}


\usepackage{manual_template}
% tem que decidir uma fonte legal
% https://tug.org/FontCatalogue/typewriterfonts.html

% para gerar texto lixo
\usepackage{blindtext}
\usepackage{lipsum}

\title{Manual de Ingressante}
\author{Centro Acadêmico da Computação}
\newcommand{\ano}{2024}
\date{\ano}
\newcommand{\shortauthor}{CACo}

\newcommand{\autoresCapa}{%
autor1 (EC021) \\
autor 2 (CC020)
}%

\newcommand{\autoresDiagramacao}{%
autora.e (curso) \\
autora.e (curso) \\
autora.e (curso)
}%

\newcommand{\autoresRevisao}{%
autora.e (curso) \\
autora.e (curso) \\
autora.e (curso) \\
autora.e (curso) \\
autora.e (curso) \\
autora.e (curso)
}%

\newcommand{\gestao}{nome da gestao (\ano)}

\begin{document}

\input{_adendos/main}

%\maketitle
%\tableofcontents

%\setcounter{page}{1}

\include{0_boas_vindas/main}

\include{1_infraestrutura_unicamp/main}

\include{2_burocracias_e_estudos/main}

\chapter{Convivendo na Unicam}
\lipsum[1-5]

\section{whatever}
\lipsum[1-5]


\chapter{outro capitulo}
\lipsum[1-5]

%\chapter{Convivendo na Unicamp\\ {\color{white}. } \hspace{5mm} {\normalsize e no mundo}}
\chapter{Convivendo na Unicamp}
\lipsum[1-5]

\setcounter{section}{17}
\section{Mais secao (numero 17)}

\lipsum[1-5]

\section{E outra secao}

\lipsum[1-5]

\section{mais outra secao}

\lipsum[1-5]

\section{Eu amo secao}
\setcounter{subsection}{3}
\subsection{Algumas formas de evitar e combater o \\ machismo no dia a dia da universidade}

\lipsum[1-5]

%\chapter{Vivendo em Barão \\ {\color{white}. } \hspace{5mm} {\normalsize Barão Geraldo ou BG}}
\chapter{Vivendo em Barão Geraldo}
\lipsum[1-5]

\chapter{Além da Graduação}
\lipsum[1-5]

\chapter{aba}
\section{aoba}
oi

terminal \faTerminal, linux \faLinux, sapo \faFrog

github \faGithub, linkedin \faLinkedin, gitlab \faGitlab, email \faEnvelope, telegram \faPaperPlane[regular] \faTelegram\ \faTelegramPlane, whats \faWhatsapp

map marker \faMapMarker*, restaurantes \faUtensils, cafe \faCoffee, ra \faIdCard[regular] \faAddressCard[regular], casa \faHome

\begin{list}{\faTerminal}{}  
\item item A bla bla bla
\item item B blablabla
\end{list}

\section{Testes de links}


\instagram{meuinstagram}

\telegram{meugrupotelegram}

\discord{meugrupodiscord}

%\pagebreak

\lipsum[1-5]

\begin{tags}
    \github{meuusergithub} \sep \gitlab{meuusergitlab}    
\end{tags}

\end{document}


\documentclass{book}

\usepackage[T1]{fontenc}
\usepackage[portuguese]{babel}


\usepackage{manual_template}
% tem que decidir uma fonte legal
% https://tug.org/FontCatalogue/typewriterfonts.html

% para gerar texto lixo
\usepackage{blindtext}
\usepackage{lipsum}

\title{Manual de Ingressante}
\author{Centro Acadêmico da Computação}
\newcommand{\ano}{2024}
\date{\ano}
\newcommand{\shortauthor}{CACo}

\newcommand{\autoresCapa}{%
autor1 (EC021) \\
autor 2 (CC020)
}%

\newcommand{\autoresDiagramacao}{%
autora.e (curso) \\
autora.e (curso) \\
autora.e (curso)
}%

\newcommand{\autoresRevisao}{%
autora.e (curso) \\
autora.e (curso) \\
autora.e (curso) \\
autora.e (curso) \\
autora.e (curso) \\
autora.e (curso)
}%

\newcommand{\gestao}{nome da gestao (\ano)}

\begin{document}

\input{_adendos/main}

%\maketitle
%\tableofcontents

%\setcounter{page}{1}

\include{0_boas_vindas/main}

\include{1_infraestrutura_unicamp/main}

\include{2_burocracias_e_estudos/main}

\chapter{Convivendo na Unicam}
\lipsum[1-5]

\section{whatever}
\lipsum[1-5]


\chapter{outro capitulo}
\lipsum[1-5]

%\chapter{Convivendo na Unicamp\\ {\color{white}. } \hspace{5mm} {\normalsize e no mundo}}
\chapter{Convivendo na Unicamp}
\lipsum[1-5]

\setcounter{section}{17}
\section{Mais secao (numero 17)}

\lipsum[1-5]

\section{E outra secao}

\lipsum[1-5]

\section{mais outra secao}

\lipsum[1-5]

\section{Eu amo secao}
\setcounter{subsection}{3}
\subsection{Algumas formas de evitar e combater o \\ machismo no dia a dia da universidade}

\lipsum[1-5]

%\chapter{Vivendo em Barão \\ {\color{white}. } \hspace{5mm} {\normalsize Barão Geraldo ou BG}}
\chapter{Vivendo em Barão Geraldo}
\lipsum[1-5]

\chapter{Além da Graduação}
\lipsum[1-5]

\chapter{aba}
\section{aoba}
oi

terminal \faTerminal, linux \faLinux, sapo \faFrog

github \faGithub, linkedin \faLinkedin, gitlab \faGitlab, email \faEnvelope, telegram \faPaperPlane[regular] \faTelegram\ \faTelegramPlane, whats \faWhatsapp

map marker \faMapMarker*, restaurantes \faUtensils, cafe \faCoffee, ra \faIdCard[regular] \faAddressCard[regular], casa \faHome

\begin{list}{\faTerminal}{}  
\item item A bla bla bla
\item item B blablabla
\end{list}

\section{Testes de links}


\instagram{meuinstagram}

\telegram{meugrupotelegram}

\discord{meugrupodiscord}

%\pagebreak

\lipsum[1-5]

\begin{tags}
    \github{meuusergithub} \sep \gitlab{meuusergitlab}    
\end{tags}

\end{document}


\chapter{Convivendo na Unicam}
\lipsum[1-5]

\section{whatever}
\lipsum[1-5]


\chapter{outro capitulo}
\lipsum[1-5]

%\chapter{Convivendo na Unicamp\\ {\color{white}. } \hspace{5mm} {\normalsize e no mundo}}
\chapter{Convivendo na Unicamp}
\lipsum[1-5]

\setcounter{section}{17}
\section{Mais secao (numero 17)}

\lipsum[1-5]

\section{E outra secao}

\lipsum[1-5]

\section{mais outra secao}

\lipsum[1-5]

\section{Eu amo secao}
\setcounter{subsection}{3}
\subsection{Algumas formas de evitar e combater o \\ machismo no dia a dia da universidade}

\lipsum[1-5]

%\chapter{Vivendo em Barão \\ {\color{white}. } \hspace{5mm} {\normalsize Barão Geraldo ou BG}}
\chapter{Vivendo em Barão Geraldo}
\lipsum[1-5]

\chapter{Além da Graduação}
\lipsum[1-5]

\chapter{aba}
\section{aoba}
oi

terminal \faTerminal, linux \faLinux, sapo \faFrog

github \faGithub, linkedin \faLinkedin, gitlab \faGitlab, email \faEnvelope, telegram \faPaperPlane[regular] \faTelegram\ \faTelegramPlane, whats \faWhatsapp

map marker \faMapMarker*, restaurantes \faUtensils, cafe \faCoffee, ra \faIdCard[regular] \faAddressCard[regular], casa \faHome

\begin{list}{\faTerminal}{}  
\item item A bla bla bla
\item item B blablabla
\end{list}

\section{Testes de links}


\instagram{meuinstagram}

\telegram{meugrupotelegram}

\discord{meugrupodiscord}

%\pagebreak

\lipsum[1-5]

\begin{tags}
    \github{meuusergithub} \sep \gitlab{meuusergitlab}    
\end{tags}

\end{document}


\documentclass{book}

\usepackage[T1]{fontenc}
\usepackage[portuguese]{babel}


\usepackage{manual_template}
% tem que decidir uma fonte legal
% https://tug.org/FontCatalogue/typewriterfonts.html

% para gerar texto lixo
\usepackage{blindtext}
\usepackage{lipsum}

\title{Manual de Ingressante}
\author{Centro Acadêmico da Computação}
\newcommand{\ano}{2024}
\date{\ano}
\newcommand{\shortauthor}{CACo}

\newcommand{\autoresCapa}{%
autor1 (EC021) \\
autor 2 (CC020)
}%

\newcommand{\autoresDiagramacao}{%
autora.e (curso) \\
autora.e (curso) \\
autora.e (curso)
}%

\newcommand{\autoresRevisao}{%
autora.e (curso) \\
autora.e (curso) \\
autora.e (curso) \\
autora.e (curso) \\
autora.e (curso) \\
autora.e (curso)
}%

\newcommand{\gestao}{nome da gestao (\ano)}

\begin{document}

\documentclass{book}

\usepackage[T1]{fontenc}
\usepackage[portuguese]{babel}


\usepackage{manual_template}
% tem que decidir uma fonte legal
% https://tug.org/FontCatalogue/typewriterfonts.html

% para gerar texto lixo
\usepackage{blindtext}
\usepackage{lipsum}

\title{Manual de Ingressante}
\author{Centro Acadêmico da Computação}
\newcommand{\ano}{2024}
\date{\ano}
\newcommand{\shortauthor}{CACo}

\newcommand{\autoresCapa}{%
autor1 (EC021) \\
autor 2 (CC020)
}%

\newcommand{\autoresDiagramacao}{%
autora.e (curso) \\
autora.e (curso) \\
autora.e (curso)
}%

\newcommand{\autoresRevisao}{%
autora.e (curso) \\
autora.e (curso) \\
autora.e (curso) \\
autora.e (curso) \\
autora.e (curso) \\
autora.e (curso)
}%

\newcommand{\gestao}{nome da gestao (\ano)}

\begin{document}

\input{_adendos/main}

%\maketitle
%\tableofcontents

%\setcounter{page}{1}

\include{0_boas_vindas/main}

\include{1_infraestrutura_unicamp/main}

\include{2_burocracias_e_estudos/main}

\chapter{Convivendo na Unicam}
\lipsum[1-5]

\section{whatever}
\lipsum[1-5]


\chapter{outro capitulo}
\lipsum[1-5]

%\chapter{Convivendo na Unicamp\\ {\color{white}. } \hspace{5mm} {\normalsize e no mundo}}
\chapter{Convivendo na Unicamp}
\lipsum[1-5]

\setcounter{section}{17}
\section{Mais secao (numero 17)}

\lipsum[1-5]

\section{E outra secao}

\lipsum[1-5]

\section{mais outra secao}

\lipsum[1-5]

\section{Eu amo secao}
\setcounter{subsection}{3}
\subsection{Algumas formas de evitar e combater o \\ machismo no dia a dia da universidade}

\lipsum[1-5]

%\chapter{Vivendo em Barão \\ {\color{white}. } \hspace{5mm} {\normalsize Barão Geraldo ou BG}}
\chapter{Vivendo em Barão Geraldo}
\lipsum[1-5]

\chapter{Além da Graduação}
\lipsum[1-5]

\chapter{aba}
\section{aoba}
oi

terminal \faTerminal, linux \faLinux, sapo \faFrog

github \faGithub, linkedin \faLinkedin, gitlab \faGitlab, email \faEnvelope, telegram \faPaperPlane[regular] \faTelegram\ \faTelegramPlane, whats \faWhatsapp

map marker \faMapMarker*, restaurantes \faUtensils, cafe \faCoffee, ra \faIdCard[regular] \faAddressCard[regular], casa \faHome

\begin{list}{\faTerminal}{}  
\item item A bla bla bla
\item item B blablabla
\end{list}

\section{Testes de links}


\instagram{meuinstagram}

\telegram{meugrupotelegram}

\discord{meugrupodiscord}

%\pagebreak

\lipsum[1-5]

\begin{tags}
    \github{meuusergithub} \sep \gitlab{meuusergitlab}    
\end{tags}

\end{document}


%\maketitle
%\tableofcontents

%\setcounter{page}{1}

\documentclass{book}

\usepackage[T1]{fontenc}
\usepackage[portuguese]{babel}


\usepackage{manual_template}
% tem que decidir uma fonte legal
% https://tug.org/FontCatalogue/typewriterfonts.html

% para gerar texto lixo
\usepackage{blindtext}
\usepackage{lipsum}

\title{Manual de Ingressante}
\author{Centro Acadêmico da Computação}
\newcommand{\ano}{2024}
\date{\ano}
\newcommand{\shortauthor}{CACo}

\newcommand{\autoresCapa}{%
autor1 (EC021) \\
autor 2 (CC020)
}%

\newcommand{\autoresDiagramacao}{%
autora.e (curso) \\
autora.e (curso) \\
autora.e (curso)
}%

\newcommand{\autoresRevisao}{%
autora.e (curso) \\
autora.e (curso) \\
autora.e (curso) \\
autora.e (curso) \\
autora.e (curso) \\
autora.e (curso)
}%

\newcommand{\gestao}{nome da gestao (\ano)}

\begin{document}

\input{_adendos/main}

%\maketitle
%\tableofcontents

%\setcounter{page}{1}

\include{0_boas_vindas/main}

\include{1_infraestrutura_unicamp/main}

\include{2_burocracias_e_estudos/main}

\chapter{Convivendo na Unicam}
\lipsum[1-5]

\section{whatever}
\lipsum[1-5]


\chapter{outro capitulo}
\lipsum[1-5]

%\chapter{Convivendo na Unicamp\\ {\color{white}. } \hspace{5mm} {\normalsize e no mundo}}
\chapter{Convivendo na Unicamp}
\lipsum[1-5]

\setcounter{section}{17}
\section{Mais secao (numero 17)}

\lipsum[1-5]

\section{E outra secao}

\lipsum[1-5]

\section{mais outra secao}

\lipsum[1-5]

\section{Eu amo secao}
\setcounter{subsection}{3}
\subsection{Algumas formas de evitar e combater o \\ machismo no dia a dia da universidade}

\lipsum[1-5]

%\chapter{Vivendo em Barão \\ {\color{white}. } \hspace{5mm} {\normalsize Barão Geraldo ou BG}}
\chapter{Vivendo em Barão Geraldo}
\lipsum[1-5]

\chapter{Além da Graduação}
\lipsum[1-5]

\chapter{aba}
\section{aoba}
oi

terminal \faTerminal, linux \faLinux, sapo \faFrog

github \faGithub, linkedin \faLinkedin, gitlab \faGitlab, email \faEnvelope, telegram \faPaperPlane[regular] \faTelegram\ \faTelegramPlane, whats \faWhatsapp

map marker \faMapMarker*, restaurantes \faUtensils, cafe \faCoffee, ra \faIdCard[regular] \faAddressCard[regular], casa \faHome

\begin{list}{\faTerminal}{}  
\item item A bla bla bla
\item item B blablabla
\end{list}

\section{Testes de links}


\instagram{meuinstagram}

\telegram{meugrupotelegram}

\discord{meugrupodiscord}

%\pagebreak

\lipsum[1-5]

\begin{tags}
    \github{meuusergithub} \sep \gitlab{meuusergitlab}    
\end{tags}

\end{document}


\documentclass{book}

\usepackage[T1]{fontenc}
\usepackage[portuguese]{babel}


\usepackage{manual_template}
% tem que decidir uma fonte legal
% https://tug.org/FontCatalogue/typewriterfonts.html

% para gerar texto lixo
\usepackage{blindtext}
\usepackage{lipsum}

\title{Manual de Ingressante}
\author{Centro Acadêmico da Computação}
\newcommand{\ano}{2024}
\date{\ano}
\newcommand{\shortauthor}{CACo}

\newcommand{\autoresCapa}{%
autor1 (EC021) \\
autor 2 (CC020)
}%

\newcommand{\autoresDiagramacao}{%
autora.e (curso) \\
autora.e (curso) \\
autora.e (curso)
}%

\newcommand{\autoresRevisao}{%
autora.e (curso) \\
autora.e (curso) \\
autora.e (curso) \\
autora.e (curso) \\
autora.e (curso) \\
autora.e (curso)
}%

\newcommand{\gestao}{nome da gestao (\ano)}

\begin{document}

\input{_adendos/main}

%\maketitle
%\tableofcontents

%\setcounter{page}{1}

\include{0_boas_vindas/main}

\include{1_infraestrutura_unicamp/main}

\include{2_burocracias_e_estudos/main}

\chapter{Convivendo na Unicam}
\lipsum[1-5]

\section{whatever}
\lipsum[1-5]


\chapter{outro capitulo}
\lipsum[1-5]

%\chapter{Convivendo na Unicamp\\ {\color{white}. } \hspace{5mm} {\normalsize e no mundo}}
\chapter{Convivendo na Unicamp}
\lipsum[1-5]

\setcounter{section}{17}
\section{Mais secao (numero 17)}

\lipsum[1-5]

\section{E outra secao}

\lipsum[1-5]

\section{mais outra secao}

\lipsum[1-5]

\section{Eu amo secao}
\setcounter{subsection}{3}
\subsection{Algumas formas de evitar e combater o \\ machismo no dia a dia da universidade}

\lipsum[1-5]

%\chapter{Vivendo em Barão \\ {\color{white}. } \hspace{5mm} {\normalsize Barão Geraldo ou BG}}
\chapter{Vivendo em Barão Geraldo}
\lipsum[1-5]

\chapter{Além da Graduação}
\lipsum[1-5]

\chapter{aba}
\section{aoba}
oi

terminal \faTerminal, linux \faLinux, sapo \faFrog

github \faGithub, linkedin \faLinkedin, gitlab \faGitlab, email \faEnvelope, telegram \faPaperPlane[regular] \faTelegram\ \faTelegramPlane, whats \faWhatsapp

map marker \faMapMarker*, restaurantes \faUtensils, cafe \faCoffee, ra \faIdCard[regular] \faAddressCard[regular], casa \faHome

\begin{list}{\faTerminal}{}  
\item item A bla bla bla
\item item B blablabla
\end{list}

\section{Testes de links}


\instagram{meuinstagram}

\telegram{meugrupotelegram}

\discord{meugrupodiscord}

%\pagebreak

\lipsum[1-5]

\begin{tags}
    \github{meuusergithub} \sep \gitlab{meuusergitlab}    
\end{tags}

\end{document}


\documentclass{book}

\usepackage[T1]{fontenc}
\usepackage[portuguese]{babel}


\usepackage{manual_template}
% tem que decidir uma fonte legal
% https://tug.org/FontCatalogue/typewriterfonts.html

% para gerar texto lixo
\usepackage{blindtext}
\usepackage{lipsum}

\title{Manual de Ingressante}
\author{Centro Acadêmico da Computação}
\newcommand{\ano}{2024}
\date{\ano}
\newcommand{\shortauthor}{CACo}

\newcommand{\autoresCapa}{%
autor1 (EC021) \\
autor 2 (CC020)
}%

\newcommand{\autoresDiagramacao}{%
autora.e (curso) \\
autora.e (curso) \\
autora.e (curso)
}%

\newcommand{\autoresRevisao}{%
autora.e (curso) \\
autora.e (curso) \\
autora.e (curso) \\
autora.e (curso) \\
autora.e (curso) \\
autora.e (curso)
}%

\newcommand{\gestao}{nome da gestao (\ano)}

\begin{document}

\input{_adendos/main}

%\maketitle
%\tableofcontents

%\setcounter{page}{1}

\include{0_boas_vindas/main}

\include{1_infraestrutura_unicamp/main}

\include{2_burocracias_e_estudos/main}

\chapter{Convivendo na Unicam}
\lipsum[1-5]

\section{whatever}
\lipsum[1-5]


\chapter{outro capitulo}
\lipsum[1-5]

%\chapter{Convivendo na Unicamp\\ {\color{white}. } \hspace{5mm} {\normalsize e no mundo}}
\chapter{Convivendo na Unicamp}
\lipsum[1-5]

\setcounter{section}{17}
\section{Mais secao (numero 17)}

\lipsum[1-5]

\section{E outra secao}

\lipsum[1-5]

\section{mais outra secao}

\lipsum[1-5]

\section{Eu amo secao}
\setcounter{subsection}{3}
\subsection{Algumas formas de evitar e combater o \\ machismo no dia a dia da universidade}

\lipsum[1-5]

%\chapter{Vivendo em Barão \\ {\color{white}. } \hspace{5mm} {\normalsize Barão Geraldo ou BG}}
\chapter{Vivendo em Barão Geraldo}
\lipsum[1-5]

\chapter{Além da Graduação}
\lipsum[1-5]

\chapter{aba}
\section{aoba}
oi

terminal \faTerminal, linux \faLinux, sapo \faFrog

github \faGithub, linkedin \faLinkedin, gitlab \faGitlab, email \faEnvelope, telegram \faPaperPlane[regular] \faTelegram\ \faTelegramPlane, whats \faWhatsapp

map marker \faMapMarker*, restaurantes \faUtensils, cafe \faCoffee, ra \faIdCard[regular] \faAddressCard[regular], casa \faHome

\begin{list}{\faTerminal}{}  
\item item A bla bla bla
\item item B blablabla
\end{list}

\section{Testes de links}


\instagram{meuinstagram}

\telegram{meugrupotelegram}

\discord{meugrupodiscord}

%\pagebreak

\lipsum[1-5]

\begin{tags}
    \github{meuusergithub} \sep \gitlab{meuusergitlab}    
\end{tags}

\end{document}


\chapter{Convivendo na Unicam}
\lipsum[1-5]

\section{whatever}
\lipsum[1-5]


\chapter{outro capitulo}
\lipsum[1-5]

%\chapter{Convivendo na Unicamp\\ {\color{white}. } \hspace{5mm} {\normalsize e no mundo}}
\chapter{Convivendo na Unicamp}
\lipsum[1-5]

\setcounter{section}{17}
\section{Mais secao (numero 17)}

\lipsum[1-5]

\section{E outra secao}

\lipsum[1-5]

\section{mais outra secao}

\lipsum[1-5]

\section{Eu amo secao}
\setcounter{subsection}{3}
\subsection{Algumas formas de evitar e combater o \\ machismo no dia a dia da universidade}

\lipsum[1-5]

%\chapter{Vivendo em Barão \\ {\color{white}. } \hspace{5mm} {\normalsize Barão Geraldo ou BG}}
\chapter{Vivendo em Barão Geraldo}
\lipsum[1-5]

\chapter{Além da Graduação}
\lipsum[1-5]

\chapter{aba}
\section{aoba}
oi

terminal \faTerminal, linux \faLinux, sapo \faFrog

github \faGithub, linkedin \faLinkedin, gitlab \faGitlab, email \faEnvelope, telegram \faPaperPlane[regular] \faTelegram\ \faTelegramPlane, whats \faWhatsapp

map marker \faMapMarker*, restaurantes \faUtensils, cafe \faCoffee, ra \faIdCard[regular] \faAddressCard[regular], casa \faHome

\begin{list}{\faTerminal}{}  
\item item A bla bla bla
\item item B blablabla
\end{list}

\section{Testes de links}


\instagram{meuinstagram}

\telegram{meugrupotelegram}

\discord{meugrupodiscord}

%\pagebreak

\lipsum[1-5]

\begin{tags}
    \github{meuusergithub} \sep \gitlab{meuusergitlab}    
\end{tags}

\end{document}


\chapter{Convivendo na Unicam}
\lipsum[1-5]

\section{whatever}
\lipsum[1-5]


\chapter{outro capitulo}
\lipsum[1-5]

%\chapter{Convivendo na Unicamp\\ {\color{white}. } \hspace{5mm} {\normalsize e no mundo}}
\chapter{Convivendo na Unicamp}
\lipsum[1-5]

\setcounter{section}{17}
\section{Mais secao (numero 17)}

\lipsum[1-5]

\section{E outra secao}

\lipsum[1-5]

\section{mais outra secao}

\lipsum[1-5]

\section{Eu amo secao}
\setcounter{subsection}{3}
\subsection{Algumas formas de evitar e combater o \\ machismo no dia a dia da universidade}

\lipsum[1-5]

%\chapter{Vivendo em Barão \\ {\color{white}. } \hspace{5mm} {\normalsize Barão Geraldo ou BG}}
\chapter{Vivendo em Barão Geraldo}
\lipsum[1-5]

\chapter{Além da Graduação}
\lipsum[1-5]

\chapter{aba}
\section{aoba}
oi

terminal \faTerminal, linux \faLinux, sapo \faFrog

github \faGithub, linkedin \faLinkedin, gitlab \faGitlab, email \faEnvelope, telegram \faPaperPlane[regular] \faTelegram\ \faTelegramPlane, whats \faWhatsapp

map marker \faMapMarker*, restaurantes \faUtensils, cafe \faCoffee, ra \faIdCard[regular] \faAddressCard[regular], casa \faHome

\begin{list}{\faTerminal}{}  
\item item A bla bla bla
\item item B blablabla
\end{list}

\section{Testes de links}


\instagram{meuinstagram}

\telegram{meugrupotelegram}

\discord{meugrupodiscord}

%\pagebreak

\lipsum[1-5]

\begin{tags}
    \github{meuusergithub} \sep \gitlab{meuusergitlab}    
\end{tags}

\end{document}


%\maketitle
%\tableofcontents

%\setcounter{page}{1}

\documentclass{book}

\usepackage[T1]{fontenc}
\usepackage[portuguese]{babel}


\usepackage{manual_template}
% tem que decidir uma fonte legal
% https://tug.org/FontCatalogue/typewriterfonts.html

% para gerar texto lixo
\usepackage{blindtext}
\usepackage{lipsum}

\title{Manual de Ingressante}
\author{Centro Acadêmico da Computação}
\newcommand{\ano}{2024}
\date{\ano}
\newcommand{\shortauthor}{CACo}

\newcommand{\autoresCapa}{%
autor1 (EC021) \\
autor 2 (CC020)
}%

\newcommand{\autoresDiagramacao}{%
autora.e (curso) \\
autora.e (curso) \\
autora.e (curso)
}%

\newcommand{\autoresRevisao}{%
autora.e (curso) \\
autora.e (curso) \\
autora.e (curso) \\
autora.e (curso) \\
autora.e (curso) \\
autora.e (curso)
}%

\newcommand{\gestao}{nome da gestao (\ano)}

\begin{document}

\documentclass{book}

\usepackage[T1]{fontenc}
\usepackage[portuguese]{babel}


\usepackage{manual_template}
% tem que decidir uma fonte legal
% https://tug.org/FontCatalogue/typewriterfonts.html

% para gerar texto lixo
\usepackage{blindtext}
\usepackage{lipsum}

\title{Manual de Ingressante}
\author{Centro Acadêmico da Computação}
\newcommand{\ano}{2024}
\date{\ano}
\newcommand{\shortauthor}{CACo}

\newcommand{\autoresCapa}{%
autor1 (EC021) \\
autor 2 (CC020)
}%

\newcommand{\autoresDiagramacao}{%
autora.e (curso) \\
autora.e (curso) \\
autora.e (curso)
}%

\newcommand{\autoresRevisao}{%
autora.e (curso) \\
autora.e (curso) \\
autora.e (curso) \\
autora.e (curso) \\
autora.e (curso) \\
autora.e (curso)
}%

\newcommand{\gestao}{nome da gestao (\ano)}

\begin{document}

\documentclass{book}

\usepackage[T1]{fontenc}
\usepackage[portuguese]{babel}


\usepackage{manual_template}
% tem que decidir uma fonte legal
% https://tug.org/FontCatalogue/typewriterfonts.html

% para gerar texto lixo
\usepackage{blindtext}
\usepackage{lipsum}

\title{Manual de Ingressante}
\author{Centro Acadêmico da Computação}
\newcommand{\ano}{2024}
\date{\ano}
\newcommand{\shortauthor}{CACo}

\newcommand{\autoresCapa}{%
autor1 (EC021) \\
autor 2 (CC020)
}%

\newcommand{\autoresDiagramacao}{%
autora.e (curso) \\
autora.e (curso) \\
autora.e (curso)
}%

\newcommand{\autoresRevisao}{%
autora.e (curso) \\
autora.e (curso) \\
autora.e (curso) \\
autora.e (curso) \\
autora.e (curso) \\
autora.e (curso)
}%

\newcommand{\gestao}{nome da gestao (\ano)}

\begin{document}

\input{_adendos/main}

%\maketitle
%\tableofcontents

%\setcounter{page}{1}

\include{0_boas_vindas/main}

\include{1_infraestrutura_unicamp/main}

\include{2_burocracias_e_estudos/main}

\chapter{Convivendo na Unicam}
\lipsum[1-5]

\section{whatever}
\lipsum[1-5]


\chapter{outro capitulo}
\lipsum[1-5]

%\chapter{Convivendo na Unicamp\\ {\color{white}. } \hspace{5mm} {\normalsize e no mundo}}
\chapter{Convivendo na Unicamp}
\lipsum[1-5]

\setcounter{section}{17}
\section{Mais secao (numero 17)}

\lipsum[1-5]

\section{E outra secao}

\lipsum[1-5]

\section{mais outra secao}

\lipsum[1-5]

\section{Eu amo secao}
\setcounter{subsection}{3}
\subsection{Algumas formas de evitar e combater o \\ machismo no dia a dia da universidade}

\lipsum[1-5]

%\chapter{Vivendo em Barão \\ {\color{white}. } \hspace{5mm} {\normalsize Barão Geraldo ou BG}}
\chapter{Vivendo em Barão Geraldo}
\lipsum[1-5]

\chapter{Além da Graduação}
\lipsum[1-5]

\chapter{aba}
\section{aoba}
oi

terminal \faTerminal, linux \faLinux, sapo \faFrog

github \faGithub, linkedin \faLinkedin, gitlab \faGitlab, email \faEnvelope, telegram \faPaperPlane[regular] \faTelegram\ \faTelegramPlane, whats \faWhatsapp

map marker \faMapMarker*, restaurantes \faUtensils, cafe \faCoffee, ra \faIdCard[regular] \faAddressCard[regular], casa \faHome

\begin{list}{\faTerminal}{}  
\item item A bla bla bla
\item item B blablabla
\end{list}

\section{Testes de links}


\instagram{meuinstagram}

\telegram{meugrupotelegram}

\discord{meugrupodiscord}

%\pagebreak

\lipsum[1-5]

\begin{tags}
    \github{meuusergithub} \sep \gitlab{meuusergitlab}    
\end{tags}

\end{document}


%\maketitle
%\tableofcontents

%\setcounter{page}{1}

\documentclass{book}

\usepackage[T1]{fontenc}
\usepackage[portuguese]{babel}


\usepackage{manual_template}
% tem que decidir uma fonte legal
% https://tug.org/FontCatalogue/typewriterfonts.html

% para gerar texto lixo
\usepackage{blindtext}
\usepackage{lipsum}

\title{Manual de Ingressante}
\author{Centro Acadêmico da Computação}
\newcommand{\ano}{2024}
\date{\ano}
\newcommand{\shortauthor}{CACo}

\newcommand{\autoresCapa}{%
autor1 (EC021) \\
autor 2 (CC020)
}%

\newcommand{\autoresDiagramacao}{%
autora.e (curso) \\
autora.e (curso) \\
autora.e (curso)
}%

\newcommand{\autoresRevisao}{%
autora.e (curso) \\
autora.e (curso) \\
autora.e (curso) \\
autora.e (curso) \\
autora.e (curso) \\
autora.e (curso)
}%

\newcommand{\gestao}{nome da gestao (\ano)}

\begin{document}

\input{_adendos/main}

%\maketitle
%\tableofcontents

%\setcounter{page}{1}

\include{0_boas_vindas/main}

\include{1_infraestrutura_unicamp/main}

\include{2_burocracias_e_estudos/main}

\chapter{Convivendo na Unicam}
\lipsum[1-5]

\section{whatever}
\lipsum[1-5]


\chapter{outro capitulo}
\lipsum[1-5]

%\chapter{Convivendo na Unicamp\\ {\color{white}. } \hspace{5mm} {\normalsize e no mundo}}
\chapter{Convivendo na Unicamp}
\lipsum[1-5]

\setcounter{section}{17}
\section{Mais secao (numero 17)}

\lipsum[1-5]

\section{E outra secao}

\lipsum[1-5]

\section{mais outra secao}

\lipsum[1-5]

\section{Eu amo secao}
\setcounter{subsection}{3}
\subsection{Algumas formas de evitar e combater o \\ machismo no dia a dia da universidade}

\lipsum[1-5]

%\chapter{Vivendo em Barão \\ {\color{white}. } \hspace{5mm} {\normalsize Barão Geraldo ou BG}}
\chapter{Vivendo em Barão Geraldo}
\lipsum[1-5]

\chapter{Além da Graduação}
\lipsum[1-5]

\chapter{aba}
\section{aoba}
oi

terminal \faTerminal, linux \faLinux, sapo \faFrog

github \faGithub, linkedin \faLinkedin, gitlab \faGitlab, email \faEnvelope, telegram \faPaperPlane[regular] \faTelegram\ \faTelegramPlane, whats \faWhatsapp

map marker \faMapMarker*, restaurantes \faUtensils, cafe \faCoffee, ra \faIdCard[regular] \faAddressCard[regular], casa \faHome

\begin{list}{\faTerminal}{}  
\item item A bla bla bla
\item item B blablabla
\end{list}

\section{Testes de links}


\instagram{meuinstagram}

\telegram{meugrupotelegram}

\discord{meugrupodiscord}

%\pagebreak

\lipsum[1-5]

\begin{tags}
    \github{meuusergithub} \sep \gitlab{meuusergitlab}    
\end{tags}

\end{document}


\documentclass{book}

\usepackage[T1]{fontenc}
\usepackage[portuguese]{babel}


\usepackage{manual_template}
% tem que decidir uma fonte legal
% https://tug.org/FontCatalogue/typewriterfonts.html

% para gerar texto lixo
\usepackage{blindtext}
\usepackage{lipsum}

\title{Manual de Ingressante}
\author{Centro Acadêmico da Computação}
\newcommand{\ano}{2024}
\date{\ano}
\newcommand{\shortauthor}{CACo}

\newcommand{\autoresCapa}{%
autor1 (EC021) \\
autor 2 (CC020)
}%

\newcommand{\autoresDiagramacao}{%
autora.e (curso) \\
autora.e (curso) \\
autora.e (curso)
}%

\newcommand{\autoresRevisao}{%
autora.e (curso) \\
autora.e (curso) \\
autora.e (curso) \\
autora.e (curso) \\
autora.e (curso) \\
autora.e (curso)
}%

\newcommand{\gestao}{nome da gestao (\ano)}

\begin{document}

\input{_adendos/main}

%\maketitle
%\tableofcontents

%\setcounter{page}{1}

\include{0_boas_vindas/main}

\include{1_infraestrutura_unicamp/main}

\include{2_burocracias_e_estudos/main}

\chapter{Convivendo na Unicam}
\lipsum[1-5]

\section{whatever}
\lipsum[1-5]


\chapter{outro capitulo}
\lipsum[1-5]

%\chapter{Convivendo na Unicamp\\ {\color{white}. } \hspace{5mm} {\normalsize e no mundo}}
\chapter{Convivendo na Unicamp}
\lipsum[1-5]

\setcounter{section}{17}
\section{Mais secao (numero 17)}

\lipsum[1-5]

\section{E outra secao}

\lipsum[1-5]

\section{mais outra secao}

\lipsum[1-5]

\section{Eu amo secao}
\setcounter{subsection}{3}
\subsection{Algumas formas de evitar e combater o \\ machismo no dia a dia da universidade}

\lipsum[1-5]

%\chapter{Vivendo em Barão \\ {\color{white}. } \hspace{5mm} {\normalsize Barão Geraldo ou BG}}
\chapter{Vivendo em Barão Geraldo}
\lipsum[1-5]

\chapter{Além da Graduação}
\lipsum[1-5]

\chapter{aba}
\section{aoba}
oi

terminal \faTerminal, linux \faLinux, sapo \faFrog

github \faGithub, linkedin \faLinkedin, gitlab \faGitlab, email \faEnvelope, telegram \faPaperPlane[regular] \faTelegram\ \faTelegramPlane, whats \faWhatsapp

map marker \faMapMarker*, restaurantes \faUtensils, cafe \faCoffee, ra \faIdCard[regular] \faAddressCard[regular], casa \faHome

\begin{list}{\faTerminal}{}  
\item item A bla bla bla
\item item B blablabla
\end{list}

\section{Testes de links}


\instagram{meuinstagram}

\telegram{meugrupotelegram}

\discord{meugrupodiscord}

%\pagebreak

\lipsum[1-5]

\begin{tags}
    \github{meuusergithub} \sep \gitlab{meuusergitlab}    
\end{tags}

\end{document}


\documentclass{book}

\usepackage[T1]{fontenc}
\usepackage[portuguese]{babel}


\usepackage{manual_template}
% tem que decidir uma fonte legal
% https://tug.org/FontCatalogue/typewriterfonts.html

% para gerar texto lixo
\usepackage{blindtext}
\usepackage{lipsum}

\title{Manual de Ingressante}
\author{Centro Acadêmico da Computação}
\newcommand{\ano}{2024}
\date{\ano}
\newcommand{\shortauthor}{CACo}

\newcommand{\autoresCapa}{%
autor1 (EC021) \\
autor 2 (CC020)
}%

\newcommand{\autoresDiagramacao}{%
autora.e (curso) \\
autora.e (curso) \\
autora.e (curso)
}%

\newcommand{\autoresRevisao}{%
autora.e (curso) \\
autora.e (curso) \\
autora.e (curso) \\
autora.e (curso) \\
autora.e (curso) \\
autora.e (curso)
}%

\newcommand{\gestao}{nome da gestao (\ano)}

\begin{document}

\input{_adendos/main}

%\maketitle
%\tableofcontents

%\setcounter{page}{1}

\include{0_boas_vindas/main}

\include{1_infraestrutura_unicamp/main}

\include{2_burocracias_e_estudos/main}

\chapter{Convivendo na Unicam}
\lipsum[1-5]

\section{whatever}
\lipsum[1-5]


\chapter{outro capitulo}
\lipsum[1-5]

%\chapter{Convivendo na Unicamp\\ {\color{white}. } \hspace{5mm} {\normalsize e no mundo}}
\chapter{Convivendo na Unicamp}
\lipsum[1-5]

\setcounter{section}{17}
\section{Mais secao (numero 17)}

\lipsum[1-5]

\section{E outra secao}

\lipsum[1-5]

\section{mais outra secao}

\lipsum[1-5]

\section{Eu amo secao}
\setcounter{subsection}{3}
\subsection{Algumas formas de evitar e combater o \\ machismo no dia a dia da universidade}

\lipsum[1-5]

%\chapter{Vivendo em Barão \\ {\color{white}. } \hspace{5mm} {\normalsize Barão Geraldo ou BG}}
\chapter{Vivendo em Barão Geraldo}
\lipsum[1-5]

\chapter{Além da Graduação}
\lipsum[1-5]

\chapter{aba}
\section{aoba}
oi

terminal \faTerminal, linux \faLinux, sapo \faFrog

github \faGithub, linkedin \faLinkedin, gitlab \faGitlab, email \faEnvelope, telegram \faPaperPlane[regular] \faTelegram\ \faTelegramPlane, whats \faWhatsapp

map marker \faMapMarker*, restaurantes \faUtensils, cafe \faCoffee, ra \faIdCard[regular] \faAddressCard[regular], casa \faHome

\begin{list}{\faTerminal}{}  
\item item A bla bla bla
\item item B blablabla
\end{list}

\section{Testes de links}


\instagram{meuinstagram}

\telegram{meugrupotelegram}

\discord{meugrupodiscord}

%\pagebreak

\lipsum[1-5]

\begin{tags}
    \github{meuusergithub} \sep \gitlab{meuusergitlab}    
\end{tags}

\end{document}


\chapter{Convivendo na Unicam}
\lipsum[1-5]

\section{whatever}
\lipsum[1-5]


\chapter{outro capitulo}
\lipsum[1-5]

%\chapter{Convivendo na Unicamp\\ {\color{white}. } \hspace{5mm} {\normalsize e no mundo}}
\chapter{Convivendo na Unicamp}
\lipsum[1-5]

\setcounter{section}{17}
\section{Mais secao (numero 17)}

\lipsum[1-5]

\section{E outra secao}

\lipsum[1-5]

\section{mais outra secao}

\lipsum[1-5]

\section{Eu amo secao}
\setcounter{subsection}{3}
\subsection{Algumas formas de evitar e combater o \\ machismo no dia a dia da universidade}

\lipsum[1-5]

%\chapter{Vivendo em Barão \\ {\color{white}. } \hspace{5mm} {\normalsize Barão Geraldo ou BG}}
\chapter{Vivendo em Barão Geraldo}
\lipsum[1-5]

\chapter{Além da Graduação}
\lipsum[1-5]

\chapter{aba}
\section{aoba}
oi

terminal \faTerminal, linux \faLinux, sapo \faFrog

github \faGithub, linkedin \faLinkedin, gitlab \faGitlab, email \faEnvelope, telegram \faPaperPlane[regular] \faTelegram\ \faTelegramPlane, whats \faWhatsapp

map marker \faMapMarker*, restaurantes \faUtensils, cafe \faCoffee, ra \faIdCard[regular] \faAddressCard[regular], casa \faHome

\begin{list}{\faTerminal}{}  
\item item A bla bla bla
\item item B blablabla
\end{list}

\section{Testes de links}


\instagram{meuinstagram}

\telegram{meugrupotelegram}

\discord{meugrupodiscord}

%\pagebreak

\lipsum[1-5]

\begin{tags}
    \github{meuusergithub} \sep \gitlab{meuusergitlab}    
\end{tags}

\end{document}


%\maketitle
%\tableofcontents

%\setcounter{page}{1}

\documentclass{book}

\usepackage[T1]{fontenc}
\usepackage[portuguese]{babel}


\usepackage{manual_template}
% tem que decidir uma fonte legal
% https://tug.org/FontCatalogue/typewriterfonts.html

% para gerar texto lixo
\usepackage{blindtext}
\usepackage{lipsum}

\title{Manual de Ingressante}
\author{Centro Acadêmico da Computação}
\newcommand{\ano}{2024}
\date{\ano}
\newcommand{\shortauthor}{CACo}

\newcommand{\autoresCapa}{%
autor1 (EC021) \\
autor 2 (CC020)
}%

\newcommand{\autoresDiagramacao}{%
autora.e (curso) \\
autora.e (curso) \\
autora.e (curso)
}%

\newcommand{\autoresRevisao}{%
autora.e (curso) \\
autora.e (curso) \\
autora.e (curso) \\
autora.e (curso) \\
autora.e (curso) \\
autora.e (curso)
}%

\newcommand{\gestao}{nome da gestao (\ano)}

\begin{document}

\documentclass{book}

\usepackage[T1]{fontenc}
\usepackage[portuguese]{babel}


\usepackage{manual_template}
% tem que decidir uma fonte legal
% https://tug.org/FontCatalogue/typewriterfonts.html

% para gerar texto lixo
\usepackage{blindtext}
\usepackage{lipsum}

\title{Manual de Ingressante}
\author{Centro Acadêmico da Computação}
\newcommand{\ano}{2024}
\date{\ano}
\newcommand{\shortauthor}{CACo}

\newcommand{\autoresCapa}{%
autor1 (EC021) \\
autor 2 (CC020)
}%

\newcommand{\autoresDiagramacao}{%
autora.e (curso) \\
autora.e (curso) \\
autora.e (curso)
}%

\newcommand{\autoresRevisao}{%
autora.e (curso) \\
autora.e (curso) \\
autora.e (curso) \\
autora.e (curso) \\
autora.e (curso) \\
autora.e (curso)
}%

\newcommand{\gestao}{nome da gestao (\ano)}

\begin{document}

\input{_adendos/main}

%\maketitle
%\tableofcontents

%\setcounter{page}{1}

\include{0_boas_vindas/main}

\include{1_infraestrutura_unicamp/main}

\include{2_burocracias_e_estudos/main}

\chapter{Convivendo na Unicam}
\lipsum[1-5]

\section{whatever}
\lipsum[1-5]


\chapter{outro capitulo}
\lipsum[1-5]

%\chapter{Convivendo na Unicamp\\ {\color{white}. } \hspace{5mm} {\normalsize e no mundo}}
\chapter{Convivendo na Unicamp}
\lipsum[1-5]

\setcounter{section}{17}
\section{Mais secao (numero 17)}

\lipsum[1-5]

\section{E outra secao}

\lipsum[1-5]

\section{mais outra secao}

\lipsum[1-5]

\section{Eu amo secao}
\setcounter{subsection}{3}
\subsection{Algumas formas de evitar e combater o \\ machismo no dia a dia da universidade}

\lipsum[1-5]

%\chapter{Vivendo em Barão \\ {\color{white}. } \hspace{5mm} {\normalsize Barão Geraldo ou BG}}
\chapter{Vivendo em Barão Geraldo}
\lipsum[1-5]

\chapter{Além da Graduação}
\lipsum[1-5]

\chapter{aba}
\section{aoba}
oi

terminal \faTerminal, linux \faLinux, sapo \faFrog

github \faGithub, linkedin \faLinkedin, gitlab \faGitlab, email \faEnvelope, telegram \faPaperPlane[regular] \faTelegram\ \faTelegramPlane, whats \faWhatsapp

map marker \faMapMarker*, restaurantes \faUtensils, cafe \faCoffee, ra \faIdCard[regular] \faAddressCard[regular], casa \faHome

\begin{list}{\faTerminal}{}  
\item item A bla bla bla
\item item B blablabla
\end{list}

\section{Testes de links}


\instagram{meuinstagram}

\telegram{meugrupotelegram}

\discord{meugrupodiscord}

%\pagebreak

\lipsum[1-5]

\begin{tags}
    \github{meuusergithub} \sep \gitlab{meuusergitlab}    
\end{tags}

\end{document}


%\maketitle
%\tableofcontents

%\setcounter{page}{1}

\documentclass{book}

\usepackage[T1]{fontenc}
\usepackage[portuguese]{babel}


\usepackage{manual_template}
% tem que decidir uma fonte legal
% https://tug.org/FontCatalogue/typewriterfonts.html

% para gerar texto lixo
\usepackage{blindtext}
\usepackage{lipsum}

\title{Manual de Ingressante}
\author{Centro Acadêmico da Computação}
\newcommand{\ano}{2024}
\date{\ano}
\newcommand{\shortauthor}{CACo}

\newcommand{\autoresCapa}{%
autor1 (EC021) \\
autor 2 (CC020)
}%

\newcommand{\autoresDiagramacao}{%
autora.e (curso) \\
autora.e (curso) \\
autora.e (curso)
}%

\newcommand{\autoresRevisao}{%
autora.e (curso) \\
autora.e (curso) \\
autora.e (curso) \\
autora.e (curso) \\
autora.e (curso) \\
autora.e (curso)
}%

\newcommand{\gestao}{nome da gestao (\ano)}

\begin{document}

\input{_adendos/main}

%\maketitle
%\tableofcontents

%\setcounter{page}{1}

\include{0_boas_vindas/main}

\include{1_infraestrutura_unicamp/main}

\include{2_burocracias_e_estudos/main}

\chapter{Convivendo na Unicam}
\lipsum[1-5]

\section{whatever}
\lipsum[1-5]


\chapter{outro capitulo}
\lipsum[1-5]

%\chapter{Convivendo na Unicamp\\ {\color{white}. } \hspace{5mm} {\normalsize e no mundo}}
\chapter{Convivendo na Unicamp}
\lipsum[1-5]

\setcounter{section}{17}
\section{Mais secao (numero 17)}

\lipsum[1-5]

\section{E outra secao}

\lipsum[1-5]

\section{mais outra secao}

\lipsum[1-5]

\section{Eu amo secao}
\setcounter{subsection}{3}
\subsection{Algumas formas de evitar e combater o \\ machismo no dia a dia da universidade}

\lipsum[1-5]

%\chapter{Vivendo em Barão \\ {\color{white}. } \hspace{5mm} {\normalsize Barão Geraldo ou BG}}
\chapter{Vivendo em Barão Geraldo}
\lipsum[1-5]

\chapter{Além da Graduação}
\lipsum[1-5]

\chapter{aba}
\section{aoba}
oi

terminal \faTerminal, linux \faLinux, sapo \faFrog

github \faGithub, linkedin \faLinkedin, gitlab \faGitlab, email \faEnvelope, telegram \faPaperPlane[regular] \faTelegram\ \faTelegramPlane, whats \faWhatsapp

map marker \faMapMarker*, restaurantes \faUtensils, cafe \faCoffee, ra \faIdCard[regular] \faAddressCard[regular], casa \faHome

\begin{list}{\faTerminal}{}  
\item item A bla bla bla
\item item B blablabla
\end{list}

\section{Testes de links}


\instagram{meuinstagram}

\telegram{meugrupotelegram}

\discord{meugrupodiscord}

%\pagebreak

\lipsum[1-5]

\begin{tags}
    \github{meuusergithub} \sep \gitlab{meuusergitlab}    
\end{tags}

\end{document}


\documentclass{book}

\usepackage[T1]{fontenc}
\usepackage[portuguese]{babel}


\usepackage{manual_template}
% tem que decidir uma fonte legal
% https://tug.org/FontCatalogue/typewriterfonts.html

% para gerar texto lixo
\usepackage{blindtext}
\usepackage{lipsum}

\title{Manual de Ingressante}
\author{Centro Acadêmico da Computação}
\newcommand{\ano}{2024}
\date{\ano}
\newcommand{\shortauthor}{CACo}

\newcommand{\autoresCapa}{%
autor1 (EC021) \\
autor 2 (CC020)
}%

\newcommand{\autoresDiagramacao}{%
autora.e (curso) \\
autora.e (curso) \\
autora.e (curso)
}%

\newcommand{\autoresRevisao}{%
autora.e (curso) \\
autora.e (curso) \\
autora.e (curso) \\
autora.e (curso) \\
autora.e (curso) \\
autora.e (curso)
}%

\newcommand{\gestao}{nome da gestao (\ano)}

\begin{document}

\input{_adendos/main}

%\maketitle
%\tableofcontents

%\setcounter{page}{1}

\include{0_boas_vindas/main}

\include{1_infraestrutura_unicamp/main}

\include{2_burocracias_e_estudos/main}

\chapter{Convivendo na Unicam}
\lipsum[1-5]

\section{whatever}
\lipsum[1-5]


\chapter{outro capitulo}
\lipsum[1-5]

%\chapter{Convivendo na Unicamp\\ {\color{white}. } \hspace{5mm} {\normalsize e no mundo}}
\chapter{Convivendo na Unicamp}
\lipsum[1-5]

\setcounter{section}{17}
\section{Mais secao (numero 17)}

\lipsum[1-5]

\section{E outra secao}

\lipsum[1-5]

\section{mais outra secao}

\lipsum[1-5]

\section{Eu amo secao}
\setcounter{subsection}{3}
\subsection{Algumas formas de evitar e combater o \\ machismo no dia a dia da universidade}

\lipsum[1-5]

%\chapter{Vivendo em Barão \\ {\color{white}. } \hspace{5mm} {\normalsize Barão Geraldo ou BG}}
\chapter{Vivendo em Barão Geraldo}
\lipsum[1-5]

\chapter{Além da Graduação}
\lipsum[1-5]

\chapter{aba}
\section{aoba}
oi

terminal \faTerminal, linux \faLinux, sapo \faFrog

github \faGithub, linkedin \faLinkedin, gitlab \faGitlab, email \faEnvelope, telegram \faPaperPlane[regular] \faTelegram\ \faTelegramPlane, whats \faWhatsapp

map marker \faMapMarker*, restaurantes \faUtensils, cafe \faCoffee, ra \faIdCard[regular] \faAddressCard[regular], casa \faHome

\begin{list}{\faTerminal}{}  
\item item A bla bla bla
\item item B blablabla
\end{list}

\section{Testes de links}


\instagram{meuinstagram}

\telegram{meugrupotelegram}

\discord{meugrupodiscord}

%\pagebreak

\lipsum[1-5]

\begin{tags}
    \github{meuusergithub} \sep \gitlab{meuusergitlab}    
\end{tags}

\end{document}


\documentclass{book}

\usepackage[T1]{fontenc}
\usepackage[portuguese]{babel}


\usepackage{manual_template}
% tem que decidir uma fonte legal
% https://tug.org/FontCatalogue/typewriterfonts.html

% para gerar texto lixo
\usepackage{blindtext}
\usepackage{lipsum}

\title{Manual de Ingressante}
\author{Centro Acadêmico da Computação}
\newcommand{\ano}{2024}
\date{\ano}
\newcommand{\shortauthor}{CACo}

\newcommand{\autoresCapa}{%
autor1 (EC021) \\
autor 2 (CC020)
}%

\newcommand{\autoresDiagramacao}{%
autora.e (curso) \\
autora.e (curso) \\
autora.e (curso)
}%

\newcommand{\autoresRevisao}{%
autora.e (curso) \\
autora.e (curso) \\
autora.e (curso) \\
autora.e (curso) \\
autora.e (curso) \\
autora.e (curso)
}%

\newcommand{\gestao}{nome da gestao (\ano)}

\begin{document}

\input{_adendos/main}

%\maketitle
%\tableofcontents

%\setcounter{page}{1}

\include{0_boas_vindas/main}

\include{1_infraestrutura_unicamp/main}

\include{2_burocracias_e_estudos/main}

\chapter{Convivendo na Unicam}
\lipsum[1-5]

\section{whatever}
\lipsum[1-5]


\chapter{outro capitulo}
\lipsum[1-5]

%\chapter{Convivendo na Unicamp\\ {\color{white}. } \hspace{5mm} {\normalsize e no mundo}}
\chapter{Convivendo na Unicamp}
\lipsum[1-5]

\setcounter{section}{17}
\section{Mais secao (numero 17)}

\lipsum[1-5]

\section{E outra secao}

\lipsum[1-5]

\section{mais outra secao}

\lipsum[1-5]

\section{Eu amo secao}
\setcounter{subsection}{3}
\subsection{Algumas formas de evitar e combater o \\ machismo no dia a dia da universidade}

\lipsum[1-5]

%\chapter{Vivendo em Barão \\ {\color{white}. } \hspace{5mm} {\normalsize Barão Geraldo ou BG}}
\chapter{Vivendo em Barão Geraldo}
\lipsum[1-5]

\chapter{Além da Graduação}
\lipsum[1-5]

\chapter{aba}
\section{aoba}
oi

terminal \faTerminal, linux \faLinux, sapo \faFrog

github \faGithub, linkedin \faLinkedin, gitlab \faGitlab, email \faEnvelope, telegram \faPaperPlane[regular] \faTelegram\ \faTelegramPlane, whats \faWhatsapp

map marker \faMapMarker*, restaurantes \faUtensils, cafe \faCoffee, ra \faIdCard[regular] \faAddressCard[regular], casa \faHome

\begin{list}{\faTerminal}{}  
\item item A bla bla bla
\item item B blablabla
\end{list}

\section{Testes de links}


\instagram{meuinstagram}

\telegram{meugrupotelegram}

\discord{meugrupodiscord}

%\pagebreak

\lipsum[1-5]

\begin{tags}
    \github{meuusergithub} \sep \gitlab{meuusergitlab}    
\end{tags}

\end{document}


\chapter{Convivendo na Unicam}
\lipsum[1-5]

\section{whatever}
\lipsum[1-5]


\chapter{outro capitulo}
\lipsum[1-5]

%\chapter{Convivendo na Unicamp\\ {\color{white}. } \hspace{5mm} {\normalsize e no mundo}}
\chapter{Convivendo na Unicamp}
\lipsum[1-5]

\setcounter{section}{17}
\section{Mais secao (numero 17)}

\lipsum[1-5]

\section{E outra secao}

\lipsum[1-5]

\section{mais outra secao}

\lipsum[1-5]

\section{Eu amo secao}
\setcounter{subsection}{3}
\subsection{Algumas formas de evitar e combater o \\ machismo no dia a dia da universidade}

\lipsum[1-5]

%\chapter{Vivendo em Barão \\ {\color{white}. } \hspace{5mm} {\normalsize Barão Geraldo ou BG}}
\chapter{Vivendo em Barão Geraldo}
\lipsum[1-5]

\chapter{Além da Graduação}
\lipsum[1-5]

\chapter{aba}
\section{aoba}
oi

terminal \faTerminal, linux \faLinux, sapo \faFrog

github \faGithub, linkedin \faLinkedin, gitlab \faGitlab, email \faEnvelope, telegram \faPaperPlane[regular] \faTelegram\ \faTelegramPlane, whats \faWhatsapp

map marker \faMapMarker*, restaurantes \faUtensils, cafe \faCoffee, ra \faIdCard[regular] \faAddressCard[regular], casa \faHome

\begin{list}{\faTerminal}{}  
\item item A bla bla bla
\item item B blablabla
\end{list}

\section{Testes de links}


\instagram{meuinstagram}

\telegram{meugrupotelegram}

\discord{meugrupodiscord}

%\pagebreak

\lipsum[1-5]

\begin{tags}
    \github{meuusergithub} \sep \gitlab{meuusergitlab}    
\end{tags}

\end{document}


\documentclass{book}

\usepackage[T1]{fontenc}
\usepackage[portuguese]{babel}


\usepackage{manual_template}
% tem que decidir uma fonte legal
% https://tug.org/FontCatalogue/typewriterfonts.html

% para gerar texto lixo
\usepackage{blindtext}
\usepackage{lipsum}

\title{Manual de Ingressante}
\author{Centro Acadêmico da Computação}
\newcommand{\ano}{2024}
\date{\ano}
\newcommand{\shortauthor}{CACo}

\newcommand{\autoresCapa}{%
autor1 (EC021) \\
autor 2 (CC020)
}%

\newcommand{\autoresDiagramacao}{%
autora.e (curso) \\
autora.e (curso) \\
autora.e (curso)
}%

\newcommand{\autoresRevisao}{%
autora.e (curso) \\
autora.e (curso) \\
autora.e (curso) \\
autora.e (curso) \\
autora.e (curso) \\
autora.e (curso)
}%

\newcommand{\gestao}{nome da gestao (\ano)}

\begin{document}

\documentclass{book}

\usepackage[T1]{fontenc}
\usepackage[portuguese]{babel}


\usepackage{manual_template}
% tem que decidir uma fonte legal
% https://tug.org/FontCatalogue/typewriterfonts.html

% para gerar texto lixo
\usepackage{blindtext}
\usepackage{lipsum}

\title{Manual de Ingressante}
\author{Centro Acadêmico da Computação}
\newcommand{\ano}{2024}
\date{\ano}
\newcommand{\shortauthor}{CACo}

\newcommand{\autoresCapa}{%
autor1 (EC021) \\
autor 2 (CC020)
}%

\newcommand{\autoresDiagramacao}{%
autora.e (curso) \\
autora.e (curso) \\
autora.e (curso)
}%

\newcommand{\autoresRevisao}{%
autora.e (curso) \\
autora.e (curso) \\
autora.e (curso) \\
autora.e (curso) \\
autora.e (curso) \\
autora.e (curso)
}%

\newcommand{\gestao}{nome da gestao (\ano)}

\begin{document}

\input{_adendos/main}

%\maketitle
%\tableofcontents

%\setcounter{page}{1}

\include{0_boas_vindas/main}

\include{1_infraestrutura_unicamp/main}

\include{2_burocracias_e_estudos/main}

\chapter{Convivendo na Unicam}
\lipsum[1-5]

\section{whatever}
\lipsum[1-5]


\chapter{outro capitulo}
\lipsum[1-5]

%\chapter{Convivendo na Unicamp\\ {\color{white}. } \hspace{5mm} {\normalsize e no mundo}}
\chapter{Convivendo na Unicamp}
\lipsum[1-5]

\setcounter{section}{17}
\section{Mais secao (numero 17)}

\lipsum[1-5]

\section{E outra secao}

\lipsum[1-5]

\section{mais outra secao}

\lipsum[1-5]

\section{Eu amo secao}
\setcounter{subsection}{3}
\subsection{Algumas formas de evitar e combater o \\ machismo no dia a dia da universidade}

\lipsum[1-5]

%\chapter{Vivendo em Barão \\ {\color{white}. } \hspace{5mm} {\normalsize Barão Geraldo ou BG}}
\chapter{Vivendo em Barão Geraldo}
\lipsum[1-5]

\chapter{Além da Graduação}
\lipsum[1-5]

\chapter{aba}
\section{aoba}
oi

terminal \faTerminal, linux \faLinux, sapo \faFrog

github \faGithub, linkedin \faLinkedin, gitlab \faGitlab, email \faEnvelope, telegram \faPaperPlane[regular] \faTelegram\ \faTelegramPlane, whats \faWhatsapp

map marker \faMapMarker*, restaurantes \faUtensils, cafe \faCoffee, ra \faIdCard[regular] \faAddressCard[regular], casa \faHome

\begin{list}{\faTerminal}{}  
\item item A bla bla bla
\item item B blablabla
\end{list}

\section{Testes de links}


\instagram{meuinstagram}

\telegram{meugrupotelegram}

\discord{meugrupodiscord}

%\pagebreak

\lipsum[1-5]

\begin{tags}
    \github{meuusergithub} \sep \gitlab{meuusergitlab}    
\end{tags}

\end{document}


%\maketitle
%\tableofcontents

%\setcounter{page}{1}

\documentclass{book}

\usepackage[T1]{fontenc}
\usepackage[portuguese]{babel}


\usepackage{manual_template}
% tem que decidir uma fonte legal
% https://tug.org/FontCatalogue/typewriterfonts.html

% para gerar texto lixo
\usepackage{blindtext}
\usepackage{lipsum}

\title{Manual de Ingressante}
\author{Centro Acadêmico da Computação}
\newcommand{\ano}{2024}
\date{\ano}
\newcommand{\shortauthor}{CACo}

\newcommand{\autoresCapa}{%
autor1 (EC021) \\
autor 2 (CC020)
}%

\newcommand{\autoresDiagramacao}{%
autora.e (curso) \\
autora.e (curso) \\
autora.e (curso)
}%

\newcommand{\autoresRevisao}{%
autora.e (curso) \\
autora.e (curso) \\
autora.e (curso) \\
autora.e (curso) \\
autora.e (curso) \\
autora.e (curso)
}%

\newcommand{\gestao}{nome da gestao (\ano)}

\begin{document}

\input{_adendos/main}

%\maketitle
%\tableofcontents

%\setcounter{page}{1}

\include{0_boas_vindas/main}

\include{1_infraestrutura_unicamp/main}

\include{2_burocracias_e_estudos/main}

\chapter{Convivendo na Unicam}
\lipsum[1-5]

\section{whatever}
\lipsum[1-5]


\chapter{outro capitulo}
\lipsum[1-5]

%\chapter{Convivendo na Unicamp\\ {\color{white}. } \hspace{5mm} {\normalsize e no mundo}}
\chapter{Convivendo na Unicamp}
\lipsum[1-5]

\setcounter{section}{17}
\section{Mais secao (numero 17)}

\lipsum[1-5]

\section{E outra secao}

\lipsum[1-5]

\section{mais outra secao}

\lipsum[1-5]

\section{Eu amo secao}
\setcounter{subsection}{3}
\subsection{Algumas formas de evitar e combater o \\ machismo no dia a dia da universidade}

\lipsum[1-5]

%\chapter{Vivendo em Barão \\ {\color{white}. } \hspace{5mm} {\normalsize Barão Geraldo ou BG}}
\chapter{Vivendo em Barão Geraldo}
\lipsum[1-5]

\chapter{Além da Graduação}
\lipsum[1-5]

\chapter{aba}
\section{aoba}
oi

terminal \faTerminal, linux \faLinux, sapo \faFrog

github \faGithub, linkedin \faLinkedin, gitlab \faGitlab, email \faEnvelope, telegram \faPaperPlane[regular] \faTelegram\ \faTelegramPlane, whats \faWhatsapp

map marker \faMapMarker*, restaurantes \faUtensils, cafe \faCoffee, ra \faIdCard[regular] \faAddressCard[regular], casa \faHome

\begin{list}{\faTerminal}{}  
\item item A bla bla bla
\item item B blablabla
\end{list}

\section{Testes de links}


\instagram{meuinstagram}

\telegram{meugrupotelegram}

\discord{meugrupodiscord}

%\pagebreak

\lipsum[1-5]

\begin{tags}
    \github{meuusergithub} \sep \gitlab{meuusergitlab}    
\end{tags}

\end{document}


\documentclass{book}

\usepackage[T1]{fontenc}
\usepackage[portuguese]{babel}


\usepackage{manual_template}
% tem que decidir uma fonte legal
% https://tug.org/FontCatalogue/typewriterfonts.html

% para gerar texto lixo
\usepackage{blindtext}
\usepackage{lipsum}

\title{Manual de Ingressante}
\author{Centro Acadêmico da Computação}
\newcommand{\ano}{2024}
\date{\ano}
\newcommand{\shortauthor}{CACo}

\newcommand{\autoresCapa}{%
autor1 (EC021) \\
autor 2 (CC020)
}%

\newcommand{\autoresDiagramacao}{%
autora.e (curso) \\
autora.e (curso) \\
autora.e (curso)
}%

\newcommand{\autoresRevisao}{%
autora.e (curso) \\
autora.e (curso) \\
autora.e (curso) \\
autora.e (curso) \\
autora.e (curso) \\
autora.e (curso)
}%

\newcommand{\gestao}{nome da gestao (\ano)}

\begin{document}

\input{_adendos/main}

%\maketitle
%\tableofcontents

%\setcounter{page}{1}

\include{0_boas_vindas/main}

\include{1_infraestrutura_unicamp/main}

\include{2_burocracias_e_estudos/main}

\chapter{Convivendo na Unicam}
\lipsum[1-5]

\section{whatever}
\lipsum[1-5]


\chapter{outro capitulo}
\lipsum[1-5]

%\chapter{Convivendo na Unicamp\\ {\color{white}. } \hspace{5mm} {\normalsize e no mundo}}
\chapter{Convivendo na Unicamp}
\lipsum[1-5]

\setcounter{section}{17}
\section{Mais secao (numero 17)}

\lipsum[1-5]

\section{E outra secao}

\lipsum[1-5]

\section{mais outra secao}

\lipsum[1-5]

\section{Eu amo secao}
\setcounter{subsection}{3}
\subsection{Algumas formas de evitar e combater o \\ machismo no dia a dia da universidade}

\lipsum[1-5]

%\chapter{Vivendo em Barão \\ {\color{white}. } \hspace{5mm} {\normalsize Barão Geraldo ou BG}}
\chapter{Vivendo em Barão Geraldo}
\lipsum[1-5]

\chapter{Além da Graduação}
\lipsum[1-5]

\chapter{aba}
\section{aoba}
oi

terminal \faTerminal, linux \faLinux, sapo \faFrog

github \faGithub, linkedin \faLinkedin, gitlab \faGitlab, email \faEnvelope, telegram \faPaperPlane[regular] \faTelegram\ \faTelegramPlane, whats \faWhatsapp

map marker \faMapMarker*, restaurantes \faUtensils, cafe \faCoffee, ra \faIdCard[regular] \faAddressCard[regular], casa \faHome

\begin{list}{\faTerminal}{}  
\item item A bla bla bla
\item item B blablabla
\end{list}

\section{Testes de links}


\instagram{meuinstagram}

\telegram{meugrupotelegram}

\discord{meugrupodiscord}

%\pagebreak

\lipsum[1-5]

\begin{tags}
    \github{meuusergithub} \sep \gitlab{meuusergitlab}    
\end{tags}

\end{document}


\documentclass{book}

\usepackage[T1]{fontenc}
\usepackage[portuguese]{babel}


\usepackage{manual_template}
% tem que decidir uma fonte legal
% https://tug.org/FontCatalogue/typewriterfonts.html

% para gerar texto lixo
\usepackage{blindtext}
\usepackage{lipsum}

\title{Manual de Ingressante}
\author{Centro Acadêmico da Computação}
\newcommand{\ano}{2024}
\date{\ano}
\newcommand{\shortauthor}{CACo}

\newcommand{\autoresCapa}{%
autor1 (EC021) \\
autor 2 (CC020)
}%

\newcommand{\autoresDiagramacao}{%
autora.e (curso) \\
autora.e (curso) \\
autora.e (curso)
}%

\newcommand{\autoresRevisao}{%
autora.e (curso) \\
autora.e (curso) \\
autora.e (curso) \\
autora.e (curso) \\
autora.e (curso) \\
autora.e (curso)
}%

\newcommand{\gestao}{nome da gestao (\ano)}

\begin{document}

\input{_adendos/main}

%\maketitle
%\tableofcontents

%\setcounter{page}{1}

\include{0_boas_vindas/main}

\include{1_infraestrutura_unicamp/main}

\include{2_burocracias_e_estudos/main}

\chapter{Convivendo na Unicam}
\lipsum[1-5]

\section{whatever}
\lipsum[1-5]


\chapter{outro capitulo}
\lipsum[1-5]

%\chapter{Convivendo na Unicamp\\ {\color{white}. } \hspace{5mm} {\normalsize e no mundo}}
\chapter{Convivendo na Unicamp}
\lipsum[1-5]

\setcounter{section}{17}
\section{Mais secao (numero 17)}

\lipsum[1-5]

\section{E outra secao}

\lipsum[1-5]

\section{mais outra secao}

\lipsum[1-5]

\section{Eu amo secao}
\setcounter{subsection}{3}
\subsection{Algumas formas de evitar e combater o \\ machismo no dia a dia da universidade}

\lipsum[1-5]

%\chapter{Vivendo em Barão \\ {\color{white}. } \hspace{5mm} {\normalsize Barão Geraldo ou BG}}
\chapter{Vivendo em Barão Geraldo}
\lipsum[1-5]

\chapter{Além da Graduação}
\lipsum[1-5]

\chapter{aba}
\section{aoba}
oi

terminal \faTerminal, linux \faLinux, sapo \faFrog

github \faGithub, linkedin \faLinkedin, gitlab \faGitlab, email \faEnvelope, telegram \faPaperPlane[regular] \faTelegram\ \faTelegramPlane, whats \faWhatsapp

map marker \faMapMarker*, restaurantes \faUtensils, cafe \faCoffee, ra \faIdCard[regular] \faAddressCard[regular], casa \faHome

\begin{list}{\faTerminal}{}  
\item item A bla bla bla
\item item B blablabla
\end{list}

\section{Testes de links}


\instagram{meuinstagram}

\telegram{meugrupotelegram}

\discord{meugrupodiscord}

%\pagebreak

\lipsum[1-5]

\begin{tags}
    \github{meuusergithub} \sep \gitlab{meuusergitlab}    
\end{tags}

\end{document}


\chapter{Convivendo na Unicam}
\lipsum[1-5]

\section{whatever}
\lipsum[1-5]


\chapter{outro capitulo}
\lipsum[1-5]

%\chapter{Convivendo na Unicamp\\ {\color{white}. } \hspace{5mm} {\normalsize e no mundo}}
\chapter{Convivendo na Unicamp}
\lipsum[1-5]

\setcounter{section}{17}
\section{Mais secao (numero 17)}

\lipsum[1-5]

\section{E outra secao}

\lipsum[1-5]

\section{mais outra secao}

\lipsum[1-5]

\section{Eu amo secao}
\setcounter{subsection}{3}
\subsection{Algumas formas de evitar e combater o \\ machismo no dia a dia da universidade}

\lipsum[1-5]

%\chapter{Vivendo em Barão \\ {\color{white}. } \hspace{5mm} {\normalsize Barão Geraldo ou BG}}
\chapter{Vivendo em Barão Geraldo}
\lipsum[1-5]

\chapter{Além da Graduação}
\lipsum[1-5]

\chapter{aba}
\section{aoba}
oi

terminal \faTerminal, linux \faLinux, sapo \faFrog

github \faGithub, linkedin \faLinkedin, gitlab \faGitlab, email \faEnvelope, telegram \faPaperPlane[regular] \faTelegram\ \faTelegramPlane, whats \faWhatsapp

map marker \faMapMarker*, restaurantes \faUtensils, cafe \faCoffee, ra \faIdCard[regular] \faAddressCard[regular], casa \faHome

\begin{list}{\faTerminal}{}  
\item item A bla bla bla
\item item B blablabla
\end{list}

\section{Testes de links}


\instagram{meuinstagram}

\telegram{meugrupotelegram}

\discord{meugrupodiscord}

%\pagebreak

\lipsum[1-5]

\begin{tags}
    \github{meuusergithub} \sep \gitlab{meuusergitlab}    
\end{tags}

\end{document}


\documentclass{book}

\usepackage[T1]{fontenc}
\usepackage[portuguese]{babel}


\usepackage{manual_template}
% tem que decidir uma fonte legal
% https://tug.org/FontCatalogue/typewriterfonts.html

% para gerar texto lixo
\usepackage{blindtext}
\usepackage{lipsum}

\title{Manual de Ingressante}
\author{Centro Acadêmico da Computação}
\newcommand{\ano}{2024}
\date{\ano}
\newcommand{\shortauthor}{CACo}

\newcommand{\autoresCapa}{%
autor1 (EC021) \\
autor 2 (CC020)
}%

\newcommand{\autoresDiagramacao}{%
autora.e (curso) \\
autora.e (curso) \\
autora.e (curso)
}%

\newcommand{\autoresRevisao}{%
autora.e (curso) \\
autora.e (curso) \\
autora.e (curso) \\
autora.e (curso) \\
autora.e (curso) \\
autora.e (curso)
}%

\newcommand{\gestao}{nome da gestao (\ano)}

\begin{document}

\documentclass{book}

\usepackage[T1]{fontenc}
\usepackage[portuguese]{babel}


\usepackage{manual_template}
% tem que decidir uma fonte legal
% https://tug.org/FontCatalogue/typewriterfonts.html

% para gerar texto lixo
\usepackage{blindtext}
\usepackage{lipsum}

\title{Manual de Ingressante}
\author{Centro Acadêmico da Computação}
\newcommand{\ano}{2024}
\date{\ano}
\newcommand{\shortauthor}{CACo}

\newcommand{\autoresCapa}{%
autor1 (EC021) \\
autor 2 (CC020)
}%

\newcommand{\autoresDiagramacao}{%
autora.e (curso) \\
autora.e (curso) \\
autora.e (curso)
}%

\newcommand{\autoresRevisao}{%
autora.e (curso) \\
autora.e (curso) \\
autora.e (curso) \\
autora.e (curso) \\
autora.e (curso) \\
autora.e (curso)
}%

\newcommand{\gestao}{nome da gestao (\ano)}

\begin{document}

\input{_adendos/main}

%\maketitle
%\tableofcontents

%\setcounter{page}{1}

\include{0_boas_vindas/main}

\include{1_infraestrutura_unicamp/main}

\include{2_burocracias_e_estudos/main}

\chapter{Convivendo na Unicam}
\lipsum[1-5]

\section{whatever}
\lipsum[1-5]


\chapter{outro capitulo}
\lipsum[1-5]

%\chapter{Convivendo na Unicamp\\ {\color{white}. } \hspace{5mm} {\normalsize e no mundo}}
\chapter{Convivendo na Unicamp}
\lipsum[1-5]

\setcounter{section}{17}
\section{Mais secao (numero 17)}

\lipsum[1-5]

\section{E outra secao}

\lipsum[1-5]

\section{mais outra secao}

\lipsum[1-5]

\section{Eu amo secao}
\setcounter{subsection}{3}
\subsection{Algumas formas de evitar e combater o \\ machismo no dia a dia da universidade}

\lipsum[1-5]

%\chapter{Vivendo em Barão \\ {\color{white}. } \hspace{5mm} {\normalsize Barão Geraldo ou BG}}
\chapter{Vivendo em Barão Geraldo}
\lipsum[1-5]

\chapter{Além da Graduação}
\lipsum[1-5]

\chapter{aba}
\section{aoba}
oi

terminal \faTerminal, linux \faLinux, sapo \faFrog

github \faGithub, linkedin \faLinkedin, gitlab \faGitlab, email \faEnvelope, telegram \faPaperPlane[regular] \faTelegram\ \faTelegramPlane, whats \faWhatsapp

map marker \faMapMarker*, restaurantes \faUtensils, cafe \faCoffee, ra \faIdCard[regular] \faAddressCard[regular], casa \faHome

\begin{list}{\faTerminal}{}  
\item item A bla bla bla
\item item B blablabla
\end{list}

\section{Testes de links}


\instagram{meuinstagram}

\telegram{meugrupotelegram}

\discord{meugrupodiscord}

%\pagebreak

\lipsum[1-5]

\begin{tags}
    \github{meuusergithub} \sep \gitlab{meuusergitlab}    
\end{tags}

\end{document}


%\maketitle
%\tableofcontents

%\setcounter{page}{1}

\documentclass{book}

\usepackage[T1]{fontenc}
\usepackage[portuguese]{babel}


\usepackage{manual_template}
% tem que decidir uma fonte legal
% https://tug.org/FontCatalogue/typewriterfonts.html

% para gerar texto lixo
\usepackage{blindtext}
\usepackage{lipsum}

\title{Manual de Ingressante}
\author{Centro Acadêmico da Computação}
\newcommand{\ano}{2024}
\date{\ano}
\newcommand{\shortauthor}{CACo}

\newcommand{\autoresCapa}{%
autor1 (EC021) \\
autor 2 (CC020)
}%

\newcommand{\autoresDiagramacao}{%
autora.e (curso) \\
autora.e (curso) \\
autora.e (curso)
}%

\newcommand{\autoresRevisao}{%
autora.e (curso) \\
autora.e (curso) \\
autora.e (curso) \\
autora.e (curso) \\
autora.e (curso) \\
autora.e (curso)
}%

\newcommand{\gestao}{nome da gestao (\ano)}

\begin{document}

\input{_adendos/main}

%\maketitle
%\tableofcontents

%\setcounter{page}{1}

\include{0_boas_vindas/main}

\include{1_infraestrutura_unicamp/main}

\include{2_burocracias_e_estudos/main}

\chapter{Convivendo na Unicam}
\lipsum[1-5]

\section{whatever}
\lipsum[1-5]


\chapter{outro capitulo}
\lipsum[1-5]

%\chapter{Convivendo na Unicamp\\ {\color{white}. } \hspace{5mm} {\normalsize e no mundo}}
\chapter{Convivendo na Unicamp}
\lipsum[1-5]

\setcounter{section}{17}
\section{Mais secao (numero 17)}

\lipsum[1-5]

\section{E outra secao}

\lipsum[1-5]

\section{mais outra secao}

\lipsum[1-5]

\section{Eu amo secao}
\setcounter{subsection}{3}
\subsection{Algumas formas de evitar e combater o \\ machismo no dia a dia da universidade}

\lipsum[1-5]

%\chapter{Vivendo em Barão \\ {\color{white}. } \hspace{5mm} {\normalsize Barão Geraldo ou BG}}
\chapter{Vivendo em Barão Geraldo}
\lipsum[1-5]

\chapter{Além da Graduação}
\lipsum[1-5]

\chapter{aba}
\section{aoba}
oi

terminal \faTerminal, linux \faLinux, sapo \faFrog

github \faGithub, linkedin \faLinkedin, gitlab \faGitlab, email \faEnvelope, telegram \faPaperPlane[regular] \faTelegram\ \faTelegramPlane, whats \faWhatsapp

map marker \faMapMarker*, restaurantes \faUtensils, cafe \faCoffee, ra \faIdCard[regular] \faAddressCard[regular], casa \faHome

\begin{list}{\faTerminal}{}  
\item item A bla bla bla
\item item B blablabla
\end{list}

\section{Testes de links}


\instagram{meuinstagram}

\telegram{meugrupotelegram}

\discord{meugrupodiscord}

%\pagebreak

\lipsum[1-5]

\begin{tags}
    \github{meuusergithub} \sep \gitlab{meuusergitlab}    
\end{tags}

\end{document}


\documentclass{book}

\usepackage[T1]{fontenc}
\usepackage[portuguese]{babel}


\usepackage{manual_template}
% tem que decidir uma fonte legal
% https://tug.org/FontCatalogue/typewriterfonts.html

% para gerar texto lixo
\usepackage{blindtext}
\usepackage{lipsum}

\title{Manual de Ingressante}
\author{Centro Acadêmico da Computação}
\newcommand{\ano}{2024}
\date{\ano}
\newcommand{\shortauthor}{CACo}

\newcommand{\autoresCapa}{%
autor1 (EC021) \\
autor 2 (CC020)
}%

\newcommand{\autoresDiagramacao}{%
autora.e (curso) \\
autora.e (curso) \\
autora.e (curso)
}%

\newcommand{\autoresRevisao}{%
autora.e (curso) \\
autora.e (curso) \\
autora.e (curso) \\
autora.e (curso) \\
autora.e (curso) \\
autora.e (curso)
}%

\newcommand{\gestao}{nome da gestao (\ano)}

\begin{document}

\input{_adendos/main}

%\maketitle
%\tableofcontents

%\setcounter{page}{1}

\include{0_boas_vindas/main}

\include{1_infraestrutura_unicamp/main}

\include{2_burocracias_e_estudos/main}

\chapter{Convivendo na Unicam}
\lipsum[1-5]

\section{whatever}
\lipsum[1-5]


\chapter{outro capitulo}
\lipsum[1-5]

%\chapter{Convivendo na Unicamp\\ {\color{white}. } \hspace{5mm} {\normalsize e no mundo}}
\chapter{Convivendo na Unicamp}
\lipsum[1-5]

\setcounter{section}{17}
\section{Mais secao (numero 17)}

\lipsum[1-5]

\section{E outra secao}

\lipsum[1-5]

\section{mais outra secao}

\lipsum[1-5]

\section{Eu amo secao}
\setcounter{subsection}{3}
\subsection{Algumas formas de evitar e combater o \\ machismo no dia a dia da universidade}

\lipsum[1-5]

%\chapter{Vivendo em Barão \\ {\color{white}. } \hspace{5mm} {\normalsize Barão Geraldo ou BG}}
\chapter{Vivendo em Barão Geraldo}
\lipsum[1-5]

\chapter{Além da Graduação}
\lipsum[1-5]

\chapter{aba}
\section{aoba}
oi

terminal \faTerminal, linux \faLinux, sapo \faFrog

github \faGithub, linkedin \faLinkedin, gitlab \faGitlab, email \faEnvelope, telegram \faPaperPlane[regular] \faTelegram\ \faTelegramPlane, whats \faWhatsapp

map marker \faMapMarker*, restaurantes \faUtensils, cafe \faCoffee, ra \faIdCard[regular] \faAddressCard[regular], casa \faHome

\begin{list}{\faTerminal}{}  
\item item A bla bla bla
\item item B blablabla
\end{list}

\section{Testes de links}


\instagram{meuinstagram}

\telegram{meugrupotelegram}

\discord{meugrupodiscord}

%\pagebreak

\lipsum[1-5]

\begin{tags}
    \github{meuusergithub} \sep \gitlab{meuusergitlab}    
\end{tags}

\end{document}


\documentclass{book}

\usepackage[T1]{fontenc}
\usepackage[portuguese]{babel}


\usepackage{manual_template}
% tem que decidir uma fonte legal
% https://tug.org/FontCatalogue/typewriterfonts.html

% para gerar texto lixo
\usepackage{blindtext}
\usepackage{lipsum}

\title{Manual de Ingressante}
\author{Centro Acadêmico da Computação}
\newcommand{\ano}{2024}
\date{\ano}
\newcommand{\shortauthor}{CACo}

\newcommand{\autoresCapa}{%
autor1 (EC021) \\
autor 2 (CC020)
}%

\newcommand{\autoresDiagramacao}{%
autora.e (curso) \\
autora.e (curso) \\
autora.e (curso)
}%

\newcommand{\autoresRevisao}{%
autora.e (curso) \\
autora.e (curso) \\
autora.e (curso) \\
autora.e (curso) \\
autora.e (curso) \\
autora.e (curso)
}%

\newcommand{\gestao}{nome da gestao (\ano)}

\begin{document}

\input{_adendos/main}

%\maketitle
%\tableofcontents

%\setcounter{page}{1}

\include{0_boas_vindas/main}

\include{1_infraestrutura_unicamp/main}

\include{2_burocracias_e_estudos/main}

\chapter{Convivendo na Unicam}
\lipsum[1-5]

\section{whatever}
\lipsum[1-5]


\chapter{outro capitulo}
\lipsum[1-5]

%\chapter{Convivendo na Unicamp\\ {\color{white}. } \hspace{5mm} {\normalsize e no mundo}}
\chapter{Convivendo na Unicamp}
\lipsum[1-5]

\setcounter{section}{17}
\section{Mais secao (numero 17)}

\lipsum[1-5]

\section{E outra secao}

\lipsum[1-5]

\section{mais outra secao}

\lipsum[1-5]

\section{Eu amo secao}
\setcounter{subsection}{3}
\subsection{Algumas formas de evitar e combater o \\ machismo no dia a dia da universidade}

\lipsum[1-5]

%\chapter{Vivendo em Barão \\ {\color{white}. } \hspace{5mm} {\normalsize Barão Geraldo ou BG}}
\chapter{Vivendo em Barão Geraldo}
\lipsum[1-5]

\chapter{Além da Graduação}
\lipsum[1-5]

\chapter{aba}
\section{aoba}
oi

terminal \faTerminal, linux \faLinux, sapo \faFrog

github \faGithub, linkedin \faLinkedin, gitlab \faGitlab, email \faEnvelope, telegram \faPaperPlane[regular] \faTelegram\ \faTelegramPlane, whats \faWhatsapp

map marker \faMapMarker*, restaurantes \faUtensils, cafe \faCoffee, ra \faIdCard[regular] \faAddressCard[regular], casa \faHome

\begin{list}{\faTerminal}{}  
\item item A bla bla bla
\item item B blablabla
\end{list}

\section{Testes de links}


\instagram{meuinstagram}

\telegram{meugrupotelegram}

\discord{meugrupodiscord}

%\pagebreak

\lipsum[1-5]

\begin{tags}
    \github{meuusergithub} \sep \gitlab{meuusergitlab}    
\end{tags}

\end{document}


\chapter{Convivendo na Unicam}
\lipsum[1-5]

\section{whatever}
\lipsum[1-5]


\chapter{outro capitulo}
\lipsum[1-5]

%\chapter{Convivendo na Unicamp\\ {\color{white}. } \hspace{5mm} {\normalsize e no mundo}}
\chapter{Convivendo na Unicamp}
\lipsum[1-5]

\setcounter{section}{17}
\section{Mais secao (numero 17)}

\lipsum[1-5]

\section{E outra secao}

\lipsum[1-5]

\section{mais outra secao}

\lipsum[1-5]

\section{Eu amo secao}
\setcounter{subsection}{3}
\subsection{Algumas formas de evitar e combater o \\ machismo no dia a dia da universidade}

\lipsum[1-5]

%\chapter{Vivendo em Barão \\ {\color{white}. } \hspace{5mm} {\normalsize Barão Geraldo ou BG}}
\chapter{Vivendo em Barão Geraldo}
\lipsum[1-5]

\chapter{Além da Graduação}
\lipsum[1-5]

\chapter{aba}
\section{aoba}
oi

terminal \faTerminal, linux \faLinux, sapo \faFrog

github \faGithub, linkedin \faLinkedin, gitlab \faGitlab, email \faEnvelope, telegram \faPaperPlane[regular] \faTelegram\ \faTelegramPlane, whats \faWhatsapp

map marker \faMapMarker*, restaurantes \faUtensils, cafe \faCoffee, ra \faIdCard[regular] \faAddressCard[regular], casa \faHome

\begin{list}{\faTerminal}{}  
\item item A bla bla bla
\item item B blablabla
\end{list}

\section{Testes de links}


\instagram{meuinstagram}

\telegram{meugrupotelegram}

\discord{meugrupodiscord}

%\pagebreak

\lipsum[1-5]

\begin{tags}
    \github{meuusergithub} \sep \gitlab{meuusergitlab}    
\end{tags}

\end{document}


\chapter{Convivendo na Unicam}
\lipsum[1-5]

\section{whatever}
\lipsum[1-5]


\chapter{outro capitulo}
\lipsum[1-5]

%\chapter{Convivendo na Unicamp\\ {\color{white}. } \hspace{5mm} {\normalsize e no mundo}}
\chapter{Convivendo na Unicamp}
\lipsum[1-5]

\setcounter{section}{17}
\section{Mais secao (numero 17)}

\lipsum[1-5]

\section{E outra secao}

\lipsum[1-5]

\section{mais outra secao}

\lipsum[1-5]

\section{Eu amo secao}
\setcounter{subsection}{3}
\subsection{Algumas formas de evitar e combater o \\ machismo no dia a dia da universidade}

\lipsum[1-5]

%\chapter{Vivendo em Barão \\ {\color{white}. } \hspace{5mm} {\normalsize Barão Geraldo ou BG}}
\chapter{Vivendo em Barão Geraldo}
\lipsum[1-5]

\chapter{Além da Graduação}
\lipsum[1-5]

\chapter{aba}
\section{aoba}
oi

terminal \faTerminal, linux \faLinux, sapo \faFrog

github \faGithub, linkedin \faLinkedin, gitlab \faGitlab, email \faEnvelope, telegram \faPaperPlane[regular] \faTelegram\ \faTelegramPlane, whats \faWhatsapp

map marker \faMapMarker*, restaurantes \faUtensils, cafe \faCoffee, ra \faIdCard[regular] \faAddressCard[regular], casa \faHome

\begin{list}{\faTerminal}{}  
\item item A bla bla bla
\item item B blablabla
\end{list}

\section{Testes de links}


\instagram{meuinstagram}

\telegram{meugrupotelegram}

\discord{meugrupodiscord}

%\pagebreak

\lipsum[1-5]

\begin{tags}
    \github{meuusergithub} \sep \gitlab{meuusergitlab}    
\end{tags}

\end{document}


\documentclass{book}

\usepackage[T1]{fontenc}
\usepackage[portuguese]{babel}


\usepackage{manual_template}
% tem que decidir uma fonte legal
% https://tug.org/FontCatalogue/typewriterfonts.html

% para gerar texto lixo
\usepackage{blindtext}
\usepackage{lipsum}

\title{Manual de Ingressante}
\author{Centro Acadêmico da Computação}
\newcommand{\ano}{2024}
\date{\ano}
\newcommand{\shortauthor}{CACo}

\newcommand{\autoresCapa}{%
autor1 (EC021) \\
autor 2 (CC020)
}%

\newcommand{\autoresDiagramacao}{%
autora.e (curso) \\
autora.e (curso) \\
autora.e (curso)
}%

\newcommand{\autoresRevisao}{%
autora.e (curso) \\
autora.e (curso) \\
autora.e (curso) \\
autora.e (curso) \\
autora.e (curso) \\
autora.e (curso)
}%

\newcommand{\gestao}{nome da gestao (\ano)}

\begin{document}

\documentclass{book}

\usepackage[T1]{fontenc}
\usepackage[portuguese]{babel}


\usepackage{manual_template}
% tem que decidir uma fonte legal
% https://tug.org/FontCatalogue/typewriterfonts.html

% para gerar texto lixo
\usepackage{blindtext}
\usepackage{lipsum}

\title{Manual de Ingressante}
\author{Centro Acadêmico da Computação}
\newcommand{\ano}{2024}
\date{\ano}
\newcommand{\shortauthor}{CACo}

\newcommand{\autoresCapa}{%
autor1 (EC021) \\
autor 2 (CC020)
}%

\newcommand{\autoresDiagramacao}{%
autora.e (curso) \\
autora.e (curso) \\
autora.e (curso)
}%

\newcommand{\autoresRevisao}{%
autora.e (curso) \\
autora.e (curso) \\
autora.e (curso) \\
autora.e (curso) \\
autora.e (curso) \\
autora.e (curso)
}%

\newcommand{\gestao}{nome da gestao (\ano)}

\begin{document}

\documentclass{book}

\usepackage[T1]{fontenc}
\usepackage[portuguese]{babel}


\usepackage{manual_template}
% tem que decidir uma fonte legal
% https://tug.org/FontCatalogue/typewriterfonts.html

% para gerar texto lixo
\usepackage{blindtext}
\usepackage{lipsum}

\title{Manual de Ingressante}
\author{Centro Acadêmico da Computação}
\newcommand{\ano}{2024}
\date{\ano}
\newcommand{\shortauthor}{CACo}

\newcommand{\autoresCapa}{%
autor1 (EC021) \\
autor 2 (CC020)
}%

\newcommand{\autoresDiagramacao}{%
autora.e (curso) \\
autora.e (curso) \\
autora.e (curso)
}%

\newcommand{\autoresRevisao}{%
autora.e (curso) \\
autora.e (curso) \\
autora.e (curso) \\
autora.e (curso) \\
autora.e (curso) \\
autora.e (curso)
}%

\newcommand{\gestao}{nome da gestao (\ano)}

\begin{document}

\input{_adendos/main}

%\maketitle
%\tableofcontents

%\setcounter{page}{1}

\include{0_boas_vindas/main}

\include{1_infraestrutura_unicamp/main}

\include{2_burocracias_e_estudos/main}

\chapter{Convivendo na Unicam}
\lipsum[1-5]

\section{whatever}
\lipsum[1-5]


\chapter{outro capitulo}
\lipsum[1-5]

%\chapter{Convivendo na Unicamp\\ {\color{white}. } \hspace{5mm} {\normalsize e no mundo}}
\chapter{Convivendo na Unicamp}
\lipsum[1-5]

\setcounter{section}{17}
\section{Mais secao (numero 17)}

\lipsum[1-5]

\section{E outra secao}

\lipsum[1-5]

\section{mais outra secao}

\lipsum[1-5]

\section{Eu amo secao}
\setcounter{subsection}{3}
\subsection{Algumas formas de evitar e combater o \\ machismo no dia a dia da universidade}

\lipsum[1-5]

%\chapter{Vivendo em Barão \\ {\color{white}. } \hspace{5mm} {\normalsize Barão Geraldo ou BG}}
\chapter{Vivendo em Barão Geraldo}
\lipsum[1-5]

\chapter{Além da Graduação}
\lipsum[1-5]

\chapter{aba}
\section{aoba}
oi

terminal \faTerminal, linux \faLinux, sapo \faFrog

github \faGithub, linkedin \faLinkedin, gitlab \faGitlab, email \faEnvelope, telegram \faPaperPlane[regular] \faTelegram\ \faTelegramPlane, whats \faWhatsapp

map marker \faMapMarker*, restaurantes \faUtensils, cafe \faCoffee, ra \faIdCard[regular] \faAddressCard[regular], casa \faHome

\begin{list}{\faTerminal}{}  
\item item A bla bla bla
\item item B blablabla
\end{list}

\section{Testes de links}


\instagram{meuinstagram}

\telegram{meugrupotelegram}

\discord{meugrupodiscord}

%\pagebreak

\lipsum[1-5]

\begin{tags}
    \github{meuusergithub} \sep \gitlab{meuusergitlab}    
\end{tags}

\end{document}


%\maketitle
%\tableofcontents

%\setcounter{page}{1}

\documentclass{book}

\usepackage[T1]{fontenc}
\usepackage[portuguese]{babel}


\usepackage{manual_template}
% tem que decidir uma fonte legal
% https://tug.org/FontCatalogue/typewriterfonts.html

% para gerar texto lixo
\usepackage{blindtext}
\usepackage{lipsum}

\title{Manual de Ingressante}
\author{Centro Acadêmico da Computação}
\newcommand{\ano}{2024}
\date{\ano}
\newcommand{\shortauthor}{CACo}

\newcommand{\autoresCapa}{%
autor1 (EC021) \\
autor 2 (CC020)
}%

\newcommand{\autoresDiagramacao}{%
autora.e (curso) \\
autora.e (curso) \\
autora.e (curso)
}%

\newcommand{\autoresRevisao}{%
autora.e (curso) \\
autora.e (curso) \\
autora.e (curso) \\
autora.e (curso) \\
autora.e (curso) \\
autora.e (curso)
}%

\newcommand{\gestao}{nome da gestao (\ano)}

\begin{document}

\input{_adendos/main}

%\maketitle
%\tableofcontents

%\setcounter{page}{1}

\include{0_boas_vindas/main}

\include{1_infraestrutura_unicamp/main}

\include{2_burocracias_e_estudos/main}

\chapter{Convivendo na Unicam}
\lipsum[1-5]

\section{whatever}
\lipsum[1-5]


\chapter{outro capitulo}
\lipsum[1-5]

%\chapter{Convivendo na Unicamp\\ {\color{white}. } \hspace{5mm} {\normalsize e no mundo}}
\chapter{Convivendo na Unicamp}
\lipsum[1-5]

\setcounter{section}{17}
\section{Mais secao (numero 17)}

\lipsum[1-5]

\section{E outra secao}

\lipsum[1-5]

\section{mais outra secao}

\lipsum[1-5]

\section{Eu amo secao}
\setcounter{subsection}{3}
\subsection{Algumas formas de evitar e combater o \\ machismo no dia a dia da universidade}

\lipsum[1-5]

%\chapter{Vivendo em Barão \\ {\color{white}. } \hspace{5mm} {\normalsize Barão Geraldo ou BG}}
\chapter{Vivendo em Barão Geraldo}
\lipsum[1-5]

\chapter{Além da Graduação}
\lipsum[1-5]

\chapter{aba}
\section{aoba}
oi

terminal \faTerminal, linux \faLinux, sapo \faFrog

github \faGithub, linkedin \faLinkedin, gitlab \faGitlab, email \faEnvelope, telegram \faPaperPlane[regular] \faTelegram\ \faTelegramPlane, whats \faWhatsapp

map marker \faMapMarker*, restaurantes \faUtensils, cafe \faCoffee, ra \faIdCard[regular] \faAddressCard[regular], casa \faHome

\begin{list}{\faTerminal}{}  
\item item A bla bla bla
\item item B blablabla
\end{list}

\section{Testes de links}


\instagram{meuinstagram}

\telegram{meugrupotelegram}

\discord{meugrupodiscord}

%\pagebreak

\lipsum[1-5]

\begin{tags}
    \github{meuusergithub} \sep \gitlab{meuusergitlab}    
\end{tags}

\end{document}


\documentclass{book}

\usepackage[T1]{fontenc}
\usepackage[portuguese]{babel}


\usepackage{manual_template}
% tem que decidir uma fonte legal
% https://tug.org/FontCatalogue/typewriterfonts.html

% para gerar texto lixo
\usepackage{blindtext}
\usepackage{lipsum}

\title{Manual de Ingressante}
\author{Centro Acadêmico da Computação}
\newcommand{\ano}{2024}
\date{\ano}
\newcommand{\shortauthor}{CACo}

\newcommand{\autoresCapa}{%
autor1 (EC021) \\
autor 2 (CC020)
}%

\newcommand{\autoresDiagramacao}{%
autora.e (curso) \\
autora.e (curso) \\
autora.e (curso)
}%

\newcommand{\autoresRevisao}{%
autora.e (curso) \\
autora.e (curso) \\
autora.e (curso) \\
autora.e (curso) \\
autora.e (curso) \\
autora.e (curso)
}%

\newcommand{\gestao}{nome da gestao (\ano)}

\begin{document}

\input{_adendos/main}

%\maketitle
%\tableofcontents

%\setcounter{page}{1}

\include{0_boas_vindas/main}

\include{1_infraestrutura_unicamp/main}

\include{2_burocracias_e_estudos/main}

\chapter{Convivendo na Unicam}
\lipsum[1-5]

\section{whatever}
\lipsum[1-5]


\chapter{outro capitulo}
\lipsum[1-5]

%\chapter{Convivendo na Unicamp\\ {\color{white}. } \hspace{5mm} {\normalsize e no mundo}}
\chapter{Convivendo na Unicamp}
\lipsum[1-5]

\setcounter{section}{17}
\section{Mais secao (numero 17)}

\lipsum[1-5]

\section{E outra secao}

\lipsum[1-5]

\section{mais outra secao}

\lipsum[1-5]

\section{Eu amo secao}
\setcounter{subsection}{3}
\subsection{Algumas formas de evitar e combater o \\ machismo no dia a dia da universidade}

\lipsum[1-5]

%\chapter{Vivendo em Barão \\ {\color{white}. } \hspace{5mm} {\normalsize Barão Geraldo ou BG}}
\chapter{Vivendo em Barão Geraldo}
\lipsum[1-5]

\chapter{Além da Graduação}
\lipsum[1-5]

\chapter{aba}
\section{aoba}
oi

terminal \faTerminal, linux \faLinux, sapo \faFrog

github \faGithub, linkedin \faLinkedin, gitlab \faGitlab, email \faEnvelope, telegram \faPaperPlane[regular] \faTelegram\ \faTelegramPlane, whats \faWhatsapp

map marker \faMapMarker*, restaurantes \faUtensils, cafe \faCoffee, ra \faIdCard[regular] \faAddressCard[regular], casa \faHome

\begin{list}{\faTerminal}{}  
\item item A bla bla bla
\item item B blablabla
\end{list}

\section{Testes de links}


\instagram{meuinstagram}

\telegram{meugrupotelegram}

\discord{meugrupodiscord}

%\pagebreak

\lipsum[1-5]

\begin{tags}
    \github{meuusergithub} \sep \gitlab{meuusergitlab}    
\end{tags}

\end{document}


\documentclass{book}

\usepackage[T1]{fontenc}
\usepackage[portuguese]{babel}


\usepackage{manual_template}
% tem que decidir uma fonte legal
% https://tug.org/FontCatalogue/typewriterfonts.html

% para gerar texto lixo
\usepackage{blindtext}
\usepackage{lipsum}

\title{Manual de Ingressante}
\author{Centro Acadêmico da Computação}
\newcommand{\ano}{2024}
\date{\ano}
\newcommand{\shortauthor}{CACo}

\newcommand{\autoresCapa}{%
autor1 (EC021) \\
autor 2 (CC020)
}%

\newcommand{\autoresDiagramacao}{%
autora.e (curso) \\
autora.e (curso) \\
autora.e (curso)
}%

\newcommand{\autoresRevisao}{%
autora.e (curso) \\
autora.e (curso) \\
autora.e (curso) \\
autora.e (curso) \\
autora.e (curso) \\
autora.e (curso)
}%

\newcommand{\gestao}{nome da gestao (\ano)}

\begin{document}

\input{_adendos/main}

%\maketitle
%\tableofcontents

%\setcounter{page}{1}

\include{0_boas_vindas/main}

\include{1_infraestrutura_unicamp/main}

\include{2_burocracias_e_estudos/main}

\chapter{Convivendo na Unicam}
\lipsum[1-5]

\section{whatever}
\lipsum[1-5]


\chapter{outro capitulo}
\lipsum[1-5]

%\chapter{Convivendo na Unicamp\\ {\color{white}. } \hspace{5mm} {\normalsize e no mundo}}
\chapter{Convivendo na Unicamp}
\lipsum[1-5]

\setcounter{section}{17}
\section{Mais secao (numero 17)}

\lipsum[1-5]

\section{E outra secao}

\lipsum[1-5]

\section{mais outra secao}

\lipsum[1-5]

\section{Eu amo secao}
\setcounter{subsection}{3}
\subsection{Algumas formas de evitar e combater o \\ machismo no dia a dia da universidade}

\lipsum[1-5]

%\chapter{Vivendo em Barão \\ {\color{white}. } \hspace{5mm} {\normalsize Barão Geraldo ou BG}}
\chapter{Vivendo em Barão Geraldo}
\lipsum[1-5]

\chapter{Além da Graduação}
\lipsum[1-5]

\chapter{aba}
\section{aoba}
oi

terminal \faTerminal, linux \faLinux, sapo \faFrog

github \faGithub, linkedin \faLinkedin, gitlab \faGitlab, email \faEnvelope, telegram \faPaperPlane[regular] \faTelegram\ \faTelegramPlane, whats \faWhatsapp

map marker \faMapMarker*, restaurantes \faUtensils, cafe \faCoffee, ra \faIdCard[regular] \faAddressCard[regular], casa \faHome

\begin{list}{\faTerminal}{}  
\item item A bla bla bla
\item item B blablabla
\end{list}

\section{Testes de links}


\instagram{meuinstagram}

\telegram{meugrupotelegram}

\discord{meugrupodiscord}

%\pagebreak

\lipsum[1-5]

\begin{tags}
    \github{meuusergithub} \sep \gitlab{meuusergitlab}    
\end{tags}

\end{document}


\chapter{Convivendo na Unicam}
\lipsum[1-5]

\section{whatever}
\lipsum[1-5]


\chapter{outro capitulo}
\lipsum[1-5]

%\chapter{Convivendo na Unicamp\\ {\color{white}. } \hspace{5mm} {\normalsize e no mundo}}
\chapter{Convivendo na Unicamp}
\lipsum[1-5]

\setcounter{section}{17}
\section{Mais secao (numero 17)}

\lipsum[1-5]

\section{E outra secao}

\lipsum[1-5]

\section{mais outra secao}

\lipsum[1-5]

\section{Eu amo secao}
\setcounter{subsection}{3}
\subsection{Algumas formas de evitar e combater o \\ machismo no dia a dia da universidade}

\lipsum[1-5]

%\chapter{Vivendo em Barão \\ {\color{white}. } \hspace{5mm} {\normalsize Barão Geraldo ou BG}}
\chapter{Vivendo em Barão Geraldo}
\lipsum[1-5]

\chapter{Além da Graduação}
\lipsum[1-5]

\chapter{aba}
\section{aoba}
oi

terminal \faTerminal, linux \faLinux, sapo \faFrog

github \faGithub, linkedin \faLinkedin, gitlab \faGitlab, email \faEnvelope, telegram \faPaperPlane[regular] \faTelegram\ \faTelegramPlane, whats \faWhatsapp

map marker \faMapMarker*, restaurantes \faUtensils, cafe \faCoffee, ra \faIdCard[regular] \faAddressCard[regular], casa \faHome

\begin{list}{\faTerminal}{}  
\item item A bla bla bla
\item item B blablabla
\end{list}

\section{Testes de links}


\instagram{meuinstagram}

\telegram{meugrupotelegram}

\discord{meugrupodiscord}

%\pagebreak

\lipsum[1-5]

\begin{tags}
    \github{meuusergithub} \sep \gitlab{meuusergitlab}    
\end{tags}

\end{document}


%\maketitle
%\tableofcontents

%\setcounter{page}{1}

\documentclass{book}

\usepackage[T1]{fontenc}
\usepackage[portuguese]{babel}


\usepackage{manual_template}
% tem que decidir uma fonte legal
% https://tug.org/FontCatalogue/typewriterfonts.html

% para gerar texto lixo
\usepackage{blindtext}
\usepackage{lipsum}

\title{Manual de Ingressante}
\author{Centro Acadêmico da Computação}
\newcommand{\ano}{2024}
\date{\ano}
\newcommand{\shortauthor}{CACo}

\newcommand{\autoresCapa}{%
autor1 (EC021) \\
autor 2 (CC020)
}%

\newcommand{\autoresDiagramacao}{%
autora.e (curso) \\
autora.e (curso) \\
autora.e (curso)
}%

\newcommand{\autoresRevisao}{%
autora.e (curso) \\
autora.e (curso) \\
autora.e (curso) \\
autora.e (curso) \\
autora.e (curso) \\
autora.e (curso)
}%

\newcommand{\gestao}{nome da gestao (\ano)}

\begin{document}

\documentclass{book}

\usepackage[T1]{fontenc}
\usepackage[portuguese]{babel}


\usepackage{manual_template}
% tem que decidir uma fonte legal
% https://tug.org/FontCatalogue/typewriterfonts.html

% para gerar texto lixo
\usepackage{blindtext}
\usepackage{lipsum}

\title{Manual de Ingressante}
\author{Centro Acadêmico da Computação}
\newcommand{\ano}{2024}
\date{\ano}
\newcommand{\shortauthor}{CACo}

\newcommand{\autoresCapa}{%
autor1 (EC021) \\
autor 2 (CC020)
}%

\newcommand{\autoresDiagramacao}{%
autora.e (curso) \\
autora.e (curso) \\
autora.e (curso)
}%

\newcommand{\autoresRevisao}{%
autora.e (curso) \\
autora.e (curso) \\
autora.e (curso) \\
autora.e (curso) \\
autora.e (curso) \\
autora.e (curso)
}%

\newcommand{\gestao}{nome da gestao (\ano)}

\begin{document}

\input{_adendos/main}

%\maketitle
%\tableofcontents

%\setcounter{page}{1}

\include{0_boas_vindas/main}

\include{1_infraestrutura_unicamp/main}

\include{2_burocracias_e_estudos/main}

\chapter{Convivendo na Unicam}
\lipsum[1-5]

\section{whatever}
\lipsum[1-5]


\chapter{outro capitulo}
\lipsum[1-5]

%\chapter{Convivendo na Unicamp\\ {\color{white}. } \hspace{5mm} {\normalsize e no mundo}}
\chapter{Convivendo na Unicamp}
\lipsum[1-5]

\setcounter{section}{17}
\section{Mais secao (numero 17)}

\lipsum[1-5]

\section{E outra secao}

\lipsum[1-5]

\section{mais outra secao}

\lipsum[1-5]

\section{Eu amo secao}
\setcounter{subsection}{3}
\subsection{Algumas formas de evitar e combater o \\ machismo no dia a dia da universidade}

\lipsum[1-5]

%\chapter{Vivendo em Barão \\ {\color{white}. } \hspace{5mm} {\normalsize Barão Geraldo ou BG}}
\chapter{Vivendo em Barão Geraldo}
\lipsum[1-5]

\chapter{Além da Graduação}
\lipsum[1-5]

\chapter{aba}
\section{aoba}
oi

terminal \faTerminal, linux \faLinux, sapo \faFrog

github \faGithub, linkedin \faLinkedin, gitlab \faGitlab, email \faEnvelope, telegram \faPaperPlane[regular] \faTelegram\ \faTelegramPlane, whats \faWhatsapp

map marker \faMapMarker*, restaurantes \faUtensils, cafe \faCoffee, ra \faIdCard[regular] \faAddressCard[regular], casa \faHome

\begin{list}{\faTerminal}{}  
\item item A bla bla bla
\item item B blablabla
\end{list}

\section{Testes de links}


\instagram{meuinstagram}

\telegram{meugrupotelegram}

\discord{meugrupodiscord}

%\pagebreak

\lipsum[1-5]

\begin{tags}
    \github{meuusergithub} \sep \gitlab{meuusergitlab}    
\end{tags}

\end{document}


%\maketitle
%\tableofcontents

%\setcounter{page}{1}

\documentclass{book}

\usepackage[T1]{fontenc}
\usepackage[portuguese]{babel}


\usepackage{manual_template}
% tem que decidir uma fonte legal
% https://tug.org/FontCatalogue/typewriterfonts.html

% para gerar texto lixo
\usepackage{blindtext}
\usepackage{lipsum}

\title{Manual de Ingressante}
\author{Centro Acadêmico da Computação}
\newcommand{\ano}{2024}
\date{\ano}
\newcommand{\shortauthor}{CACo}

\newcommand{\autoresCapa}{%
autor1 (EC021) \\
autor 2 (CC020)
}%

\newcommand{\autoresDiagramacao}{%
autora.e (curso) \\
autora.e (curso) \\
autora.e (curso)
}%

\newcommand{\autoresRevisao}{%
autora.e (curso) \\
autora.e (curso) \\
autora.e (curso) \\
autora.e (curso) \\
autora.e (curso) \\
autora.e (curso)
}%

\newcommand{\gestao}{nome da gestao (\ano)}

\begin{document}

\input{_adendos/main}

%\maketitle
%\tableofcontents

%\setcounter{page}{1}

\include{0_boas_vindas/main}

\include{1_infraestrutura_unicamp/main}

\include{2_burocracias_e_estudos/main}

\chapter{Convivendo na Unicam}
\lipsum[1-5]

\section{whatever}
\lipsum[1-5]


\chapter{outro capitulo}
\lipsum[1-5]

%\chapter{Convivendo na Unicamp\\ {\color{white}. } \hspace{5mm} {\normalsize e no mundo}}
\chapter{Convivendo na Unicamp}
\lipsum[1-5]

\setcounter{section}{17}
\section{Mais secao (numero 17)}

\lipsum[1-5]

\section{E outra secao}

\lipsum[1-5]

\section{mais outra secao}

\lipsum[1-5]

\section{Eu amo secao}
\setcounter{subsection}{3}
\subsection{Algumas formas de evitar e combater o \\ machismo no dia a dia da universidade}

\lipsum[1-5]

%\chapter{Vivendo em Barão \\ {\color{white}. } \hspace{5mm} {\normalsize Barão Geraldo ou BG}}
\chapter{Vivendo em Barão Geraldo}
\lipsum[1-5]

\chapter{Além da Graduação}
\lipsum[1-5]

\chapter{aba}
\section{aoba}
oi

terminal \faTerminal, linux \faLinux, sapo \faFrog

github \faGithub, linkedin \faLinkedin, gitlab \faGitlab, email \faEnvelope, telegram \faPaperPlane[regular] \faTelegram\ \faTelegramPlane, whats \faWhatsapp

map marker \faMapMarker*, restaurantes \faUtensils, cafe \faCoffee, ra \faIdCard[regular] \faAddressCard[regular], casa \faHome

\begin{list}{\faTerminal}{}  
\item item A bla bla bla
\item item B blablabla
\end{list}

\section{Testes de links}


\instagram{meuinstagram}

\telegram{meugrupotelegram}

\discord{meugrupodiscord}

%\pagebreak

\lipsum[1-5]

\begin{tags}
    \github{meuusergithub} \sep \gitlab{meuusergitlab}    
\end{tags}

\end{document}


\documentclass{book}

\usepackage[T1]{fontenc}
\usepackage[portuguese]{babel}


\usepackage{manual_template}
% tem que decidir uma fonte legal
% https://tug.org/FontCatalogue/typewriterfonts.html

% para gerar texto lixo
\usepackage{blindtext}
\usepackage{lipsum}

\title{Manual de Ingressante}
\author{Centro Acadêmico da Computação}
\newcommand{\ano}{2024}
\date{\ano}
\newcommand{\shortauthor}{CACo}

\newcommand{\autoresCapa}{%
autor1 (EC021) \\
autor 2 (CC020)
}%

\newcommand{\autoresDiagramacao}{%
autora.e (curso) \\
autora.e (curso) \\
autora.e (curso)
}%

\newcommand{\autoresRevisao}{%
autora.e (curso) \\
autora.e (curso) \\
autora.e (curso) \\
autora.e (curso) \\
autora.e (curso) \\
autora.e (curso)
}%

\newcommand{\gestao}{nome da gestao (\ano)}

\begin{document}

\input{_adendos/main}

%\maketitle
%\tableofcontents

%\setcounter{page}{1}

\include{0_boas_vindas/main}

\include{1_infraestrutura_unicamp/main}

\include{2_burocracias_e_estudos/main}

\chapter{Convivendo na Unicam}
\lipsum[1-5]

\section{whatever}
\lipsum[1-5]


\chapter{outro capitulo}
\lipsum[1-5]

%\chapter{Convivendo na Unicamp\\ {\color{white}. } \hspace{5mm} {\normalsize e no mundo}}
\chapter{Convivendo na Unicamp}
\lipsum[1-5]

\setcounter{section}{17}
\section{Mais secao (numero 17)}

\lipsum[1-5]

\section{E outra secao}

\lipsum[1-5]

\section{mais outra secao}

\lipsum[1-5]

\section{Eu amo secao}
\setcounter{subsection}{3}
\subsection{Algumas formas de evitar e combater o \\ machismo no dia a dia da universidade}

\lipsum[1-5]

%\chapter{Vivendo em Barão \\ {\color{white}. } \hspace{5mm} {\normalsize Barão Geraldo ou BG}}
\chapter{Vivendo em Barão Geraldo}
\lipsum[1-5]

\chapter{Além da Graduação}
\lipsum[1-5]

\chapter{aba}
\section{aoba}
oi

terminal \faTerminal, linux \faLinux, sapo \faFrog

github \faGithub, linkedin \faLinkedin, gitlab \faGitlab, email \faEnvelope, telegram \faPaperPlane[regular] \faTelegram\ \faTelegramPlane, whats \faWhatsapp

map marker \faMapMarker*, restaurantes \faUtensils, cafe \faCoffee, ra \faIdCard[regular] \faAddressCard[regular], casa \faHome

\begin{list}{\faTerminal}{}  
\item item A bla bla bla
\item item B blablabla
\end{list}

\section{Testes de links}


\instagram{meuinstagram}

\telegram{meugrupotelegram}

\discord{meugrupodiscord}

%\pagebreak

\lipsum[1-5]

\begin{tags}
    \github{meuusergithub} \sep \gitlab{meuusergitlab}    
\end{tags}

\end{document}


\documentclass{book}

\usepackage[T1]{fontenc}
\usepackage[portuguese]{babel}


\usepackage{manual_template}
% tem que decidir uma fonte legal
% https://tug.org/FontCatalogue/typewriterfonts.html

% para gerar texto lixo
\usepackage{blindtext}
\usepackage{lipsum}

\title{Manual de Ingressante}
\author{Centro Acadêmico da Computação}
\newcommand{\ano}{2024}
\date{\ano}
\newcommand{\shortauthor}{CACo}

\newcommand{\autoresCapa}{%
autor1 (EC021) \\
autor 2 (CC020)
}%

\newcommand{\autoresDiagramacao}{%
autora.e (curso) \\
autora.e (curso) \\
autora.e (curso)
}%

\newcommand{\autoresRevisao}{%
autora.e (curso) \\
autora.e (curso) \\
autora.e (curso) \\
autora.e (curso) \\
autora.e (curso) \\
autora.e (curso)
}%

\newcommand{\gestao}{nome da gestao (\ano)}

\begin{document}

\input{_adendos/main}

%\maketitle
%\tableofcontents

%\setcounter{page}{1}

\include{0_boas_vindas/main}

\include{1_infraestrutura_unicamp/main}

\include{2_burocracias_e_estudos/main}

\chapter{Convivendo na Unicam}
\lipsum[1-5]

\section{whatever}
\lipsum[1-5]


\chapter{outro capitulo}
\lipsum[1-5]

%\chapter{Convivendo na Unicamp\\ {\color{white}. } \hspace{5mm} {\normalsize e no mundo}}
\chapter{Convivendo na Unicamp}
\lipsum[1-5]

\setcounter{section}{17}
\section{Mais secao (numero 17)}

\lipsum[1-5]

\section{E outra secao}

\lipsum[1-5]

\section{mais outra secao}

\lipsum[1-5]

\section{Eu amo secao}
\setcounter{subsection}{3}
\subsection{Algumas formas de evitar e combater o \\ machismo no dia a dia da universidade}

\lipsum[1-5]

%\chapter{Vivendo em Barão \\ {\color{white}. } \hspace{5mm} {\normalsize Barão Geraldo ou BG}}
\chapter{Vivendo em Barão Geraldo}
\lipsum[1-5]

\chapter{Além da Graduação}
\lipsum[1-5]

\chapter{aba}
\section{aoba}
oi

terminal \faTerminal, linux \faLinux, sapo \faFrog

github \faGithub, linkedin \faLinkedin, gitlab \faGitlab, email \faEnvelope, telegram \faPaperPlane[regular] \faTelegram\ \faTelegramPlane, whats \faWhatsapp

map marker \faMapMarker*, restaurantes \faUtensils, cafe \faCoffee, ra \faIdCard[regular] \faAddressCard[regular], casa \faHome

\begin{list}{\faTerminal}{}  
\item item A bla bla bla
\item item B blablabla
\end{list}

\section{Testes de links}


\instagram{meuinstagram}

\telegram{meugrupotelegram}

\discord{meugrupodiscord}

%\pagebreak

\lipsum[1-5]

\begin{tags}
    \github{meuusergithub} \sep \gitlab{meuusergitlab}    
\end{tags}

\end{document}


\chapter{Convivendo na Unicam}
\lipsum[1-5]

\section{whatever}
\lipsum[1-5]


\chapter{outro capitulo}
\lipsum[1-5]

%\chapter{Convivendo na Unicamp\\ {\color{white}. } \hspace{5mm} {\normalsize e no mundo}}
\chapter{Convivendo na Unicamp}
\lipsum[1-5]

\setcounter{section}{17}
\section{Mais secao (numero 17)}

\lipsum[1-5]

\section{E outra secao}

\lipsum[1-5]

\section{mais outra secao}

\lipsum[1-5]

\section{Eu amo secao}
\setcounter{subsection}{3}
\subsection{Algumas formas de evitar e combater o \\ machismo no dia a dia da universidade}

\lipsum[1-5]

%\chapter{Vivendo em Barão \\ {\color{white}. } \hspace{5mm} {\normalsize Barão Geraldo ou BG}}
\chapter{Vivendo em Barão Geraldo}
\lipsum[1-5]

\chapter{Além da Graduação}
\lipsum[1-5]

\chapter{aba}
\section{aoba}
oi

terminal \faTerminal, linux \faLinux, sapo \faFrog

github \faGithub, linkedin \faLinkedin, gitlab \faGitlab, email \faEnvelope, telegram \faPaperPlane[regular] \faTelegram\ \faTelegramPlane, whats \faWhatsapp

map marker \faMapMarker*, restaurantes \faUtensils, cafe \faCoffee, ra \faIdCard[regular] \faAddressCard[regular], casa \faHome

\begin{list}{\faTerminal}{}  
\item item A bla bla bla
\item item B blablabla
\end{list}

\section{Testes de links}


\instagram{meuinstagram}

\telegram{meugrupotelegram}

\discord{meugrupodiscord}

%\pagebreak

\lipsum[1-5]

\begin{tags}
    \github{meuusergithub} \sep \gitlab{meuusergitlab}    
\end{tags}

\end{document}


\documentclass{book}

\usepackage[T1]{fontenc}
\usepackage[portuguese]{babel}


\usepackage{manual_template}
% tem que decidir uma fonte legal
% https://tug.org/FontCatalogue/typewriterfonts.html

% para gerar texto lixo
\usepackage{blindtext}
\usepackage{lipsum}

\title{Manual de Ingressante}
\author{Centro Acadêmico da Computação}
\newcommand{\ano}{2024}
\date{\ano}
\newcommand{\shortauthor}{CACo}

\newcommand{\autoresCapa}{%
autor1 (EC021) \\
autor 2 (CC020)
}%

\newcommand{\autoresDiagramacao}{%
autora.e (curso) \\
autora.e (curso) \\
autora.e (curso)
}%

\newcommand{\autoresRevisao}{%
autora.e (curso) \\
autora.e (curso) \\
autora.e (curso) \\
autora.e (curso) \\
autora.e (curso) \\
autora.e (curso)
}%

\newcommand{\gestao}{nome da gestao (\ano)}

\begin{document}

\documentclass{book}

\usepackage[T1]{fontenc}
\usepackage[portuguese]{babel}


\usepackage{manual_template}
% tem que decidir uma fonte legal
% https://tug.org/FontCatalogue/typewriterfonts.html

% para gerar texto lixo
\usepackage{blindtext}
\usepackage{lipsum}

\title{Manual de Ingressante}
\author{Centro Acadêmico da Computação}
\newcommand{\ano}{2024}
\date{\ano}
\newcommand{\shortauthor}{CACo}

\newcommand{\autoresCapa}{%
autor1 (EC021) \\
autor 2 (CC020)
}%

\newcommand{\autoresDiagramacao}{%
autora.e (curso) \\
autora.e (curso) \\
autora.e (curso)
}%

\newcommand{\autoresRevisao}{%
autora.e (curso) \\
autora.e (curso) \\
autora.e (curso) \\
autora.e (curso) \\
autora.e (curso) \\
autora.e (curso)
}%

\newcommand{\gestao}{nome da gestao (\ano)}

\begin{document}

\input{_adendos/main}

%\maketitle
%\tableofcontents

%\setcounter{page}{1}

\include{0_boas_vindas/main}

\include{1_infraestrutura_unicamp/main}

\include{2_burocracias_e_estudos/main}

\chapter{Convivendo na Unicam}
\lipsum[1-5]

\section{whatever}
\lipsum[1-5]


\chapter{outro capitulo}
\lipsum[1-5]

%\chapter{Convivendo na Unicamp\\ {\color{white}. } \hspace{5mm} {\normalsize e no mundo}}
\chapter{Convivendo na Unicamp}
\lipsum[1-5]

\setcounter{section}{17}
\section{Mais secao (numero 17)}

\lipsum[1-5]

\section{E outra secao}

\lipsum[1-5]

\section{mais outra secao}

\lipsum[1-5]

\section{Eu amo secao}
\setcounter{subsection}{3}
\subsection{Algumas formas de evitar e combater o \\ machismo no dia a dia da universidade}

\lipsum[1-5]

%\chapter{Vivendo em Barão \\ {\color{white}. } \hspace{5mm} {\normalsize Barão Geraldo ou BG}}
\chapter{Vivendo em Barão Geraldo}
\lipsum[1-5]

\chapter{Além da Graduação}
\lipsum[1-5]

\chapter{aba}
\section{aoba}
oi

terminal \faTerminal, linux \faLinux, sapo \faFrog

github \faGithub, linkedin \faLinkedin, gitlab \faGitlab, email \faEnvelope, telegram \faPaperPlane[regular] \faTelegram\ \faTelegramPlane, whats \faWhatsapp

map marker \faMapMarker*, restaurantes \faUtensils, cafe \faCoffee, ra \faIdCard[regular] \faAddressCard[regular], casa \faHome

\begin{list}{\faTerminal}{}  
\item item A bla bla bla
\item item B blablabla
\end{list}

\section{Testes de links}


\instagram{meuinstagram}

\telegram{meugrupotelegram}

\discord{meugrupodiscord}

%\pagebreak

\lipsum[1-5]

\begin{tags}
    \github{meuusergithub} \sep \gitlab{meuusergitlab}    
\end{tags}

\end{document}


%\maketitle
%\tableofcontents

%\setcounter{page}{1}

\documentclass{book}

\usepackage[T1]{fontenc}
\usepackage[portuguese]{babel}


\usepackage{manual_template}
% tem que decidir uma fonte legal
% https://tug.org/FontCatalogue/typewriterfonts.html

% para gerar texto lixo
\usepackage{blindtext}
\usepackage{lipsum}

\title{Manual de Ingressante}
\author{Centro Acadêmico da Computação}
\newcommand{\ano}{2024}
\date{\ano}
\newcommand{\shortauthor}{CACo}

\newcommand{\autoresCapa}{%
autor1 (EC021) \\
autor 2 (CC020)
}%

\newcommand{\autoresDiagramacao}{%
autora.e (curso) \\
autora.e (curso) \\
autora.e (curso)
}%

\newcommand{\autoresRevisao}{%
autora.e (curso) \\
autora.e (curso) \\
autora.e (curso) \\
autora.e (curso) \\
autora.e (curso) \\
autora.e (curso)
}%

\newcommand{\gestao}{nome da gestao (\ano)}

\begin{document}

\input{_adendos/main}

%\maketitle
%\tableofcontents

%\setcounter{page}{1}

\include{0_boas_vindas/main}

\include{1_infraestrutura_unicamp/main}

\include{2_burocracias_e_estudos/main}

\chapter{Convivendo na Unicam}
\lipsum[1-5]

\section{whatever}
\lipsum[1-5]


\chapter{outro capitulo}
\lipsum[1-5]

%\chapter{Convivendo na Unicamp\\ {\color{white}. } \hspace{5mm} {\normalsize e no mundo}}
\chapter{Convivendo na Unicamp}
\lipsum[1-5]

\setcounter{section}{17}
\section{Mais secao (numero 17)}

\lipsum[1-5]

\section{E outra secao}

\lipsum[1-5]

\section{mais outra secao}

\lipsum[1-5]

\section{Eu amo secao}
\setcounter{subsection}{3}
\subsection{Algumas formas de evitar e combater o \\ machismo no dia a dia da universidade}

\lipsum[1-5]

%\chapter{Vivendo em Barão \\ {\color{white}. } \hspace{5mm} {\normalsize Barão Geraldo ou BG}}
\chapter{Vivendo em Barão Geraldo}
\lipsum[1-5]

\chapter{Além da Graduação}
\lipsum[1-5]

\chapter{aba}
\section{aoba}
oi

terminal \faTerminal, linux \faLinux, sapo \faFrog

github \faGithub, linkedin \faLinkedin, gitlab \faGitlab, email \faEnvelope, telegram \faPaperPlane[regular] \faTelegram\ \faTelegramPlane, whats \faWhatsapp

map marker \faMapMarker*, restaurantes \faUtensils, cafe \faCoffee, ra \faIdCard[regular] \faAddressCard[regular], casa \faHome

\begin{list}{\faTerminal}{}  
\item item A bla bla bla
\item item B blablabla
\end{list}

\section{Testes de links}


\instagram{meuinstagram}

\telegram{meugrupotelegram}

\discord{meugrupodiscord}

%\pagebreak

\lipsum[1-5]

\begin{tags}
    \github{meuusergithub} \sep \gitlab{meuusergitlab}    
\end{tags}

\end{document}


\documentclass{book}

\usepackage[T1]{fontenc}
\usepackage[portuguese]{babel}


\usepackage{manual_template}
% tem que decidir uma fonte legal
% https://tug.org/FontCatalogue/typewriterfonts.html

% para gerar texto lixo
\usepackage{blindtext}
\usepackage{lipsum}

\title{Manual de Ingressante}
\author{Centro Acadêmico da Computação}
\newcommand{\ano}{2024}
\date{\ano}
\newcommand{\shortauthor}{CACo}

\newcommand{\autoresCapa}{%
autor1 (EC021) \\
autor 2 (CC020)
}%

\newcommand{\autoresDiagramacao}{%
autora.e (curso) \\
autora.e (curso) \\
autora.e (curso)
}%

\newcommand{\autoresRevisao}{%
autora.e (curso) \\
autora.e (curso) \\
autora.e (curso) \\
autora.e (curso) \\
autora.e (curso) \\
autora.e (curso)
}%

\newcommand{\gestao}{nome da gestao (\ano)}

\begin{document}

\input{_adendos/main}

%\maketitle
%\tableofcontents

%\setcounter{page}{1}

\include{0_boas_vindas/main}

\include{1_infraestrutura_unicamp/main}

\include{2_burocracias_e_estudos/main}

\chapter{Convivendo na Unicam}
\lipsum[1-5]

\section{whatever}
\lipsum[1-5]


\chapter{outro capitulo}
\lipsum[1-5]

%\chapter{Convivendo na Unicamp\\ {\color{white}. } \hspace{5mm} {\normalsize e no mundo}}
\chapter{Convivendo na Unicamp}
\lipsum[1-5]

\setcounter{section}{17}
\section{Mais secao (numero 17)}

\lipsum[1-5]

\section{E outra secao}

\lipsum[1-5]

\section{mais outra secao}

\lipsum[1-5]

\section{Eu amo secao}
\setcounter{subsection}{3}
\subsection{Algumas formas de evitar e combater o \\ machismo no dia a dia da universidade}

\lipsum[1-5]

%\chapter{Vivendo em Barão \\ {\color{white}. } \hspace{5mm} {\normalsize Barão Geraldo ou BG}}
\chapter{Vivendo em Barão Geraldo}
\lipsum[1-5]

\chapter{Além da Graduação}
\lipsum[1-5]

\chapter{aba}
\section{aoba}
oi

terminal \faTerminal, linux \faLinux, sapo \faFrog

github \faGithub, linkedin \faLinkedin, gitlab \faGitlab, email \faEnvelope, telegram \faPaperPlane[regular] \faTelegram\ \faTelegramPlane, whats \faWhatsapp

map marker \faMapMarker*, restaurantes \faUtensils, cafe \faCoffee, ra \faIdCard[regular] \faAddressCard[regular], casa \faHome

\begin{list}{\faTerminal}{}  
\item item A bla bla bla
\item item B blablabla
\end{list}

\section{Testes de links}


\instagram{meuinstagram}

\telegram{meugrupotelegram}

\discord{meugrupodiscord}

%\pagebreak

\lipsum[1-5]

\begin{tags}
    \github{meuusergithub} \sep \gitlab{meuusergitlab}    
\end{tags}

\end{document}


\documentclass{book}

\usepackage[T1]{fontenc}
\usepackage[portuguese]{babel}


\usepackage{manual_template}
% tem que decidir uma fonte legal
% https://tug.org/FontCatalogue/typewriterfonts.html

% para gerar texto lixo
\usepackage{blindtext}
\usepackage{lipsum}

\title{Manual de Ingressante}
\author{Centro Acadêmico da Computação}
\newcommand{\ano}{2024}
\date{\ano}
\newcommand{\shortauthor}{CACo}

\newcommand{\autoresCapa}{%
autor1 (EC021) \\
autor 2 (CC020)
}%

\newcommand{\autoresDiagramacao}{%
autora.e (curso) \\
autora.e (curso) \\
autora.e (curso)
}%

\newcommand{\autoresRevisao}{%
autora.e (curso) \\
autora.e (curso) \\
autora.e (curso) \\
autora.e (curso) \\
autora.e (curso) \\
autora.e (curso)
}%

\newcommand{\gestao}{nome da gestao (\ano)}

\begin{document}

\input{_adendos/main}

%\maketitle
%\tableofcontents

%\setcounter{page}{1}

\include{0_boas_vindas/main}

\include{1_infraestrutura_unicamp/main}

\include{2_burocracias_e_estudos/main}

\chapter{Convivendo na Unicam}
\lipsum[1-5]

\section{whatever}
\lipsum[1-5]


\chapter{outro capitulo}
\lipsum[1-5]

%\chapter{Convivendo na Unicamp\\ {\color{white}. } \hspace{5mm} {\normalsize e no mundo}}
\chapter{Convivendo na Unicamp}
\lipsum[1-5]

\setcounter{section}{17}
\section{Mais secao (numero 17)}

\lipsum[1-5]

\section{E outra secao}

\lipsum[1-5]

\section{mais outra secao}

\lipsum[1-5]

\section{Eu amo secao}
\setcounter{subsection}{3}
\subsection{Algumas formas de evitar e combater o \\ machismo no dia a dia da universidade}

\lipsum[1-5]

%\chapter{Vivendo em Barão \\ {\color{white}. } \hspace{5mm} {\normalsize Barão Geraldo ou BG}}
\chapter{Vivendo em Barão Geraldo}
\lipsum[1-5]

\chapter{Além da Graduação}
\lipsum[1-5]

\chapter{aba}
\section{aoba}
oi

terminal \faTerminal, linux \faLinux, sapo \faFrog

github \faGithub, linkedin \faLinkedin, gitlab \faGitlab, email \faEnvelope, telegram \faPaperPlane[regular] \faTelegram\ \faTelegramPlane, whats \faWhatsapp

map marker \faMapMarker*, restaurantes \faUtensils, cafe \faCoffee, ra \faIdCard[regular] \faAddressCard[regular], casa \faHome

\begin{list}{\faTerminal}{}  
\item item A bla bla bla
\item item B blablabla
\end{list}

\section{Testes de links}


\instagram{meuinstagram}

\telegram{meugrupotelegram}

\discord{meugrupodiscord}

%\pagebreak

\lipsum[1-5]

\begin{tags}
    \github{meuusergithub} \sep \gitlab{meuusergitlab}    
\end{tags}

\end{document}


\chapter{Convivendo na Unicam}
\lipsum[1-5]

\section{whatever}
\lipsum[1-5]


\chapter{outro capitulo}
\lipsum[1-5]

%\chapter{Convivendo na Unicamp\\ {\color{white}. } \hspace{5mm} {\normalsize e no mundo}}
\chapter{Convivendo na Unicamp}
\lipsum[1-5]

\setcounter{section}{17}
\section{Mais secao (numero 17)}

\lipsum[1-5]

\section{E outra secao}

\lipsum[1-5]

\section{mais outra secao}

\lipsum[1-5]

\section{Eu amo secao}
\setcounter{subsection}{3}
\subsection{Algumas formas de evitar e combater o \\ machismo no dia a dia da universidade}

\lipsum[1-5]

%\chapter{Vivendo em Barão \\ {\color{white}. } \hspace{5mm} {\normalsize Barão Geraldo ou BG}}
\chapter{Vivendo em Barão Geraldo}
\lipsum[1-5]

\chapter{Além da Graduação}
\lipsum[1-5]

\chapter{aba}
\section{aoba}
oi

terminal \faTerminal, linux \faLinux, sapo \faFrog

github \faGithub, linkedin \faLinkedin, gitlab \faGitlab, email \faEnvelope, telegram \faPaperPlane[regular] \faTelegram\ \faTelegramPlane, whats \faWhatsapp

map marker \faMapMarker*, restaurantes \faUtensils, cafe \faCoffee, ra \faIdCard[regular] \faAddressCard[regular], casa \faHome

\begin{list}{\faTerminal}{}  
\item item A bla bla bla
\item item B blablabla
\end{list}

\section{Testes de links}


\instagram{meuinstagram}

\telegram{meugrupotelegram}

\discord{meugrupodiscord}

%\pagebreak

\lipsum[1-5]

\begin{tags}
    \github{meuusergithub} \sep \gitlab{meuusergitlab}    
\end{tags}

\end{document}


\documentclass{book}

\usepackage[T1]{fontenc}
\usepackage[portuguese]{babel}


\usepackage{manual_template}
% tem que decidir uma fonte legal
% https://tug.org/FontCatalogue/typewriterfonts.html

% para gerar texto lixo
\usepackage{blindtext}
\usepackage{lipsum}

\title{Manual de Ingressante}
\author{Centro Acadêmico da Computação}
\newcommand{\ano}{2024}
\date{\ano}
\newcommand{\shortauthor}{CACo}

\newcommand{\autoresCapa}{%
autor1 (EC021) \\
autor 2 (CC020)
}%

\newcommand{\autoresDiagramacao}{%
autora.e (curso) \\
autora.e (curso) \\
autora.e (curso)
}%

\newcommand{\autoresRevisao}{%
autora.e (curso) \\
autora.e (curso) \\
autora.e (curso) \\
autora.e (curso) \\
autora.e (curso) \\
autora.e (curso)
}%

\newcommand{\gestao}{nome da gestao (\ano)}

\begin{document}

\documentclass{book}

\usepackage[T1]{fontenc}
\usepackage[portuguese]{babel}


\usepackage{manual_template}
% tem que decidir uma fonte legal
% https://tug.org/FontCatalogue/typewriterfonts.html

% para gerar texto lixo
\usepackage{blindtext}
\usepackage{lipsum}

\title{Manual de Ingressante}
\author{Centro Acadêmico da Computação}
\newcommand{\ano}{2024}
\date{\ano}
\newcommand{\shortauthor}{CACo}

\newcommand{\autoresCapa}{%
autor1 (EC021) \\
autor 2 (CC020)
}%

\newcommand{\autoresDiagramacao}{%
autora.e (curso) \\
autora.e (curso) \\
autora.e (curso)
}%

\newcommand{\autoresRevisao}{%
autora.e (curso) \\
autora.e (curso) \\
autora.e (curso) \\
autora.e (curso) \\
autora.e (curso) \\
autora.e (curso)
}%

\newcommand{\gestao}{nome da gestao (\ano)}

\begin{document}

\input{_adendos/main}

%\maketitle
%\tableofcontents

%\setcounter{page}{1}

\include{0_boas_vindas/main}

\include{1_infraestrutura_unicamp/main}

\include{2_burocracias_e_estudos/main}

\chapter{Convivendo na Unicam}
\lipsum[1-5]

\section{whatever}
\lipsum[1-5]


\chapter{outro capitulo}
\lipsum[1-5]

%\chapter{Convivendo na Unicamp\\ {\color{white}. } \hspace{5mm} {\normalsize e no mundo}}
\chapter{Convivendo na Unicamp}
\lipsum[1-5]

\setcounter{section}{17}
\section{Mais secao (numero 17)}

\lipsum[1-5]

\section{E outra secao}

\lipsum[1-5]

\section{mais outra secao}

\lipsum[1-5]

\section{Eu amo secao}
\setcounter{subsection}{3}
\subsection{Algumas formas de evitar e combater o \\ machismo no dia a dia da universidade}

\lipsum[1-5]

%\chapter{Vivendo em Barão \\ {\color{white}. } \hspace{5mm} {\normalsize Barão Geraldo ou BG}}
\chapter{Vivendo em Barão Geraldo}
\lipsum[1-5]

\chapter{Além da Graduação}
\lipsum[1-5]

\chapter{aba}
\section{aoba}
oi

terminal \faTerminal, linux \faLinux, sapo \faFrog

github \faGithub, linkedin \faLinkedin, gitlab \faGitlab, email \faEnvelope, telegram \faPaperPlane[regular] \faTelegram\ \faTelegramPlane, whats \faWhatsapp

map marker \faMapMarker*, restaurantes \faUtensils, cafe \faCoffee, ra \faIdCard[regular] \faAddressCard[regular], casa \faHome

\begin{list}{\faTerminal}{}  
\item item A bla bla bla
\item item B blablabla
\end{list}

\section{Testes de links}


\instagram{meuinstagram}

\telegram{meugrupotelegram}

\discord{meugrupodiscord}

%\pagebreak

\lipsum[1-5]

\begin{tags}
    \github{meuusergithub} \sep \gitlab{meuusergitlab}    
\end{tags}

\end{document}


%\maketitle
%\tableofcontents

%\setcounter{page}{1}

\documentclass{book}

\usepackage[T1]{fontenc}
\usepackage[portuguese]{babel}


\usepackage{manual_template}
% tem que decidir uma fonte legal
% https://tug.org/FontCatalogue/typewriterfonts.html

% para gerar texto lixo
\usepackage{blindtext}
\usepackage{lipsum}

\title{Manual de Ingressante}
\author{Centro Acadêmico da Computação}
\newcommand{\ano}{2024}
\date{\ano}
\newcommand{\shortauthor}{CACo}

\newcommand{\autoresCapa}{%
autor1 (EC021) \\
autor 2 (CC020)
}%

\newcommand{\autoresDiagramacao}{%
autora.e (curso) \\
autora.e (curso) \\
autora.e (curso)
}%

\newcommand{\autoresRevisao}{%
autora.e (curso) \\
autora.e (curso) \\
autora.e (curso) \\
autora.e (curso) \\
autora.e (curso) \\
autora.e (curso)
}%

\newcommand{\gestao}{nome da gestao (\ano)}

\begin{document}

\input{_adendos/main}

%\maketitle
%\tableofcontents

%\setcounter{page}{1}

\include{0_boas_vindas/main}

\include{1_infraestrutura_unicamp/main}

\include{2_burocracias_e_estudos/main}

\chapter{Convivendo na Unicam}
\lipsum[1-5]

\section{whatever}
\lipsum[1-5]


\chapter{outro capitulo}
\lipsum[1-5]

%\chapter{Convivendo na Unicamp\\ {\color{white}. } \hspace{5mm} {\normalsize e no mundo}}
\chapter{Convivendo na Unicamp}
\lipsum[1-5]

\setcounter{section}{17}
\section{Mais secao (numero 17)}

\lipsum[1-5]

\section{E outra secao}

\lipsum[1-5]

\section{mais outra secao}

\lipsum[1-5]

\section{Eu amo secao}
\setcounter{subsection}{3}
\subsection{Algumas formas de evitar e combater o \\ machismo no dia a dia da universidade}

\lipsum[1-5]

%\chapter{Vivendo em Barão \\ {\color{white}. } \hspace{5mm} {\normalsize Barão Geraldo ou BG}}
\chapter{Vivendo em Barão Geraldo}
\lipsum[1-5]

\chapter{Além da Graduação}
\lipsum[1-5]

\chapter{aba}
\section{aoba}
oi

terminal \faTerminal, linux \faLinux, sapo \faFrog

github \faGithub, linkedin \faLinkedin, gitlab \faGitlab, email \faEnvelope, telegram \faPaperPlane[regular] \faTelegram\ \faTelegramPlane, whats \faWhatsapp

map marker \faMapMarker*, restaurantes \faUtensils, cafe \faCoffee, ra \faIdCard[regular] \faAddressCard[regular], casa \faHome

\begin{list}{\faTerminal}{}  
\item item A bla bla bla
\item item B blablabla
\end{list}

\section{Testes de links}


\instagram{meuinstagram}

\telegram{meugrupotelegram}

\discord{meugrupodiscord}

%\pagebreak

\lipsum[1-5]

\begin{tags}
    \github{meuusergithub} \sep \gitlab{meuusergitlab}    
\end{tags}

\end{document}


\documentclass{book}

\usepackage[T1]{fontenc}
\usepackage[portuguese]{babel}


\usepackage{manual_template}
% tem que decidir uma fonte legal
% https://tug.org/FontCatalogue/typewriterfonts.html

% para gerar texto lixo
\usepackage{blindtext}
\usepackage{lipsum}

\title{Manual de Ingressante}
\author{Centro Acadêmico da Computação}
\newcommand{\ano}{2024}
\date{\ano}
\newcommand{\shortauthor}{CACo}

\newcommand{\autoresCapa}{%
autor1 (EC021) \\
autor 2 (CC020)
}%

\newcommand{\autoresDiagramacao}{%
autora.e (curso) \\
autora.e (curso) \\
autora.e (curso)
}%

\newcommand{\autoresRevisao}{%
autora.e (curso) \\
autora.e (curso) \\
autora.e (curso) \\
autora.e (curso) \\
autora.e (curso) \\
autora.e (curso)
}%

\newcommand{\gestao}{nome da gestao (\ano)}

\begin{document}

\input{_adendos/main}

%\maketitle
%\tableofcontents

%\setcounter{page}{1}

\include{0_boas_vindas/main}

\include{1_infraestrutura_unicamp/main}

\include{2_burocracias_e_estudos/main}

\chapter{Convivendo na Unicam}
\lipsum[1-5]

\section{whatever}
\lipsum[1-5]


\chapter{outro capitulo}
\lipsum[1-5]

%\chapter{Convivendo na Unicamp\\ {\color{white}. } \hspace{5mm} {\normalsize e no mundo}}
\chapter{Convivendo na Unicamp}
\lipsum[1-5]

\setcounter{section}{17}
\section{Mais secao (numero 17)}

\lipsum[1-5]

\section{E outra secao}

\lipsum[1-5]

\section{mais outra secao}

\lipsum[1-5]

\section{Eu amo secao}
\setcounter{subsection}{3}
\subsection{Algumas formas de evitar e combater o \\ machismo no dia a dia da universidade}

\lipsum[1-5]

%\chapter{Vivendo em Barão \\ {\color{white}. } \hspace{5mm} {\normalsize Barão Geraldo ou BG}}
\chapter{Vivendo em Barão Geraldo}
\lipsum[1-5]

\chapter{Além da Graduação}
\lipsum[1-5]

\chapter{aba}
\section{aoba}
oi

terminal \faTerminal, linux \faLinux, sapo \faFrog

github \faGithub, linkedin \faLinkedin, gitlab \faGitlab, email \faEnvelope, telegram \faPaperPlane[regular] \faTelegram\ \faTelegramPlane, whats \faWhatsapp

map marker \faMapMarker*, restaurantes \faUtensils, cafe \faCoffee, ra \faIdCard[regular] \faAddressCard[regular], casa \faHome

\begin{list}{\faTerminal}{}  
\item item A bla bla bla
\item item B blablabla
\end{list}

\section{Testes de links}


\instagram{meuinstagram}

\telegram{meugrupotelegram}

\discord{meugrupodiscord}

%\pagebreak

\lipsum[1-5]

\begin{tags}
    \github{meuusergithub} \sep \gitlab{meuusergitlab}    
\end{tags}

\end{document}


\documentclass{book}

\usepackage[T1]{fontenc}
\usepackage[portuguese]{babel}


\usepackage{manual_template}
% tem que decidir uma fonte legal
% https://tug.org/FontCatalogue/typewriterfonts.html

% para gerar texto lixo
\usepackage{blindtext}
\usepackage{lipsum}

\title{Manual de Ingressante}
\author{Centro Acadêmico da Computação}
\newcommand{\ano}{2024}
\date{\ano}
\newcommand{\shortauthor}{CACo}

\newcommand{\autoresCapa}{%
autor1 (EC021) \\
autor 2 (CC020)
}%

\newcommand{\autoresDiagramacao}{%
autora.e (curso) \\
autora.e (curso) \\
autora.e (curso)
}%

\newcommand{\autoresRevisao}{%
autora.e (curso) \\
autora.e (curso) \\
autora.e (curso) \\
autora.e (curso) \\
autora.e (curso) \\
autora.e (curso)
}%

\newcommand{\gestao}{nome da gestao (\ano)}

\begin{document}

\input{_adendos/main}

%\maketitle
%\tableofcontents

%\setcounter{page}{1}

\include{0_boas_vindas/main}

\include{1_infraestrutura_unicamp/main}

\include{2_burocracias_e_estudos/main}

\chapter{Convivendo na Unicam}
\lipsum[1-5]

\section{whatever}
\lipsum[1-5]


\chapter{outro capitulo}
\lipsum[1-5]

%\chapter{Convivendo na Unicamp\\ {\color{white}. } \hspace{5mm} {\normalsize e no mundo}}
\chapter{Convivendo na Unicamp}
\lipsum[1-5]

\setcounter{section}{17}
\section{Mais secao (numero 17)}

\lipsum[1-5]

\section{E outra secao}

\lipsum[1-5]

\section{mais outra secao}

\lipsum[1-5]

\section{Eu amo secao}
\setcounter{subsection}{3}
\subsection{Algumas formas de evitar e combater o \\ machismo no dia a dia da universidade}

\lipsum[1-5]

%\chapter{Vivendo em Barão \\ {\color{white}. } \hspace{5mm} {\normalsize Barão Geraldo ou BG}}
\chapter{Vivendo em Barão Geraldo}
\lipsum[1-5]

\chapter{Além da Graduação}
\lipsum[1-5]

\chapter{aba}
\section{aoba}
oi

terminal \faTerminal, linux \faLinux, sapo \faFrog

github \faGithub, linkedin \faLinkedin, gitlab \faGitlab, email \faEnvelope, telegram \faPaperPlane[regular] \faTelegram\ \faTelegramPlane, whats \faWhatsapp

map marker \faMapMarker*, restaurantes \faUtensils, cafe \faCoffee, ra \faIdCard[regular] \faAddressCard[regular], casa \faHome

\begin{list}{\faTerminal}{}  
\item item A bla bla bla
\item item B blablabla
\end{list}

\section{Testes de links}


\instagram{meuinstagram}

\telegram{meugrupotelegram}

\discord{meugrupodiscord}

%\pagebreak

\lipsum[1-5]

\begin{tags}
    \github{meuusergithub} \sep \gitlab{meuusergitlab}    
\end{tags}

\end{document}


\chapter{Convivendo na Unicam}
\lipsum[1-5]

\section{whatever}
\lipsum[1-5]


\chapter{outro capitulo}
\lipsum[1-5]

%\chapter{Convivendo na Unicamp\\ {\color{white}. } \hspace{5mm} {\normalsize e no mundo}}
\chapter{Convivendo na Unicamp}
\lipsum[1-5]

\setcounter{section}{17}
\section{Mais secao (numero 17)}

\lipsum[1-5]

\section{E outra secao}

\lipsum[1-5]

\section{mais outra secao}

\lipsum[1-5]

\section{Eu amo secao}
\setcounter{subsection}{3}
\subsection{Algumas formas de evitar e combater o \\ machismo no dia a dia da universidade}

\lipsum[1-5]

%\chapter{Vivendo em Barão \\ {\color{white}. } \hspace{5mm} {\normalsize Barão Geraldo ou BG}}
\chapter{Vivendo em Barão Geraldo}
\lipsum[1-5]

\chapter{Além da Graduação}
\lipsum[1-5]

\chapter{aba}
\section{aoba}
oi

terminal \faTerminal, linux \faLinux, sapo \faFrog

github \faGithub, linkedin \faLinkedin, gitlab \faGitlab, email \faEnvelope, telegram \faPaperPlane[regular] \faTelegram\ \faTelegramPlane, whats \faWhatsapp

map marker \faMapMarker*, restaurantes \faUtensils, cafe \faCoffee, ra \faIdCard[regular] \faAddressCard[regular], casa \faHome

\begin{list}{\faTerminal}{}  
\item item A bla bla bla
\item item B blablabla
\end{list}

\section{Testes de links}


\instagram{meuinstagram}

\telegram{meugrupotelegram}

\discord{meugrupodiscord}

%\pagebreak

\lipsum[1-5]

\begin{tags}
    \github{meuusergithub} \sep \gitlab{meuusergitlab}    
\end{tags}

\end{document}


\chapter{Convivendo na Unicam}
\lipsum[1-5]

\section{whatever}
\lipsum[1-5]


\chapter{outro capitulo}
\lipsum[1-5]

%\chapter{Convivendo na Unicamp\\ {\color{white}. } \hspace{5mm} {\normalsize e no mundo}}
\chapter{Convivendo na Unicamp}
\lipsum[1-5]

\setcounter{section}{17}
\section{Mais secao (numero 17)}

\lipsum[1-5]

\section{E outra secao}

\lipsum[1-5]

\section{mais outra secao}

\lipsum[1-5]

\section{Eu amo secao}
\setcounter{subsection}{3}
\subsection{Algumas formas de evitar e combater o \\ machismo no dia a dia da universidade}

\lipsum[1-5]

%\chapter{Vivendo em Barão \\ {\color{white}. } \hspace{5mm} {\normalsize Barão Geraldo ou BG}}
\chapter{Vivendo em Barão Geraldo}
\lipsum[1-5]

\chapter{Além da Graduação}
\lipsum[1-5]

\chapter{aba}
\section{aoba}
oi

terminal \faTerminal, linux \faLinux, sapo \faFrog

github \faGithub, linkedin \faLinkedin, gitlab \faGitlab, email \faEnvelope, telegram \faPaperPlane[regular] \faTelegram\ \faTelegramPlane, whats \faWhatsapp

map marker \faMapMarker*, restaurantes \faUtensils, cafe \faCoffee, ra \faIdCard[regular] \faAddressCard[regular], casa \faHome

\begin{list}{\faTerminal}{}  
\item item A bla bla bla
\item item B blablabla
\end{list}

\section{Testes de links}


\instagram{meuinstagram}

\telegram{meugrupotelegram}

\discord{meugrupodiscord}

%\pagebreak

\lipsum[1-5]

\begin{tags}
    \github{meuusergithub} \sep \gitlab{meuusergitlab}    
\end{tags}

\end{document}


\documentclass{book}

\usepackage[T1]{fontenc}
\usepackage[portuguese]{babel}


\usepackage{manual_template}
% tem que decidir uma fonte legal
% https://tug.org/FontCatalogue/typewriterfonts.html

% para gerar texto lixo
\usepackage{blindtext}
\usepackage{lipsum}

\title{Manual de Ingressante}
\author{Centro Acadêmico da Computação}
\newcommand{\ano}{2024}
\date{\ano}
\newcommand{\shortauthor}{CACo}

\newcommand{\autoresCapa}{%
autor1 (EC021) \\
autor 2 (CC020)
}%

\newcommand{\autoresDiagramacao}{%
autora.e (curso) \\
autora.e (curso) \\
autora.e (curso)
}%

\newcommand{\autoresRevisao}{%
autora.e (curso) \\
autora.e (curso) \\
autora.e (curso) \\
autora.e (curso) \\
autora.e (curso) \\
autora.e (curso)
}%

\newcommand{\gestao}{nome da gestao (\ano)}

\begin{document}

\documentclass{book}

\usepackage[T1]{fontenc}
\usepackage[portuguese]{babel}


\usepackage{manual_template}
% tem que decidir uma fonte legal
% https://tug.org/FontCatalogue/typewriterfonts.html

% para gerar texto lixo
\usepackage{blindtext}
\usepackage{lipsum}

\title{Manual de Ingressante}
\author{Centro Acadêmico da Computação}
\newcommand{\ano}{2024}
\date{\ano}
\newcommand{\shortauthor}{CACo}

\newcommand{\autoresCapa}{%
autor1 (EC021) \\
autor 2 (CC020)
}%

\newcommand{\autoresDiagramacao}{%
autora.e (curso) \\
autora.e (curso) \\
autora.e (curso)
}%

\newcommand{\autoresRevisao}{%
autora.e (curso) \\
autora.e (curso) \\
autora.e (curso) \\
autora.e (curso) \\
autora.e (curso) \\
autora.e (curso)
}%

\newcommand{\gestao}{nome da gestao (\ano)}

\begin{document}

\documentclass{book}

\usepackage[T1]{fontenc}
\usepackage[portuguese]{babel}


\usepackage{manual_template}
% tem que decidir uma fonte legal
% https://tug.org/FontCatalogue/typewriterfonts.html

% para gerar texto lixo
\usepackage{blindtext}
\usepackage{lipsum}

\title{Manual de Ingressante}
\author{Centro Acadêmico da Computação}
\newcommand{\ano}{2024}
\date{\ano}
\newcommand{\shortauthor}{CACo}

\newcommand{\autoresCapa}{%
autor1 (EC021) \\
autor 2 (CC020)
}%

\newcommand{\autoresDiagramacao}{%
autora.e (curso) \\
autora.e (curso) \\
autora.e (curso)
}%

\newcommand{\autoresRevisao}{%
autora.e (curso) \\
autora.e (curso) \\
autora.e (curso) \\
autora.e (curso) \\
autora.e (curso) \\
autora.e (curso)
}%

\newcommand{\gestao}{nome da gestao (\ano)}

\begin{document}

\input{_adendos/main}

%\maketitle
%\tableofcontents

%\setcounter{page}{1}

\include{0_boas_vindas/main}

\include{1_infraestrutura_unicamp/main}

\include{2_burocracias_e_estudos/main}

\chapter{Convivendo na Unicam}
\lipsum[1-5]

\section{whatever}
\lipsum[1-5]


\chapter{outro capitulo}
\lipsum[1-5]

%\chapter{Convivendo na Unicamp\\ {\color{white}. } \hspace{5mm} {\normalsize e no mundo}}
\chapter{Convivendo na Unicamp}
\lipsum[1-5]

\setcounter{section}{17}
\section{Mais secao (numero 17)}

\lipsum[1-5]

\section{E outra secao}

\lipsum[1-5]

\section{mais outra secao}

\lipsum[1-5]

\section{Eu amo secao}
\setcounter{subsection}{3}
\subsection{Algumas formas de evitar e combater o \\ machismo no dia a dia da universidade}

\lipsum[1-5]

%\chapter{Vivendo em Barão \\ {\color{white}. } \hspace{5mm} {\normalsize Barão Geraldo ou BG}}
\chapter{Vivendo em Barão Geraldo}
\lipsum[1-5]

\chapter{Além da Graduação}
\lipsum[1-5]

\chapter{aba}
\section{aoba}
oi

terminal \faTerminal, linux \faLinux, sapo \faFrog

github \faGithub, linkedin \faLinkedin, gitlab \faGitlab, email \faEnvelope, telegram \faPaperPlane[regular] \faTelegram\ \faTelegramPlane, whats \faWhatsapp

map marker \faMapMarker*, restaurantes \faUtensils, cafe \faCoffee, ra \faIdCard[regular] \faAddressCard[regular], casa \faHome

\begin{list}{\faTerminal}{}  
\item item A bla bla bla
\item item B blablabla
\end{list}

\section{Testes de links}


\instagram{meuinstagram}

\telegram{meugrupotelegram}

\discord{meugrupodiscord}

%\pagebreak

\lipsum[1-5]

\begin{tags}
    \github{meuusergithub} \sep \gitlab{meuusergitlab}    
\end{tags}

\end{document}


%\maketitle
%\tableofcontents

%\setcounter{page}{1}

\documentclass{book}

\usepackage[T1]{fontenc}
\usepackage[portuguese]{babel}


\usepackage{manual_template}
% tem que decidir uma fonte legal
% https://tug.org/FontCatalogue/typewriterfonts.html

% para gerar texto lixo
\usepackage{blindtext}
\usepackage{lipsum}

\title{Manual de Ingressante}
\author{Centro Acadêmico da Computação}
\newcommand{\ano}{2024}
\date{\ano}
\newcommand{\shortauthor}{CACo}

\newcommand{\autoresCapa}{%
autor1 (EC021) \\
autor 2 (CC020)
}%

\newcommand{\autoresDiagramacao}{%
autora.e (curso) \\
autora.e (curso) \\
autora.e (curso)
}%

\newcommand{\autoresRevisao}{%
autora.e (curso) \\
autora.e (curso) \\
autora.e (curso) \\
autora.e (curso) \\
autora.e (curso) \\
autora.e (curso)
}%

\newcommand{\gestao}{nome da gestao (\ano)}

\begin{document}

\input{_adendos/main}

%\maketitle
%\tableofcontents

%\setcounter{page}{1}

\include{0_boas_vindas/main}

\include{1_infraestrutura_unicamp/main}

\include{2_burocracias_e_estudos/main}

\chapter{Convivendo na Unicam}
\lipsum[1-5]

\section{whatever}
\lipsum[1-5]


\chapter{outro capitulo}
\lipsum[1-5]

%\chapter{Convivendo na Unicamp\\ {\color{white}. } \hspace{5mm} {\normalsize e no mundo}}
\chapter{Convivendo na Unicamp}
\lipsum[1-5]

\setcounter{section}{17}
\section{Mais secao (numero 17)}

\lipsum[1-5]

\section{E outra secao}

\lipsum[1-5]

\section{mais outra secao}

\lipsum[1-5]

\section{Eu amo secao}
\setcounter{subsection}{3}
\subsection{Algumas formas de evitar e combater o \\ machismo no dia a dia da universidade}

\lipsum[1-5]

%\chapter{Vivendo em Barão \\ {\color{white}. } \hspace{5mm} {\normalsize Barão Geraldo ou BG}}
\chapter{Vivendo em Barão Geraldo}
\lipsum[1-5]

\chapter{Além da Graduação}
\lipsum[1-5]

\chapter{aba}
\section{aoba}
oi

terminal \faTerminal, linux \faLinux, sapo \faFrog

github \faGithub, linkedin \faLinkedin, gitlab \faGitlab, email \faEnvelope, telegram \faPaperPlane[regular] \faTelegram\ \faTelegramPlane, whats \faWhatsapp

map marker \faMapMarker*, restaurantes \faUtensils, cafe \faCoffee, ra \faIdCard[regular] \faAddressCard[regular], casa \faHome

\begin{list}{\faTerminal}{}  
\item item A bla bla bla
\item item B blablabla
\end{list}

\section{Testes de links}


\instagram{meuinstagram}

\telegram{meugrupotelegram}

\discord{meugrupodiscord}

%\pagebreak

\lipsum[1-5]

\begin{tags}
    \github{meuusergithub} \sep \gitlab{meuusergitlab}    
\end{tags}

\end{document}


\documentclass{book}

\usepackage[T1]{fontenc}
\usepackage[portuguese]{babel}


\usepackage{manual_template}
% tem que decidir uma fonte legal
% https://tug.org/FontCatalogue/typewriterfonts.html

% para gerar texto lixo
\usepackage{blindtext}
\usepackage{lipsum}

\title{Manual de Ingressante}
\author{Centro Acadêmico da Computação}
\newcommand{\ano}{2024}
\date{\ano}
\newcommand{\shortauthor}{CACo}

\newcommand{\autoresCapa}{%
autor1 (EC021) \\
autor 2 (CC020)
}%

\newcommand{\autoresDiagramacao}{%
autora.e (curso) \\
autora.e (curso) \\
autora.e (curso)
}%

\newcommand{\autoresRevisao}{%
autora.e (curso) \\
autora.e (curso) \\
autora.e (curso) \\
autora.e (curso) \\
autora.e (curso) \\
autora.e (curso)
}%

\newcommand{\gestao}{nome da gestao (\ano)}

\begin{document}

\input{_adendos/main}

%\maketitle
%\tableofcontents

%\setcounter{page}{1}

\include{0_boas_vindas/main}

\include{1_infraestrutura_unicamp/main}

\include{2_burocracias_e_estudos/main}

\chapter{Convivendo na Unicam}
\lipsum[1-5]

\section{whatever}
\lipsum[1-5]


\chapter{outro capitulo}
\lipsum[1-5]

%\chapter{Convivendo na Unicamp\\ {\color{white}. } \hspace{5mm} {\normalsize e no mundo}}
\chapter{Convivendo na Unicamp}
\lipsum[1-5]

\setcounter{section}{17}
\section{Mais secao (numero 17)}

\lipsum[1-5]

\section{E outra secao}

\lipsum[1-5]

\section{mais outra secao}

\lipsum[1-5]

\section{Eu amo secao}
\setcounter{subsection}{3}
\subsection{Algumas formas de evitar e combater o \\ machismo no dia a dia da universidade}

\lipsum[1-5]

%\chapter{Vivendo em Barão \\ {\color{white}. } \hspace{5mm} {\normalsize Barão Geraldo ou BG}}
\chapter{Vivendo em Barão Geraldo}
\lipsum[1-5]

\chapter{Além da Graduação}
\lipsum[1-5]

\chapter{aba}
\section{aoba}
oi

terminal \faTerminal, linux \faLinux, sapo \faFrog

github \faGithub, linkedin \faLinkedin, gitlab \faGitlab, email \faEnvelope, telegram \faPaperPlane[regular] \faTelegram\ \faTelegramPlane, whats \faWhatsapp

map marker \faMapMarker*, restaurantes \faUtensils, cafe \faCoffee, ra \faIdCard[regular] \faAddressCard[regular], casa \faHome

\begin{list}{\faTerminal}{}  
\item item A bla bla bla
\item item B blablabla
\end{list}

\section{Testes de links}


\instagram{meuinstagram}

\telegram{meugrupotelegram}

\discord{meugrupodiscord}

%\pagebreak

\lipsum[1-5]

\begin{tags}
    \github{meuusergithub} \sep \gitlab{meuusergitlab}    
\end{tags}

\end{document}


\documentclass{book}

\usepackage[T1]{fontenc}
\usepackage[portuguese]{babel}


\usepackage{manual_template}
% tem que decidir uma fonte legal
% https://tug.org/FontCatalogue/typewriterfonts.html

% para gerar texto lixo
\usepackage{blindtext}
\usepackage{lipsum}

\title{Manual de Ingressante}
\author{Centro Acadêmico da Computação}
\newcommand{\ano}{2024}
\date{\ano}
\newcommand{\shortauthor}{CACo}

\newcommand{\autoresCapa}{%
autor1 (EC021) \\
autor 2 (CC020)
}%

\newcommand{\autoresDiagramacao}{%
autora.e (curso) \\
autora.e (curso) \\
autora.e (curso)
}%

\newcommand{\autoresRevisao}{%
autora.e (curso) \\
autora.e (curso) \\
autora.e (curso) \\
autora.e (curso) \\
autora.e (curso) \\
autora.e (curso)
}%

\newcommand{\gestao}{nome da gestao (\ano)}

\begin{document}

\input{_adendos/main}

%\maketitle
%\tableofcontents

%\setcounter{page}{1}

\include{0_boas_vindas/main}

\include{1_infraestrutura_unicamp/main}

\include{2_burocracias_e_estudos/main}

\chapter{Convivendo na Unicam}
\lipsum[1-5]

\section{whatever}
\lipsum[1-5]


\chapter{outro capitulo}
\lipsum[1-5]

%\chapter{Convivendo na Unicamp\\ {\color{white}. } \hspace{5mm} {\normalsize e no mundo}}
\chapter{Convivendo na Unicamp}
\lipsum[1-5]

\setcounter{section}{17}
\section{Mais secao (numero 17)}

\lipsum[1-5]

\section{E outra secao}

\lipsum[1-5]

\section{mais outra secao}

\lipsum[1-5]

\section{Eu amo secao}
\setcounter{subsection}{3}
\subsection{Algumas formas de evitar e combater o \\ machismo no dia a dia da universidade}

\lipsum[1-5]

%\chapter{Vivendo em Barão \\ {\color{white}. } \hspace{5mm} {\normalsize Barão Geraldo ou BG}}
\chapter{Vivendo em Barão Geraldo}
\lipsum[1-5]

\chapter{Além da Graduação}
\lipsum[1-5]

\chapter{aba}
\section{aoba}
oi

terminal \faTerminal, linux \faLinux, sapo \faFrog

github \faGithub, linkedin \faLinkedin, gitlab \faGitlab, email \faEnvelope, telegram \faPaperPlane[regular] \faTelegram\ \faTelegramPlane, whats \faWhatsapp

map marker \faMapMarker*, restaurantes \faUtensils, cafe \faCoffee, ra \faIdCard[regular] \faAddressCard[regular], casa \faHome

\begin{list}{\faTerminal}{}  
\item item A bla bla bla
\item item B blablabla
\end{list}

\section{Testes de links}


\instagram{meuinstagram}

\telegram{meugrupotelegram}

\discord{meugrupodiscord}

%\pagebreak

\lipsum[1-5]

\begin{tags}
    \github{meuusergithub} \sep \gitlab{meuusergitlab}    
\end{tags}

\end{document}


\chapter{Convivendo na Unicam}
\lipsum[1-5]

\section{whatever}
\lipsum[1-5]


\chapter{outro capitulo}
\lipsum[1-5]

%\chapter{Convivendo na Unicamp\\ {\color{white}. } \hspace{5mm} {\normalsize e no mundo}}
\chapter{Convivendo na Unicamp}
\lipsum[1-5]

\setcounter{section}{17}
\section{Mais secao (numero 17)}

\lipsum[1-5]

\section{E outra secao}

\lipsum[1-5]

\section{mais outra secao}

\lipsum[1-5]

\section{Eu amo secao}
\setcounter{subsection}{3}
\subsection{Algumas formas de evitar e combater o \\ machismo no dia a dia da universidade}

\lipsum[1-5]

%\chapter{Vivendo em Barão \\ {\color{white}. } \hspace{5mm} {\normalsize Barão Geraldo ou BG}}
\chapter{Vivendo em Barão Geraldo}
\lipsum[1-5]

\chapter{Além da Graduação}
\lipsum[1-5]

\chapter{aba}
\section{aoba}
oi

terminal \faTerminal, linux \faLinux, sapo \faFrog

github \faGithub, linkedin \faLinkedin, gitlab \faGitlab, email \faEnvelope, telegram \faPaperPlane[regular] \faTelegram\ \faTelegramPlane, whats \faWhatsapp

map marker \faMapMarker*, restaurantes \faUtensils, cafe \faCoffee, ra \faIdCard[regular] \faAddressCard[regular], casa \faHome

\begin{list}{\faTerminal}{}  
\item item A bla bla bla
\item item B blablabla
\end{list}

\section{Testes de links}


\instagram{meuinstagram}

\telegram{meugrupotelegram}

\discord{meugrupodiscord}

%\pagebreak

\lipsum[1-5]

\begin{tags}
    \github{meuusergithub} \sep \gitlab{meuusergitlab}    
\end{tags}

\end{document}


%\maketitle
%\tableofcontents

%\setcounter{page}{1}

\documentclass{book}

\usepackage[T1]{fontenc}
\usepackage[portuguese]{babel}


\usepackage{manual_template}
% tem que decidir uma fonte legal
% https://tug.org/FontCatalogue/typewriterfonts.html

% para gerar texto lixo
\usepackage{blindtext}
\usepackage{lipsum}

\title{Manual de Ingressante}
\author{Centro Acadêmico da Computação}
\newcommand{\ano}{2024}
\date{\ano}
\newcommand{\shortauthor}{CACo}

\newcommand{\autoresCapa}{%
autor1 (EC021) \\
autor 2 (CC020)
}%

\newcommand{\autoresDiagramacao}{%
autora.e (curso) \\
autora.e (curso) \\
autora.e (curso)
}%

\newcommand{\autoresRevisao}{%
autora.e (curso) \\
autora.e (curso) \\
autora.e (curso) \\
autora.e (curso) \\
autora.e (curso) \\
autora.e (curso)
}%

\newcommand{\gestao}{nome da gestao (\ano)}

\begin{document}

\documentclass{book}

\usepackage[T1]{fontenc}
\usepackage[portuguese]{babel}


\usepackage{manual_template}
% tem que decidir uma fonte legal
% https://tug.org/FontCatalogue/typewriterfonts.html

% para gerar texto lixo
\usepackage{blindtext}
\usepackage{lipsum}

\title{Manual de Ingressante}
\author{Centro Acadêmico da Computação}
\newcommand{\ano}{2024}
\date{\ano}
\newcommand{\shortauthor}{CACo}

\newcommand{\autoresCapa}{%
autor1 (EC021) \\
autor 2 (CC020)
}%

\newcommand{\autoresDiagramacao}{%
autora.e (curso) \\
autora.e (curso) \\
autora.e (curso)
}%

\newcommand{\autoresRevisao}{%
autora.e (curso) \\
autora.e (curso) \\
autora.e (curso) \\
autora.e (curso) \\
autora.e (curso) \\
autora.e (curso)
}%

\newcommand{\gestao}{nome da gestao (\ano)}

\begin{document}

\input{_adendos/main}

%\maketitle
%\tableofcontents

%\setcounter{page}{1}

\include{0_boas_vindas/main}

\include{1_infraestrutura_unicamp/main}

\include{2_burocracias_e_estudos/main}

\chapter{Convivendo na Unicam}
\lipsum[1-5]

\section{whatever}
\lipsum[1-5]


\chapter{outro capitulo}
\lipsum[1-5]

%\chapter{Convivendo na Unicamp\\ {\color{white}. } \hspace{5mm} {\normalsize e no mundo}}
\chapter{Convivendo na Unicamp}
\lipsum[1-5]

\setcounter{section}{17}
\section{Mais secao (numero 17)}

\lipsum[1-5]

\section{E outra secao}

\lipsum[1-5]

\section{mais outra secao}

\lipsum[1-5]

\section{Eu amo secao}
\setcounter{subsection}{3}
\subsection{Algumas formas de evitar e combater o \\ machismo no dia a dia da universidade}

\lipsum[1-5]

%\chapter{Vivendo em Barão \\ {\color{white}. } \hspace{5mm} {\normalsize Barão Geraldo ou BG}}
\chapter{Vivendo em Barão Geraldo}
\lipsum[1-5]

\chapter{Além da Graduação}
\lipsum[1-5]

\chapter{aba}
\section{aoba}
oi

terminal \faTerminal, linux \faLinux, sapo \faFrog

github \faGithub, linkedin \faLinkedin, gitlab \faGitlab, email \faEnvelope, telegram \faPaperPlane[regular] \faTelegram\ \faTelegramPlane, whats \faWhatsapp

map marker \faMapMarker*, restaurantes \faUtensils, cafe \faCoffee, ra \faIdCard[regular] \faAddressCard[regular], casa \faHome

\begin{list}{\faTerminal}{}  
\item item A bla bla bla
\item item B blablabla
\end{list}

\section{Testes de links}


\instagram{meuinstagram}

\telegram{meugrupotelegram}

\discord{meugrupodiscord}

%\pagebreak

\lipsum[1-5]

\begin{tags}
    \github{meuusergithub} \sep \gitlab{meuusergitlab}    
\end{tags}

\end{document}


%\maketitle
%\tableofcontents

%\setcounter{page}{1}

\documentclass{book}

\usepackage[T1]{fontenc}
\usepackage[portuguese]{babel}


\usepackage{manual_template}
% tem que decidir uma fonte legal
% https://tug.org/FontCatalogue/typewriterfonts.html

% para gerar texto lixo
\usepackage{blindtext}
\usepackage{lipsum}

\title{Manual de Ingressante}
\author{Centro Acadêmico da Computação}
\newcommand{\ano}{2024}
\date{\ano}
\newcommand{\shortauthor}{CACo}

\newcommand{\autoresCapa}{%
autor1 (EC021) \\
autor 2 (CC020)
}%

\newcommand{\autoresDiagramacao}{%
autora.e (curso) \\
autora.e (curso) \\
autora.e (curso)
}%

\newcommand{\autoresRevisao}{%
autora.e (curso) \\
autora.e (curso) \\
autora.e (curso) \\
autora.e (curso) \\
autora.e (curso) \\
autora.e (curso)
}%

\newcommand{\gestao}{nome da gestao (\ano)}

\begin{document}

\input{_adendos/main}

%\maketitle
%\tableofcontents

%\setcounter{page}{1}

\include{0_boas_vindas/main}

\include{1_infraestrutura_unicamp/main}

\include{2_burocracias_e_estudos/main}

\chapter{Convivendo na Unicam}
\lipsum[1-5]

\section{whatever}
\lipsum[1-5]


\chapter{outro capitulo}
\lipsum[1-5]

%\chapter{Convivendo na Unicamp\\ {\color{white}. } \hspace{5mm} {\normalsize e no mundo}}
\chapter{Convivendo na Unicamp}
\lipsum[1-5]

\setcounter{section}{17}
\section{Mais secao (numero 17)}

\lipsum[1-5]

\section{E outra secao}

\lipsum[1-5]

\section{mais outra secao}

\lipsum[1-5]

\section{Eu amo secao}
\setcounter{subsection}{3}
\subsection{Algumas formas de evitar e combater o \\ machismo no dia a dia da universidade}

\lipsum[1-5]

%\chapter{Vivendo em Barão \\ {\color{white}. } \hspace{5mm} {\normalsize Barão Geraldo ou BG}}
\chapter{Vivendo em Barão Geraldo}
\lipsum[1-5]

\chapter{Além da Graduação}
\lipsum[1-5]

\chapter{aba}
\section{aoba}
oi

terminal \faTerminal, linux \faLinux, sapo \faFrog

github \faGithub, linkedin \faLinkedin, gitlab \faGitlab, email \faEnvelope, telegram \faPaperPlane[regular] \faTelegram\ \faTelegramPlane, whats \faWhatsapp

map marker \faMapMarker*, restaurantes \faUtensils, cafe \faCoffee, ra \faIdCard[regular] \faAddressCard[regular], casa \faHome

\begin{list}{\faTerminal}{}  
\item item A bla bla bla
\item item B blablabla
\end{list}

\section{Testes de links}


\instagram{meuinstagram}

\telegram{meugrupotelegram}

\discord{meugrupodiscord}

%\pagebreak

\lipsum[1-5]

\begin{tags}
    \github{meuusergithub} \sep \gitlab{meuusergitlab}    
\end{tags}

\end{document}


\documentclass{book}

\usepackage[T1]{fontenc}
\usepackage[portuguese]{babel}


\usepackage{manual_template}
% tem que decidir uma fonte legal
% https://tug.org/FontCatalogue/typewriterfonts.html

% para gerar texto lixo
\usepackage{blindtext}
\usepackage{lipsum}

\title{Manual de Ingressante}
\author{Centro Acadêmico da Computação}
\newcommand{\ano}{2024}
\date{\ano}
\newcommand{\shortauthor}{CACo}

\newcommand{\autoresCapa}{%
autor1 (EC021) \\
autor 2 (CC020)
}%

\newcommand{\autoresDiagramacao}{%
autora.e (curso) \\
autora.e (curso) \\
autora.e (curso)
}%

\newcommand{\autoresRevisao}{%
autora.e (curso) \\
autora.e (curso) \\
autora.e (curso) \\
autora.e (curso) \\
autora.e (curso) \\
autora.e (curso)
}%

\newcommand{\gestao}{nome da gestao (\ano)}

\begin{document}

\input{_adendos/main}

%\maketitle
%\tableofcontents

%\setcounter{page}{1}

\include{0_boas_vindas/main}

\include{1_infraestrutura_unicamp/main}

\include{2_burocracias_e_estudos/main}

\chapter{Convivendo na Unicam}
\lipsum[1-5]

\section{whatever}
\lipsum[1-5]


\chapter{outro capitulo}
\lipsum[1-5]

%\chapter{Convivendo na Unicamp\\ {\color{white}. } \hspace{5mm} {\normalsize e no mundo}}
\chapter{Convivendo na Unicamp}
\lipsum[1-5]

\setcounter{section}{17}
\section{Mais secao (numero 17)}

\lipsum[1-5]

\section{E outra secao}

\lipsum[1-5]

\section{mais outra secao}

\lipsum[1-5]

\section{Eu amo secao}
\setcounter{subsection}{3}
\subsection{Algumas formas de evitar e combater o \\ machismo no dia a dia da universidade}

\lipsum[1-5]

%\chapter{Vivendo em Barão \\ {\color{white}. } \hspace{5mm} {\normalsize Barão Geraldo ou BG}}
\chapter{Vivendo em Barão Geraldo}
\lipsum[1-5]

\chapter{Além da Graduação}
\lipsum[1-5]

\chapter{aba}
\section{aoba}
oi

terminal \faTerminal, linux \faLinux, sapo \faFrog

github \faGithub, linkedin \faLinkedin, gitlab \faGitlab, email \faEnvelope, telegram \faPaperPlane[regular] \faTelegram\ \faTelegramPlane, whats \faWhatsapp

map marker \faMapMarker*, restaurantes \faUtensils, cafe \faCoffee, ra \faIdCard[regular] \faAddressCard[regular], casa \faHome

\begin{list}{\faTerminal}{}  
\item item A bla bla bla
\item item B blablabla
\end{list}

\section{Testes de links}


\instagram{meuinstagram}

\telegram{meugrupotelegram}

\discord{meugrupodiscord}

%\pagebreak

\lipsum[1-5]

\begin{tags}
    \github{meuusergithub} \sep \gitlab{meuusergitlab}    
\end{tags}

\end{document}


\documentclass{book}

\usepackage[T1]{fontenc}
\usepackage[portuguese]{babel}


\usepackage{manual_template}
% tem que decidir uma fonte legal
% https://tug.org/FontCatalogue/typewriterfonts.html

% para gerar texto lixo
\usepackage{blindtext}
\usepackage{lipsum}

\title{Manual de Ingressante}
\author{Centro Acadêmico da Computação}
\newcommand{\ano}{2024}
\date{\ano}
\newcommand{\shortauthor}{CACo}

\newcommand{\autoresCapa}{%
autor1 (EC021) \\
autor 2 (CC020)
}%

\newcommand{\autoresDiagramacao}{%
autora.e (curso) \\
autora.e (curso) \\
autora.e (curso)
}%

\newcommand{\autoresRevisao}{%
autora.e (curso) \\
autora.e (curso) \\
autora.e (curso) \\
autora.e (curso) \\
autora.e (curso) \\
autora.e (curso)
}%

\newcommand{\gestao}{nome da gestao (\ano)}

\begin{document}

\input{_adendos/main}

%\maketitle
%\tableofcontents

%\setcounter{page}{1}

\include{0_boas_vindas/main}

\include{1_infraestrutura_unicamp/main}

\include{2_burocracias_e_estudos/main}

\chapter{Convivendo na Unicam}
\lipsum[1-5]

\section{whatever}
\lipsum[1-5]


\chapter{outro capitulo}
\lipsum[1-5]

%\chapter{Convivendo na Unicamp\\ {\color{white}. } \hspace{5mm} {\normalsize e no mundo}}
\chapter{Convivendo na Unicamp}
\lipsum[1-5]

\setcounter{section}{17}
\section{Mais secao (numero 17)}

\lipsum[1-5]

\section{E outra secao}

\lipsum[1-5]

\section{mais outra secao}

\lipsum[1-5]

\section{Eu amo secao}
\setcounter{subsection}{3}
\subsection{Algumas formas de evitar e combater o \\ machismo no dia a dia da universidade}

\lipsum[1-5]

%\chapter{Vivendo em Barão \\ {\color{white}. } \hspace{5mm} {\normalsize Barão Geraldo ou BG}}
\chapter{Vivendo em Barão Geraldo}
\lipsum[1-5]

\chapter{Além da Graduação}
\lipsum[1-5]

\chapter{aba}
\section{aoba}
oi

terminal \faTerminal, linux \faLinux, sapo \faFrog

github \faGithub, linkedin \faLinkedin, gitlab \faGitlab, email \faEnvelope, telegram \faPaperPlane[regular] \faTelegram\ \faTelegramPlane, whats \faWhatsapp

map marker \faMapMarker*, restaurantes \faUtensils, cafe \faCoffee, ra \faIdCard[regular] \faAddressCard[regular], casa \faHome

\begin{list}{\faTerminal}{}  
\item item A bla bla bla
\item item B blablabla
\end{list}

\section{Testes de links}


\instagram{meuinstagram}

\telegram{meugrupotelegram}

\discord{meugrupodiscord}

%\pagebreak

\lipsum[1-5]

\begin{tags}
    \github{meuusergithub} \sep \gitlab{meuusergitlab}    
\end{tags}

\end{document}


\chapter{Convivendo na Unicam}
\lipsum[1-5]

\section{whatever}
\lipsum[1-5]


\chapter{outro capitulo}
\lipsum[1-5]

%\chapter{Convivendo na Unicamp\\ {\color{white}. } \hspace{5mm} {\normalsize e no mundo}}
\chapter{Convivendo na Unicamp}
\lipsum[1-5]

\setcounter{section}{17}
\section{Mais secao (numero 17)}

\lipsum[1-5]

\section{E outra secao}

\lipsum[1-5]

\section{mais outra secao}

\lipsum[1-5]

\section{Eu amo secao}
\setcounter{subsection}{3}
\subsection{Algumas formas de evitar e combater o \\ machismo no dia a dia da universidade}

\lipsum[1-5]

%\chapter{Vivendo em Barão \\ {\color{white}. } \hspace{5mm} {\normalsize Barão Geraldo ou BG}}
\chapter{Vivendo em Barão Geraldo}
\lipsum[1-5]

\chapter{Além da Graduação}
\lipsum[1-5]

\chapter{aba}
\section{aoba}
oi

terminal \faTerminal, linux \faLinux, sapo \faFrog

github \faGithub, linkedin \faLinkedin, gitlab \faGitlab, email \faEnvelope, telegram \faPaperPlane[regular] \faTelegram\ \faTelegramPlane, whats \faWhatsapp

map marker \faMapMarker*, restaurantes \faUtensils, cafe \faCoffee, ra \faIdCard[regular] \faAddressCard[regular], casa \faHome

\begin{list}{\faTerminal}{}  
\item item A bla bla bla
\item item B blablabla
\end{list}

\section{Testes de links}


\instagram{meuinstagram}

\telegram{meugrupotelegram}

\discord{meugrupodiscord}

%\pagebreak

\lipsum[1-5]

\begin{tags}
    \github{meuusergithub} \sep \gitlab{meuusergitlab}    
\end{tags}

\end{document}


\documentclass{book}

\usepackage[T1]{fontenc}
\usepackage[portuguese]{babel}


\usepackage{manual_template}
% tem que decidir uma fonte legal
% https://tug.org/FontCatalogue/typewriterfonts.html

% para gerar texto lixo
\usepackage{blindtext}
\usepackage{lipsum}

\title{Manual de Ingressante}
\author{Centro Acadêmico da Computação}
\newcommand{\ano}{2024}
\date{\ano}
\newcommand{\shortauthor}{CACo}

\newcommand{\autoresCapa}{%
autor1 (EC021) \\
autor 2 (CC020)
}%

\newcommand{\autoresDiagramacao}{%
autora.e (curso) \\
autora.e (curso) \\
autora.e (curso)
}%

\newcommand{\autoresRevisao}{%
autora.e (curso) \\
autora.e (curso) \\
autora.e (curso) \\
autora.e (curso) \\
autora.e (curso) \\
autora.e (curso)
}%

\newcommand{\gestao}{nome da gestao (\ano)}

\begin{document}

\documentclass{book}

\usepackage[T1]{fontenc}
\usepackage[portuguese]{babel}


\usepackage{manual_template}
% tem que decidir uma fonte legal
% https://tug.org/FontCatalogue/typewriterfonts.html

% para gerar texto lixo
\usepackage{blindtext}
\usepackage{lipsum}

\title{Manual de Ingressante}
\author{Centro Acadêmico da Computação}
\newcommand{\ano}{2024}
\date{\ano}
\newcommand{\shortauthor}{CACo}

\newcommand{\autoresCapa}{%
autor1 (EC021) \\
autor 2 (CC020)
}%

\newcommand{\autoresDiagramacao}{%
autora.e (curso) \\
autora.e (curso) \\
autora.e (curso)
}%

\newcommand{\autoresRevisao}{%
autora.e (curso) \\
autora.e (curso) \\
autora.e (curso) \\
autora.e (curso) \\
autora.e (curso) \\
autora.e (curso)
}%

\newcommand{\gestao}{nome da gestao (\ano)}

\begin{document}

\input{_adendos/main}

%\maketitle
%\tableofcontents

%\setcounter{page}{1}

\include{0_boas_vindas/main}

\include{1_infraestrutura_unicamp/main}

\include{2_burocracias_e_estudos/main}

\chapter{Convivendo na Unicam}
\lipsum[1-5]

\section{whatever}
\lipsum[1-5]


\chapter{outro capitulo}
\lipsum[1-5]

%\chapter{Convivendo na Unicamp\\ {\color{white}. } \hspace{5mm} {\normalsize e no mundo}}
\chapter{Convivendo na Unicamp}
\lipsum[1-5]

\setcounter{section}{17}
\section{Mais secao (numero 17)}

\lipsum[1-5]

\section{E outra secao}

\lipsum[1-5]

\section{mais outra secao}

\lipsum[1-5]

\section{Eu amo secao}
\setcounter{subsection}{3}
\subsection{Algumas formas de evitar e combater o \\ machismo no dia a dia da universidade}

\lipsum[1-5]

%\chapter{Vivendo em Barão \\ {\color{white}. } \hspace{5mm} {\normalsize Barão Geraldo ou BG}}
\chapter{Vivendo em Barão Geraldo}
\lipsum[1-5]

\chapter{Além da Graduação}
\lipsum[1-5]

\chapter{aba}
\section{aoba}
oi

terminal \faTerminal, linux \faLinux, sapo \faFrog

github \faGithub, linkedin \faLinkedin, gitlab \faGitlab, email \faEnvelope, telegram \faPaperPlane[regular] \faTelegram\ \faTelegramPlane, whats \faWhatsapp

map marker \faMapMarker*, restaurantes \faUtensils, cafe \faCoffee, ra \faIdCard[regular] \faAddressCard[regular], casa \faHome

\begin{list}{\faTerminal}{}  
\item item A bla bla bla
\item item B blablabla
\end{list}

\section{Testes de links}


\instagram{meuinstagram}

\telegram{meugrupotelegram}

\discord{meugrupodiscord}

%\pagebreak

\lipsum[1-5]

\begin{tags}
    \github{meuusergithub} \sep \gitlab{meuusergitlab}    
\end{tags}

\end{document}


%\maketitle
%\tableofcontents

%\setcounter{page}{1}

\documentclass{book}

\usepackage[T1]{fontenc}
\usepackage[portuguese]{babel}


\usepackage{manual_template}
% tem que decidir uma fonte legal
% https://tug.org/FontCatalogue/typewriterfonts.html

% para gerar texto lixo
\usepackage{blindtext}
\usepackage{lipsum}

\title{Manual de Ingressante}
\author{Centro Acadêmico da Computação}
\newcommand{\ano}{2024}
\date{\ano}
\newcommand{\shortauthor}{CACo}

\newcommand{\autoresCapa}{%
autor1 (EC021) \\
autor 2 (CC020)
}%

\newcommand{\autoresDiagramacao}{%
autora.e (curso) \\
autora.e (curso) \\
autora.e (curso)
}%

\newcommand{\autoresRevisao}{%
autora.e (curso) \\
autora.e (curso) \\
autora.e (curso) \\
autora.e (curso) \\
autora.e (curso) \\
autora.e (curso)
}%

\newcommand{\gestao}{nome da gestao (\ano)}

\begin{document}

\input{_adendos/main}

%\maketitle
%\tableofcontents

%\setcounter{page}{1}

\include{0_boas_vindas/main}

\include{1_infraestrutura_unicamp/main}

\include{2_burocracias_e_estudos/main}

\chapter{Convivendo na Unicam}
\lipsum[1-5]

\section{whatever}
\lipsum[1-5]


\chapter{outro capitulo}
\lipsum[1-5]

%\chapter{Convivendo na Unicamp\\ {\color{white}. } \hspace{5mm} {\normalsize e no mundo}}
\chapter{Convivendo na Unicamp}
\lipsum[1-5]

\setcounter{section}{17}
\section{Mais secao (numero 17)}

\lipsum[1-5]

\section{E outra secao}

\lipsum[1-5]

\section{mais outra secao}

\lipsum[1-5]

\section{Eu amo secao}
\setcounter{subsection}{3}
\subsection{Algumas formas de evitar e combater o \\ machismo no dia a dia da universidade}

\lipsum[1-5]

%\chapter{Vivendo em Barão \\ {\color{white}. } \hspace{5mm} {\normalsize Barão Geraldo ou BG}}
\chapter{Vivendo em Barão Geraldo}
\lipsum[1-5]

\chapter{Além da Graduação}
\lipsum[1-5]

\chapter{aba}
\section{aoba}
oi

terminal \faTerminal, linux \faLinux, sapo \faFrog

github \faGithub, linkedin \faLinkedin, gitlab \faGitlab, email \faEnvelope, telegram \faPaperPlane[regular] \faTelegram\ \faTelegramPlane, whats \faWhatsapp

map marker \faMapMarker*, restaurantes \faUtensils, cafe \faCoffee, ra \faIdCard[regular] \faAddressCard[regular], casa \faHome

\begin{list}{\faTerminal}{}  
\item item A bla bla bla
\item item B blablabla
\end{list}

\section{Testes de links}


\instagram{meuinstagram}

\telegram{meugrupotelegram}

\discord{meugrupodiscord}

%\pagebreak

\lipsum[1-5]

\begin{tags}
    \github{meuusergithub} \sep \gitlab{meuusergitlab}    
\end{tags}

\end{document}


\documentclass{book}

\usepackage[T1]{fontenc}
\usepackage[portuguese]{babel}


\usepackage{manual_template}
% tem que decidir uma fonte legal
% https://tug.org/FontCatalogue/typewriterfonts.html

% para gerar texto lixo
\usepackage{blindtext}
\usepackage{lipsum}

\title{Manual de Ingressante}
\author{Centro Acadêmico da Computação}
\newcommand{\ano}{2024}
\date{\ano}
\newcommand{\shortauthor}{CACo}

\newcommand{\autoresCapa}{%
autor1 (EC021) \\
autor 2 (CC020)
}%

\newcommand{\autoresDiagramacao}{%
autora.e (curso) \\
autora.e (curso) \\
autora.e (curso)
}%

\newcommand{\autoresRevisao}{%
autora.e (curso) \\
autora.e (curso) \\
autora.e (curso) \\
autora.e (curso) \\
autora.e (curso) \\
autora.e (curso)
}%

\newcommand{\gestao}{nome da gestao (\ano)}

\begin{document}

\input{_adendos/main}

%\maketitle
%\tableofcontents

%\setcounter{page}{1}

\include{0_boas_vindas/main}

\include{1_infraestrutura_unicamp/main}

\include{2_burocracias_e_estudos/main}

\chapter{Convivendo na Unicam}
\lipsum[1-5]

\section{whatever}
\lipsum[1-5]


\chapter{outro capitulo}
\lipsum[1-5]

%\chapter{Convivendo na Unicamp\\ {\color{white}. } \hspace{5mm} {\normalsize e no mundo}}
\chapter{Convivendo na Unicamp}
\lipsum[1-5]

\setcounter{section}{17}
\section{Mais secao (numero 17)}

\lipsum[1-5]

\section{E outra secao}

\lipsum[1-5]

\section{mais outra secao}

\lipsum[1-5]

\section{Eu amo secao}
\setcounter{subsection}{3}
\subsection{Algumas formas de evitar e combater o \\ machismo no dia a dia da universidade}

\lipsum[1-5]

%\chapter{Vivendo em Barão \\ {\color{white}. } \hspace{5mm} {\normalsize Barão Geraldo ou BG}}
\chapter{Vivendo em Barão Geraldo}
\lipsum[1-5]

\chapter{Além da Graduação}
\lipsum[1-5]

\chapter{aba}
\section{aoba}
oi

terminal \faTerminal, linux \faLinux, sapo \faFrog

github \faGithub, linkedin \faLinkedin, gitlab \faGitlab, email \faEnvelope, telegram \faPaperPlane[regular] \faTelegram\ \faTelegramPlane, whats \faWhatsapp

map marker \faMapMarker*, restaurantes \faUtensils, cafe \faCoffee, ra \faIdCard[regular] \faAddressCard[regular], casa \faHome

\begin{list}{\faTerminal}{}  
\item item A bla bla bla
\item item B blablabla
\end{list}

\section{Testes de links}


\instagram{meuinstagram}

\telegram{meugrupotelegram}

\discord{meugrupodiscord}

%\pagebreak

\lipsum[1-5]

\begin{tags}
    \github{meuusergithub} \sep \gitlab{meuusergitlab}    
\end{tags}

\end{document}


\documentclass{book}

\usepackage[T1]{fontenc}
\usepackage[portuguese]{babel}


\usepackage{manual_template}
% tem que decidir uma fonte legal
% https://tug.org/FontCatalogue/typewriterfonts.html

% para gerar texto lixo
\usepackage{blindtext}
\usepackage{lipsum}

\title{Manual de Ingressante}
\author{Centro Acadêmico da Computação}
\newcommand{\ano}{2024}
\date{\ano}
\newcommand{\shortauthor}{CACo}

\newcommand{\autoresCapa}{%
autor1 (EC021) \\
autor 2 (CC020)
}%

\newcommand{\autoresDiagramacao}{%
autora.e (curso) \\
autora.e (curso) \\
autora.e (curso)
}%

\newcommand{\autoresRevisao}{%
autora.e (curso) \\
autora.e (curso) \\
autora.e (curso) \\
autora.e (curso) \\
autora.e (curso) \\
autora.e (curso)
}%

\newcommand{\gestao}{nome da gestao (\ano)}

\begin{document}

\input{_adendos/main}

%\maketitle
%\tableofcontents

%\setcounter{page}{1}

\include{0_boas_vindas/main}

\include{1_infraestrutura_unicamp/main}

\include{2_burocracias_e_estudos/main}

\chapter{Convivendo na Unicam}
\lipsum[1-5]

\section{whatever}
\lipsum[1-5]


\chapter{outro capitulo}
\lipsum[1-5]

%\chapter{Convivendo na Unicamp\\ {\color{white}. } \hspace{5mm} {\normalsize e no mundo}}
\chapter{Convivendo na Unicamp}
\lipsum[1-5]

\setcounter{section}{17}
\section{Mais secao (numero 17)}

\lipsum[1-5]

\section{E outra secao}

\lipsum[1-5]

\section{mais outra secao}

\lipsum[1-5]

\section{Eu amo secao}
\setcounter{subsection}{3}
\subsection{Algumas formas de evitar e combater o \\ machismo no dia a dia da universidade}

\lipsum[1-5]

%\chapter{Vivendo em Barão \\ {\color{white}. } \hspace{5mm} {\normalsize Barão Geraldo ou BG}}
\chapter{Vivendo em Barão Geraldo}
\lipsum[1-5]

\chapter{Além da Graduação}
\lipsum[1-5]

\chapter{aba}
\section{aoba}
oi

terminal \faTerminal, linux \faLinux, sapo \faFrog

github \faGithub, linkedin \faLinkedin, gitlab \faGitlab, email \faEnvelope, telegram \faPaperPlane[regular] \faTelegram\ \faTelegramPlane, whats \faWhatsapp

map marker \faMapMarker*, restaurantes \faUtensils, cafe \faCoffee, ra \faIdCard[regular] \faAddressCard[regular], casa \faHome

\begin{list}{\faTerminal}{}  
\item item A bla bla bla
\item item B blablabla
\end{list}

\section{Testes de links}


\instagram{meuinstagram}

\telegram{meugrupotelegram}

\discord{meugrupodiscord}

%\pagebreak

\lipsum[1-5]

\begin{tags}
    \github{meuusergithub} \sep \gitlab{meuusergitlab}    
\end{tags}

\end{document}


\chapter{Convivendo na Unicam}
\lipsum[1-5]

\section{whatever}
\lipsum[1-5]


\chapter{outro capitulo}
\lipsum[1-5]

%\chapter{Convivendo na Unicamp\\ {\color{white}. } \hspace{5mm} {\normalsize e no mundo}}
\chapter{Convivendo na Unicamp}
\lipsum[1-5]

\setcounter{section}{17}
\section{Mais secao (numero 17)}

\lipsum[1-5]

\section{E outra secao}

\lipsum[1-5]

\section{mais outra secao}

\lipsum[1-5]

\section{Eu amo secao}
\setcounter{subsection}{3}
\subsection{Algumas formas de evitar e combater o \\ machismo no dia a dia da universidade}

\lipsum[1-5]

%\chapter{Vivendo em Barão \\ {\color{white}. } \hspace{5mm} {\normalsize Barão Geraldo ou BG}}
\chapter{Vivendo em Barão Geraldo}
\lipsum[1-5]

\chapter{Além da Graduação}
\lipsum[1-5]

\chapter{aba}
\section{aoba}
oi

terminal \faTerminal, linux \faLinux, sapo \faFrog

github \faGithub, linkedin \faLinkedin, gitlab \faGitlab, email \faEnvelope, telegram \faPaperPlane[regular] \faTelegram\ \faTelegramPlane, whats \faWhatsapp

map marker \faMapMarker*, restaurantes \faUtensils, cafe \faCoffee, ra \faIdCard[regular] \faAddressCard[regular], casa \faHome

\begin{list}{\faTerminal}{}  
\item item A bla bla bla
\item item B blablabla
\end{list}

\section{Testes de links}


\instagram{meuinstagram}

\telegram{meugrupotelegram}

\discord{meugrupodiscord}

%\pagebreak

\lipsum[1-5]

\begin{tags}
    \github{meuusergithub} \sep \gitlab{meuusergitlab}    
\end{tags}

\end{document}


\documentclass{book}

\usepackage[T1]{fontenc}
\usepackage[portuguese]{babel}


\usepackage{manual_template}
% tem que decidir uma fonte legal
% https://tug.org/FontCatalogue/typewriterfonts.html

% para gerar texto lixo
\usepackage{blindtext}
\usepackage{lipsum}

\title{Manual de Ingressante}
\author{Centro Acadêmico da Computação}
\newcommand{\ano}{2024}
\date{\ano}
\newcommand{\shortauthor}{CACo}

\newcommand{\autoresCapa}{%
autor1 (EC021) \\
autor 2 (CC020)
}%

\newcommand{\autoresDiagramacao}{%
autora.e (curso) \\
autora.e (curso) \\
autora.e (curso)
}%

\newcommand{\autoresRevisao}{%
autora.e (curso) \\
autora.e (curso) \\
autora.e (curso) \\
autora.e (curso) \\
autora.e (curso) \\
autora.e (curso)
}%

\newcommand{\gestao}{nome da gestao (\ano)}

\begin{document}

\documentclass{book}

\usepackage[T1]{fontenc}
\usepackage[portuguese]{babel}


\usepackage{manual_template}
% tem que decidir uma fonte legal
% https://tug.org/FontCatalogue/typewriterfonts.html

% para gerar texto lixo
\usepackage{blindtext}
\usepackage{lipsum}

\title{Manual de Ingressante}
\author{Centro Acadêmico da Computação}
\newcommand{\ano}{2024}
\date{\ano}
\newcommand{\shortauthor}{CACo}

\newcommand{\autoresCapa}{%
autor1 (EC021) \\
autor 2 (CC020)
}%

\newcommand{\autoresDiagramacao}{%
autora.e (curso) \\
autora.e (curso) \\
autora.e (curso)
}%

\newcommand{\autoresRevisao}{%
autora.e (curso) \\
autora.e (curso) \\
autora.e (curso) \\
autora.e (curso) \\
autora.e (curso) \\
autora.e (curso)
}%

\newcommand{\gestao}{nome da gestao (\ano)}

\begin{document}

\input{_adendos/main}

%\maketitle
%\tableofcontents

%\setcounter{page}{1}

\include{0_boas_vindas/main}

\include{1_infraestrutura_unicamp/main}

\include{2_burocracias_e_estudos/main}

\chapter{Convivendo na Unicam}
\lipsum[1-5]

\section{whatever}
\lipsum[1-5]


\chapter{outro capitulo}
\lipsum[1-5]

%\chapter{Convivendo na Unicamp\\ {\color{white}. } \hspace{5mm} {\normalsize e no mundo}}
\chapter{Convivendo na Unicamp}
\lipsum[1-5]

\setcounter{section}{17}
\section{Mais secao (numero 17)}

\lipsum[1-5]

\section{E outra secao}

\lipsum[1-5]

\section{mais outra secao}

\lipsum[1-5]

\section{Eu amo secao}
\setcounter{subsection}{3}
\subsection{Algumas formas de evitar e combater o \\ machismo no dia a dia da universidade}

\lipsum[1-5]

%\chapter{Vivendo em Barão \\ {\color{white}. } \hspace{5mm} {\normalsize Barão Geraldo ou BG}}
\chapter{Vivendo em Barão Geraldo}
\lipsum[1-5]

\chapter{Além da Graduação}
\lipsum[1-5]

\chapter{aba}
\section{aoba}
oi

terminal \faTerminal, linux \faLinux, sapo \faFrog

github \faGithub, linkedin \faLinkedin, gitlab \faGitlab, email \faEnvelope, telegram \faPaperPlane[regular] \faTelegram\ \faTelegramPlane, whats \faWhatsapp

map marker \faMapMarker*, restaurantes \faUtensils, cafe \faCoffee, ra \faIdCard[regular] \faAddressCard[regular], casa \faHome

\begin{list}{\faTerminal}{}  
\item item A bla bla bla
\item item B blablabla
\end{list}

\section{Testes de links}


\instagram{meuinstagram}

\telegram{meugrupotelegram}

\discord{meugrupodiscord}

%\pagebreak

\lipsum[1-5]

\begin{tags}
    \github{meuusergithub} \sep \gitlab{meuusergitlab}    
\end{tags}

\end{document}


%\maketitle
%\tableofcontents

%\setcounter{page}{1}

\documentclass{book}

\usepackage[T1]{fontenc}
\usepackage[portuguese]{babel}


\usepackage{manual_template}
% tem que decidir uma fonte legal
% https://tug.org/FontCatalogue/typewriterfonts.html

% para gerar texto lixo
\usepackage{blindtext}
\usepackage{lipsum}

\title{Manual de Ingressante}
\author{Centro Acadêmico da Computação}
\newcommand{\ano}{2024}
\date{\ano}
\newcommand{\shortauthor}{CACo}

\newcommand{\autoresCapa}{%
autor1 (EC021) \\
autor 2 (CC020)
}%

\newcommand{\autoresDiagramacao}{%
autora.e (curso) \\
autora.e (curso) \\
autora.e (curso)
}%

\newcommand{\autoresRevisao}{%
autora.e (curso) \\
autora.e (curso) \\
autora.e (curso) \\
autora.e (curso) \\
autora.e (curso) \\
autora.e (curso)
}%

\newcommand{\gestao}{nome da gestao (\ano)}

\begin{document}

\input{_adendos/main}

%\maketitle
%\tableofcontents

%\setcounter{page}{1}

\include{0_boas_vindas/main}

\include{1_infraestrutura_unicamp/main}

\include{2_burocracias_e_estudos/main}

\chapter{Convivendo na Unicam}
\lipsum[1-5]

\section{whatever}
\lipsum[1-5]


\chapter{outro capitulo}
\lipsum[1-5]

%\chapter{Convivendo na Unicamp\\ {\color{white}. } \hspace{5mm} {\normalsize e no mundo}}
\chapter{Convivendo na Unicamp}
\lipsum[1-5]

\setcounter{section}{17}
\section{Mais secao (numero 17)}

\lipsum[1-5]

\section{E outra secao}

\lipsum[1-5]

\section{mais outra secao}

\lipsum[1-5]

\section{Eu amo secao}
\setcounter{subsection}{3}
\subsection{Algumas formas de evitar e combater o \\ machismo no dia a dia da universidade}

\lipsum[1-5]

%\chapter{Vivendo em Barão \\ {\color{white}. } \hspace{5mm} {\normalsize Barão Geraldo ou BG}}
\chapter{Vivendo em Barão Geraldo}
\lipsum[1-5]

\chapter{Além da Graduação}
\lipsum[1-5]

\chapter{aba}
\section{aoba}
oi

terminal \faTerminal, linux \faLinux, sapo \faFrog

github \faGithub, linkedin \faLinkedin, gitlab \faGitlab, email \faEnvelope, telegram \faPaperPlane[regular] \faTelegram\ \faTelegramPlane, whats \faWhatsapp

map marker \faMapMarker*, restaurantes \faUtensils, cafe \faCoffee, ra \faIdCard[regular] \faAddressCard[regular], casa \faHome

\begin{list}{\faTerminal}{}  
\item item A bla bla bla
\item item B blablabla
\end{list}

\section{Testes de links}


\instagram{meuinstagram}

\telegram{meugrupotelegram}

\discord{meugrupodiscord}

%\pagebreak

\lipsum[1-5]

\begin{tags}
    \github{meuusergithub} \sep \gitlab{meuusergitlab}    
\end{tags}

\end{document}


\documentclass{book}

\usepackage[T1]{fontenc}
\usepackage[portuguese]{babel}


\usepackage{manual_template}
% tem que decidir uma fonte legal
% https://tug.org/FontCatalogue/typewriterfonts.html

% para gerar texto lixo
\usepackage{blindtext}
\usepackage{lipsum}

\title{Manual de Ingressante}
\author{Centro Acadêmico da Computação}
\newcommand{\ano}{2024}
\date{\ano}
\newcommand{\shortauthor}{CACo}

\newcommand{\autoresCapa}{%
autor1 (EC021) \\
autor 2 (CC020)
}%

\newcommand{\autoresDiagramacao}{%
autora.e (curso) \\
autora.e (curso) \\
autora.e (curso)
}%

\newcommand{\autoresRevisao}{%
autora.e (curso) \\
autora.e (curso) \\
autora.e (curso) \\
autora.e (curso) \\
autora.e (curso) \\
autora.e (curso)
}%

\newcommand{\gestao}{nome da gestao (\ano)}

\begin{document}

\input{_adendos/main}

%\maketitle
%\tableofcontents

%\setcounter{page}{1}

\include{0_boas_vindas/main}

\include{1_infraestrutura_unicamp/main}

\include{2_burocracias_e_estudos/main}

\chapter{Convivendo na Unicam}
\lipsum[1-5]

\section{whatever}
\lipsum[1-5]


\chapter{outro capitulo}
\lipsum[1-5]

%\chapter{Convivendo na Unicamp\\ {\color{white}. } \hspace{5mm} {\normalsize e no mundo}}
\chapter{Convivendo na Unicamp}
\lipsum[1-5]

\setcounter{section}{17}
\section{Mais secao (numero 17)}

\lipsum[1-5]

\section{E outra secao}

\lipsum[1-5]

\section{mais outra secao}

\lipsum[1-5]

\section{Eu amo secao}
\setcounter{subsection}{3}
\subsection{Algumas formas de evitar e combater o \\ machismo no dia a dia da universidade}

\lipsum[1-5]

%\chapter{Vivendo em Barão \\ {\color{white}. } \hspace{5mm} {\normalsize Barão Geraldo ou BG}}
\chapter{Vivendo em Barão Geraldo}
\lipsum[1-5]

\chapter{Além da Graduação}
\lipsum[1-5]

\chapter{aba}
\section{aoba}
oi

terminal \faTerminal, linux \faLinux, sapo \faFrog

github \faGithub, linkedin \faLinkedin, gitlab \faGitlab, email \faEnvelope, telegram \faPaperPlane[regular] \faTelegram\ \faTelegramPlane, whats \faWhatsapp

map marker \faMapMarker*, restaurantes \faUtensils, cafe \faCoffee, ra \faIdCard[regular] \faAddressCard[regular], casa \faHome

\begin{list}{\faTerminal}{}  
\item item A bla bla bla
\item item B blablabla
\end{list}

\section{Testes de links}


\instagram{meuinstagram}

\telegram{meugrupotelegram}

\discord{meugrupodiscord}

%\pagebreak

\lipsum[1-5]

\begin{tags}
    \github{meuusergithub} \sep \gitlab{meuusergitlab}    
\end{tags}

\end{document}


\documentclass{book}

\usepackage[T1]{fontenc}
\usepackage[portuguese]{babel}


\usepackage{manual_template}
% tem que decidir uma fonte legal
% https://tug.org/FontCatalogue/typewriterfonts.html

% para gerar texto lixo
\usepackage{blindtext}
\usepackage{lipsum}

\title{Manual de Ingressante}
\author{Centro Acadêmico da Computação}
\newcommand{\ano}{2024}
\date{\ano}
\newcommand{\shortauthor}{CACo}

\newcommand{\autoresCapa}{%
autor1 (EC021) \\
autor 2 (CC020)
}%

\newcommand{\autoresDiagramacao}{%
autora.e (curso) \\
autora.e (curso) \\
autora.e (curso)
}%

\newcommand{\autoresRevisao}{%
autora.e (curso) \\
autora.e (curso) \\
autora.e (curso) \\
autora.e (curso) \\
autora.e (curso) \\
autora.e (curso)
}%

\newcommand{\gestao}{nome da gestao (\ano)}

\begin{document}

\input{_adendos/main}

%\maketitle
%\tableofcontents

%\setcounter{page}{1}

\include{0_boas_vindas/main}

\include{1_infraestrutura_unicamp/main}

\include{2_burocracias_e_estudos/main}

\chapter{Convivendo na Unicam}
\lipsum[1-5]

\section{whatever}
\lipsum[1-5]


\chapter{outro capitulo}
\lipsum[1-5]

%\chapter{Convivendo na Unicamp\\ {\color{white}. } \hspace{5mm} {\normalsize e no mundo}}
\chapter{Convivendo na Unicamp}
\lipsum[1-5]

\setcounter{section}{17}
\section{Mais secao (numero 17)}

\lipsum[1-5]

\section{E outra secao}

\lipsum[1-5]

\section{mais outra secao}

\lipsum[1-5]

\section{Eu amo secao}
\setcounter{subsection}{3}
\subsection{Algumas formas de evitar e combater o \\ machismo no dia a dia da universidade}

\lipsum[1-5]

%\chapter{Vivendo em Barão \\ {\color{white}. } \hspace{5mm} {\normalsize Barão Geraldo ou BG}}
\chapter{Vivendo em Barão Geraldo}
\lipsum[1-5]

\chapter{Além da Graduação}
\lipsum[1-5]

\chapter{aba}
\section{aoba}
oi

terminal \faTerminal, linux \faLinux, sapo \faFrog

github \faGithub, linkedin \faLinkedin, gitlab \faGitlab, email \faEnvelope, telegram \faPaperPlane[regular] \faTelegram\ \faTelegramPlane, whats \faWhatsapp

map marker \faMapMarker*, restaurantes \faUtensils, cafe \faCoffee, ra \faIdCard[regular] \faAddressCard[regular], casa \faHome

\begin{list}{\faTerminal}{}  
\item item A bla bla bla
\item item B blablabla
\end{list}

\section{Testes de links}


\instagram{meuinstagram}

\telegram{meugrupotelegram}

\discord{meugrupodiscord}

%\pagebreak

\lipsum[1-5]

\begin{tags}
    \github{meuusergithub} \sep \gitlab{meuusergitlab}    
\end{tags}

\end{document}


\chapter{Convivendo na Unicam}
\lipsum[1-5]

\section{whatever}
\lipsum[1-5]


\chapter{outro capitulo}
\lipsum[1-5]

%\chapter{Convivendo na Unicamp\\ {\color{white}. } \hspace{5mm} {\normalsize e no mundo}}
\chapter{Convivendo na Unicamp}
\lipsum[1-5]

\setcounter{section}{17}
\section{Mais secao (numero 17)}

\lipsum[1-5]

\section{E outra secao}

\lipsum[1-5]

\section{mais outra secao}

\lipsum[1-5]

\section{Eu amo secao}
\setcounter{subsection}{3}
\subsection{Algumas formas de evitar e combater o \\ machismo no dia a dia da universidade}

\lipsum[1-5]

%\chapter{Vivendo em Barão \\ {\color{white}. } \hspace{5mm} {\normalsize Barão Geraldo ou BG}}
\chapter{Vivendo em Barão Geraldo}
\lipsum[1-5]

\chapter{Além da Graduação}
\lipsum[1-5]

\chapter{aba}
\section{aoba}
oi

terminal \faTerminal, linux \faLinux, sapo \faFrog

github \faGithub, linkedin \faLinkedin, gitlab \faGitlab, email \faEnvelope, telegram \faPaperPlane[regular] \faTelegram\ \faTelegramPlane, whats \faWhatsapp

map marker \faMapMarker*, restaurantes \faUtensils, cafe \faCoffee, ra \faIdCard[regular] \faAddressCard[regular], casa \faHome

\begin{list}{\faTerminal}{}  
\item item A bla bla bla
\item item B blablabla
\end{list}

\section{Testes de links}


\instagram{meuinstagram}

\telegram{meugrupotelegram}

\discord{meugrupodiscord}

%\pagebreak

\lipsum[1-5]

\begin{tags}
    \github{meuusergithub} \sep \gitlab{meuusergitlab}    
\end{tags}

\end{document}


\chapter{Convivendo na Unicam}
\lipsum[1-5]

\section{whatever}
\lipsum[1-5]


\chapter{outro capitulo}
\lipsum[1-5]

%\chapter{Convivendo na Unicamp\\ {\color{white}. } \hspace{5mm} {\normalsize e no mundo}}
\chapter{Convivendo na Unicamp}
\lipsum[1-5]

\setcounter{section}{17}
\section{Mais secao (numero 17)}

\lipsum[1-5]

\section{E outra secao}

\lipsum[1-5]

\section{mais outra secao}

\lipsum[1-5]

\section{Eu amo secao}
\setcounter{subsection}{3}
\subsection{Algumas formas de evitar e combater o \\ machismo no dia a dia da universidade}

\lipsum[1-5]

%\chapter{Vivendo em Barão \\ {\color{white}. } \hspace{5mm} {\normalsize Barão Geraldo ou BG}}
\chapter{Vivendo em Barão Geraldo}
\lipsum[1-5]

\chapter{Além da Graduação}
\lipsum[1-5]

\chapter{aba}
\section{aoba}
oi

terminal \faTerminal, linux \faLinux, sapo \faFrog

github \faGithub, linkedin \faLinkedin, gitlab \faGitlab, email \faEnvelope, telegram \faPaperPlane[regular] \faTelegram\ \faTelegramPlane, whats \faWhatsapp

map marker \faMapMarker*, restaurantes \faUtensils, cafe \faCoffee, ra \faIdCard[regular] \faAddressCard[regular], casa \faHome

\begin{list}{\faTerminal}{}  
\item item A bla bla bla
\item item B blablabla
\end{list}

\section{Testes de links}


\instagram{meuinstagram}

\telegram{meugrupotelegram}

\discord{meugrupodiscord}

%\pagebreak

\lipsum[1-5]

\begin{tags}
    \github{meuusergithub} \sep \gitlab{meuusergitlab}    
\end{tags}

\end{document}


\chapter{Convivendo na Unicam}
\lipsum[1-5]

\section{whatever}
\lipsum[1-5]


\chapter{outro capitulo}
\lipsum[1-5]

%\chapter{Convivendo na Unicamp\\ {\color{white}. } \hspace{5mm} {\normalsize e no mundo}}
\chapter{Convivendo na Unicamp}
\lipsum[1-5]

\setcounter{section}{17}
\section{Mais secao (numero 17)}

\lipsum[1-5]

\section{E outra secao}

\lipsum[1-5]

\section{mais outra secao}

\lipsum[1-5]

\section{Eu amo secao}
\setcounter{subsection}{3}
\subsection{Algumas formas de evitar e combater o \\ machismo no dia a dia da universidade}

\lipsum[1-5]

%\chapter{Vivendo em Barão \\ {\color{white}. } \hspace{5mm} {\normalsize Barão Geraldo ou BG}}
\chapter{Vivendo em Barão Geraldo}
\lipsum[1-5]

\chapter{Além da Graduação}
\lipsum[1-5]

\chapter{aba}
\section{aoba}
oi

terminal \faTerminal, linux \faLinux, sapo \faFrog

github \faGithub, linkedin \faLinkedin, gitlab \faGitlab, email \faEnvelope, telegram \faPaperPlane[regular] \faTelegram\ \faTelegramPlane, whats \faWhatsapp

map marker \faMapMarker*, restaurantes \faUtensils, cafe \faCoffee, ra \faIdCard[regular] \faAddressCard[regular], casa \faHome

\begin{list}{\faTerminal}{}  
\item item A bla bla bla
\item item B blablabla
\end{list}

\section{Testes de links}


\instagram{meuinstagram}

\telegram{meugrupotelegram}

\discord{meugrupodiscord}

%\pagebreak

\lipsum[1-5]

\begin{tags}
    \github{meuusergithub} \sep \gitlab{meuusergitlab}    
\end{tags}

\end{document}
