\chapter{Olá, Mundo!}

\setcounter{page}{1}

\section{Prefácio}

Bem-vinde!

Este é o Manual de Ingressante do CACo, o Centro Acadêmico da Computação. Esse manual foi elaborado por diverses veteranes e inclui várias dicas para auxiliar a sua sobrevivência nestes primeiros anos na Unicamp. Você deve receber a versão impressa dele nos primeiros dias ou semanas de aula! Se ainda não a pegou, entre em contato com alguém do CACo.

Nós trabalhamos muito para que as informações aqui presentes sejam úteis e atuais; e queremos ouvir de você seu feedback sobre este material! Se você encontrar quaisquer informações incorretas ou desatualizada, nos avise por e-mail ou por mensagem no Instagram.

\begin{tags}
    \email{caco.gestao@gmail.com} \sep \instagram{cacounicamp}    
\end{tags}

\subsection{Sobre o gênero dos substantivos neste manual}

Você, ex-vestibulande, humane do século XXI, certamente está antenade na lutas sociais que acontecem à sua volta e uma delas é a da igualdade de gêneros. Ela se baseia no respeito e inclusão de todos os gêneros, de forma que todes tenham os mesmos direitos em nossa sociedade - inclusive o direito a ser corretamente representade.

Infelizmente, nosso idioma não foi feito levando em conta a igualdade de gênero. Isso acabou gerando discrepâncias como a regra irracional e machista de que devemos nos referir a um grupo de 100 mulheres e apenas um homem utilizando o pronome "eles". Por esse motivo, utilizamos em nosso manual uma linguagem diferente, centrada nos artigos neutros.

Como estamos acostumados ao português padrão e este manual passa por constantes atualizações, talvez você encontre lugares onde o texto ainda não utiliza uma linguagem neutra. Se isso acontecer, nos envie uma solicitação para alterarmos o texto e nos ajude a deixar o manual cada vez melhor para todo mundo.

Vale enfatizar que fazemos isso porque valorizamos as minorias presentes em nossa comunidade e deixamos a seguinte mensagem para todes es seus representantes: \textbf{Nós, o seu Centro Acadêmico, estamos aqui para você!}

\section{Mensagem da FEEC}

<mensagem da FEEC>


\section{Mensagem do IC}

<mensagem do ic>

\section{Os melhores anos da sua vida!}

Parabéns!

Entrar na Universidade e cursar o ensino superior é uma grande conquista e pode lhe trazer algumas das experiências mais interessantes da sua vida. Uma delas é a oportunidade de se especializar em uma área do conhecimento e aprender com cientistas e profissionais que sabem muito bem - ou até mesmo descobriram - os assuntos que estão ensinando.

Nunca é tarde demais para estudar, mas muites de nós, quando temos essa oportunidade, a agarramos ainda jovens. E assim é comum que muitas coisas se misturem aos desafios da vida acadêmica: a entrada no mundo adulto, relações marcantes entre colegas e amigues, dezenas de convites para festas por mês, responsabilidades crescentes e as famosas contas pra pagar.

Enfim, o peso das novas responsabilidades e dos estudos, assim como as deficiências e exigências da própria universidade podem impactar nossa saúde mental - um tema que está sendo cada vez mais discutido e que incluímos neste manual para você poder começar o curso mais tranquile.

Mas muito mais do que estudar, aqui você vai encontrar toda uma comunidade realizando uma serie de atividades extras incríveis. Atividades para todos os gostos: de grupos de estudo a redes de apoio, times de esportes e e-sports, atléticas, empresas juniores e nós o seu centro acadêmico. Essas experiências elevam o significado dos anos na universidade a muito mais do que aulas, provas, livros e exercícios. E, através delas, você pode encontrar amizades pra toda a vida!

Entrar numa universidade internacionalmente reconhecida, como a Unicamp, tem ainda mais desdobramentos. Suas atividades cotidianas poderão ter um impacto tremendamente grande lá fora!

Primeiro, porque nossa universidade é pública. Isso significa que você é ume des pouques que a população brasileira conseguiu colocar pra dentro de um ensino superior de qualidade que não apenas ensina uma nova profissão, mas que também produz ciência. De fato, as universidades públicas produzem 95\% da ciência do nosso país \footnote{buscar a referência}. Não por acaso, a poucos quilômetros do campus da Unicamp fica o Sírius, um acelerador de partículas de padrão mundial! Por esses motivos, pelo nosso trabalho científico e de nossos colegas, podemos nos orgulhar do retorno que damos à sociedade, fazendo o peso do nome Unicamp valer muito mais do que um diploma na parede.

Segundo, porque as universidades públicas estão em disputa e sofrendo ataques, com constantes tentativas de descredibilização e precarização. Houve cortes de verbas bilionários que afetaram as universidades federais por motivos de "balbúrdia", de acordo com o ex-ministro da Educação, Abraham Weintraub. Recentemente, após a pandemia e o
aprofundamento da crise econômica, os ataques têm sido cada vez mais intensos e o ensino superior tem sofrido de uma crise orçamentária em níveis nacionais.

A Unicamp não é uma bolha: esses ataques tem consequências aqui. Sofremos cortes de verbas; há muita terceirização, diminuindo a qualidade do serviço e as condições de trabalho e sustento dos funcionários; faltam vagas na moradia estudantil e bolsas para estudantes mais pobres. A iniciativa privada age predatoriamente em prol desse desmonte e se infiltram cada vez mais fundo na universidade, controlando quais pesquisas serão realizadas ou não em prol dos seus próprios interesses - diminuindo seus gastos através do uso da infraestrutura paga com dinheiro público. Isso impede a democratização do conhecimento e prejudica inclusive quem pesquisa, que não só se torna refém como passa a ter uma empresa decidindo o que pode ou não ser publicado. Essas são questões fundamentais que você vai conhecer mais adiante neste manual e no
seu dia a dia.

A realidade é que a universidade pública precisa que você a defenda como parte de sua jornada por aqui. A ciência e a educação são grandes patrimônios do humanidade e somos nós os responsáveis por zelar por elas. Não desperdice a sua chance de cumprir esse papel!

Por tudo isso, bem-vinde! Os melhores anos da sua vida serão o que você fizer que eles sejam. Você tem um longo caminho a trilhar agora e vai se deparar com muitos desafios. Então, não se esqueça: você sempre pode contar com a gente.
