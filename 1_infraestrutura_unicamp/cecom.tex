\section{Atendimento médico e odontológico - CECOM}

O CECOM é o Centro de Saúde da Comunidade e oferece atendimento ambulatorial
gratuito aos seus alunos em diversas especialidades médicas, saúde mental, fisioterapia,
nutrição, enfermagem e odontologia..

A CSS, Coordenadoria de Serviços Sociais, é responsável pelo planejamento e
execução de programas de saúde voltados à comunidade universitária da Unicamp –
alunes, funcionáries e docentes.

Na prática é quase como um plano de saúde da Unicamp. Você pode marcar
consultas médicas e fazer exames. O CECOM fica próximo ao HC (Hospital das Clínicas)
e para chega lá você pode pegar o circular interno, é o ponto final de algumas linhas.

Fique de olho na seu e-mail institucional perto do período de inverno, pois \textbf{o CECOM
costuma disponibilizar vacina contra gripe gratuitamente} - uma dose em clínica
particular é bem cara. O CECOM também oferece vacinas contra COVID, que são
obrigatórias e se você não estiver com elas em dia pode ter problemas para se
matricular.

Outra coisa que você não pode perder no CECOM são os testes de ISTs (infecções
sexualmente transmissíveis). O teste cobre várias ISTs, inclusive o HIV, e o resultado sai
na hora. É necessário agendar, exceto se você tiver tido exposição a alguma IST.

Para marcar consultas com dentista, vá ao CECOM e peça informações. Basicamente é
embaixo do CECOM, muito fácil de entender como chegar se alguém apontar com e
dedo e disser "ali". O atendimento funciona muito bem e já começa com uma palestra
instruindo como cuidar bem da boca no dia a dia. Afinal, a melhor saúde é a preventiva.
Mas se você estiver com dores, te atendem na hora sem marcar consulta nem assistir
palestra.

Veja todas as informações sobre atendimento e contato no site do CECOM:
\url{www.cecom.unicamp.br}.