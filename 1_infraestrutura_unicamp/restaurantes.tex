\section{O que comer na Unicamp}

\subsection{Histórico do Bandejão}

A universidade, após sua fundação, passou 10 anos sem ter bandejão. O primeiro foi o
RU (Restaurante Universitário); em seguida, um prédio que pertencia ao CABS (Centro
Acadêmico Bernardo Sayão, da Engenharia Elétrica) foi convertido no RA (Restaurante
Administrativo), perto da FEEC; depois foi construído o RS (Restaurante Saturnino), mais
moderno e perto do IC. Hoje todos os campi da universidade têm acesso a bandejões,
inclusive os colégios técnicos (COTUCA em Campinas e COTIL em Limeira).

O preço das refeições é subsidiado e, sem ele, cada estudante teria que viver na base
da marmita, passando maus bocados para ter tempo para cozinhar ou pagar o preço
comercial de todas as refeições.

O Bandejão começou a servir café da manhã a pouco tempo e é uma reivindicação
quase tão antiga quanto a abertura nos finais de semana - uma pauta importantíssima
para quem não pode voltar pra casa da família nos fins de semana e tem que se virar
por aqui, mas que ainda não conseguimos conquistar.

\subsection{Sobre o Bandejão em si – ou Bandeco, Bandex, RU, RA, RS, etc!}

Um dos momentos de glória do dia de ume future engenheire, cientista ou bacharel é
o Bandejão. É a hora de intensas e indiscutíveis emoções. Jamais tente descobrir o sabor
do suco pelo paladar, é mais fácil ler no cardápio do dia. O Bandejão é o lugar de você se
encontrar com amigues (combinando antes ou não), contar os micos nas aulas e jogar
conversa fora.

Eles funcionam de segunda à sexta e são bem afastados um do outro, o que torna
qualquer instituto razoavelmente perto de um bandeco, independente de onde tenha
sido ou vai ser sua aula. Em períodos especiais, como férias, os restaurantes podem
funcionar em horários reduzidos ou não abrirem, então fique de olho no site da
prefeitura, no GDE, ou principalmente no aplicativo da Unicamp. Nessas plataformas
você também pode encontrar o cardápio da semana, que conta com a famigerada carne
picada pelo menos um dia.

Atualmente, o preço do café da manhã é de R\$ 2,00 e do almoço e jantar é de R\$
3,00. Sem dúvida ainda é um ótimo custo benefício, porém ele já foi melhor:
historicamente, o preço do bandejão era de R\$ 2,00. No entanto, entre 2017 e 2018,
numa reunião extraordinária do Conselho Universitário, foi aprovado o aumento das
refeições dos restaurantes universitários. Por mais que a inflação tenha aumentado, as
bolsas estudantis não tiveram um reajuste equivalente.

Existem outras opções para se alimentar dentro da Unicamp. Contudo, elas
certamente custam mais do que R\$ 3,00.

\subsection{Cantinas}

As cantinas oferecem algo mais voltado para o "lanche da tarde", ao invés de
refeições completas, como salgados e bolos. No entanto, existem algumas com pratos
feitos ou com self-service por quilo, saindo bem mais caro que o bandejão. Elas eram
bem numerosas no passado, porém conforme suas licenças de funcionamento foram
expiran-do, a universidade não abriu novos editais para renová-las e ainda não se
pronunciou a respeito. Por conta disso, nem todos os institutos tem cantinas e não
sabemos ao certo quantas cantinas estarão funcionando nesse ano. O mesmo fenômeno
ocorre com as barraquinhas de impressão e xerox dos institutos.

\subsection{Feirinha}

Carinhosamente apelidada de feirinha, é uma feira que ocorre na praça do Ciclo
Básico às quartas e quintas (quando temos feriados, os dias mudam). A variedade de
opções é imensa: existem barracas de suco, tapioca, macarrão, açaí, pastel, doces,
strogonoff, entre muitas outras – além de produtores de artesanato. Também há
restaurantes e supermercados em Barão, veja na seção Vivendo em Barão.